\def\module{M3P21 Geometry II: Algebraic Topology}
\def\lecturer{Dr Christian Urech}
\def\term{Spring 2019}
\def\cover{
$$
\begin{tikzcd}[ampersand replacement=\&, row sep=tiny]
\& \& \& 0 \arrow{dd} \& \& 0 \arrow{dd} \& \& 0 \arrow{dd} \& \& \\
\& \& 0 \& \& 0 \& \& 0 \& \& \& \\
\& \dots \arrow[near start]{rr}{\partial} \& \& A_{n + 1}' \arrow[near start]{rr}{\partial} \arrow[near start]{dd}{i'} \& \& A_n' \arrow[near start]{rr}{\partial} \arrow[near start]{dd}{i'} \& \& A_{n - 1}' \arrow[near start]{rr}{\partial} \arrow[near start]{dd}{i'} \& \& \dots \\
\dots \arrow[near start]{rr}{\partial} \& \& A_{n + 1} \arrow[from=uu, crossing over, near start] \arrow{ur}{\alpha} \arrow[crossing over, near start]{rr}{\partial} \& \& A_n \arrow[from=uu, crossing over, near start] \arrow{ur}{\alpha} \arrow[crossing over, near start]{rr}{\partial} \& \& A_{n - 1} \arrow[from=uu, crossing over, near start] \arrow{ur}{\alpha} \arrow[crossing over, near start]{rr}{\partial} \& \& \dots \& \\
\& \dots \arrow[near start]{rr}{\partial} \& \& B_{n + 1}' \arrow[near start]{rr}{\partial} \arrow[near start]{dd}{j'} \& \& B_n' \arrow[near start]{rr}{\partial} \arrow[near start]{dd}{j'} \& \& B_{n - 1}' \arrow[near start]{rr}{\partial} \arrow[near start]{dd}{j'} \& \& \dots \\
\dots \arrow[near start]{rr}{\partial} \& \& B_{n + 1} \arrow[from=uu, crossing over, near start]{}{i} \arrow{ur}{\beta} \arrow[crossing over, near start]{rr}{\partial} \& \& B_n \arrow[from=uu, crossing over, near start]{}{i} \arrow{ur}{\beta} \arrow[crossing over, near start]{rr}{\partial} \& \& B_{n - 1} \arrow[from=uu, crossing over, near start]{}{i} \arrow{ur}{\beta} \arrow[crossing over, near start]{rr}{\partial} \& \& \dots \& \\
\& \dots \arrow[near start]{rr}{\partial} \& \& C_{n + 1}' \arrow[near start]{rr}{\partial} \arrow[near start]{dd} \& \& C_n' \arrow[near start]{rr}{\partial} \arrow[near start]{dd} \& \& C_{n - 1}' \arrow[near start]{rr}{\partial} \arrow[near start]{dd} \& \& \dots \\
\dots \arrow[near start]{rr}{\partial} \& \& C_{n + 1} \arrow[from=uu, crossing over, near start]{}{j} \arrow{ur}{\gamma} \arrow[crossing over, near start]{rr}{\partial} \& \& C_n \arrow[from=uu, crossing over, near start]{}{j} \arrow{ur}{\gamma} \arrow[crossing over, near start]{rr}{\partial} \& \& C_{n - 1} \arrow[from=uu, crossing over, near start]{}{j} \arrow{ur}{\gamma} \arrow[crossing over, near start]{rr}{\partial} \& \& \dots \& \\
\& \& \& 0 \& \& 0 \& \& 0 \& \& \\
\& \& 0 \arrow[from=uu] \& \& 0 \arrow[from=uu] \& \& 0 \arrow[from=uu] \& \& \&
\end{tikzcd}
$$
$$
\Downarrow
$$
$$
\begin{tikzcd}[ampersand replacement=\&, row sep=small]
\dots \arrow{r}{\partial} \& \H_{n + 1}\br{A} \arrow{r}{i_*} \arrow{d}{\alpha_*} \& \H_{n + 1}\br{B} \arrow{r}{j_*} \arrow{d}{\beta_*} \& \H_{n + 1}\br{C} \arrow{d}{\gamma_*} \arrow{ddl}{\partial} \& \& \& \\
\dots \arrow{r}{\partial} \& \H_{n + 1}\br{A'} \arrow{r}{i_*} \& \H_{n + 1}\br{B'} \arrow{r}{j_*} \& \H_{n + 1}\br{C'} \arrow{ddl}{\partial} \& \& \& \\
\& \& \H_n\br{A} \arrow{r}{i_*} \arrow{d}{\alpha_*} \& \H_n\br{B} \arrow{r}{j_*} \arrow{d}{\beta_*} \& \H_n\br{C} \arrow{d}{\gamma_*} \arrow{ddl}{\partial} \& \& \\
\& \& \H_n\br{A'} \arrow{r}{i_*} \& \H_n\br{B'} \arrow{r}{j_*} \& \H_n\br{C'} \arrow{ddl}{\partial} \& \& \\
\& \& \& \H_{n - 1}\br{A} \arrow{r}{i_*} \arrow{d}{\alpha_*} \& \H_{n - 1}\br{B} \arrow{r}{j_*} \arrow{d}{\beta_*} \& \H_{n - 1}\br{C} \arrow{r}{\partial} \arrow{d}{\gamma_*} \& \dots \\
\& \& \& \H_{n - 1}\br{A'} \arrow{r}{i_*} \& \H_{n - 1}\br{B'} \arrow{r}{j_*} \& \H_{n - 1}\br{C'} \arrow{r}{\partial} \& \dots
\end{tikzcd}
$$
}
\def\syllabus{Homotopy and homotopy type. Cell complexes. Basic constructions of the fundamental group. Seifert-van Kampen theorem. Covering spaces. $ \Delta $-complexes. Simplicial homology. Singular homology. Homotopy invariance. Exact sequences and excision. Mayer-Vietoris sequences. Degree.}
\def\thm{section}

\documentclass{article}

% Packages

\usepackage{amssymb}
\usepackage{amsthm}
\usepackage[UKenglish]{babel}
\usepackage{commath}
\usepackage{enumitem}
\usepackage{fancyhdr}
\usepackage[margin=1in]{geometry}
\usepackage{graphicx}
\usepackage[hidelinks]{hyperref}
\usepackage[utf8]{inputenc}
\usepackage{listings}
\usepackage{mathtools}
\usepackage{stmaryrd}
\usepackage{tikz-cd}
\usepackage{csquotes}

% Formatting

\addto\captionsUKenglish{\renewcommand{\abstractname}{Syllabus}}
\delimitershortfall5pt
\ifx\thm\undefined\newtheorem{n}{}\else\newtheorem{n}{}[\thm]\fi
\let\g\gg
\let\l\ll
\setlength{\parindent}{0cm}

% Environments

\theoremstyle{plain}
\newtheorem{algorithm}[n]{Algorithm}
\newtheorem{conjecture}[n]{Conjecture}
\newtheorem{corollary}[n]{Corollary}
\newtheorem{lemma}[n]{Lemma}
\newtheorem{proposition}[n]{Proposition}
\newtheorem{theorem}[n]{Theorem}

\theoremstyle{definition}
\newtheorem{axiom}{Axiom}
\newtheorem{definition}[n]{Definition}
\newtheorem{example}[n]{Example}
\newtheorem{exercise}{Exercise}
\newtheorem*{definition*}{Definition}
\newtheorem*{example*}{Example}

\theoremstyle{remark}
\newtheorem{notation}[n]{Notation}
\newtheorem{remark}[n]{Remark}
\newtheorem*{fact*}{Fact}
\newtheorem*{notation*}{Notation}
\newtheorem*{note*}{Note}
\newtheorem*{remark*}{Remark}

% Commands

\newcommand{\lecture}[3]{ % Lecture
  \marginpar{
    Lecture #1 \\
    #2 \\
    #3
  }
}

\renewcommand{\eval}[1]{\left. #1 \right|}          % Evaluation
\newcommand{\br}{\del}                              % Brackets
\newcommand{\abr}[1]{\left\langle #1 \right\rangle} % Angle brackets
\newcommand{\fbr}[1]{\left\lfloor #1 \right\rfloor} % Floor brackets
\newcommand{\lbr}[1]{\left\lfloor #1 \right\rfloor} % Ceiling brackets

\newcommand{\function}[5][]{ % Function
  \ifx &#1&
    \begin{array}{rcl}
      #2 & \longrightarrow & #3 \\
      #4 & \longmapsto     & #5
    \end{array}
  \else
    \fullfunction{#1}{#2}{#3}{#4}{#5}
  \fi
}

\newcommand{\F}{\mathbb{F}}   % Finite fields
\newcommand{\N}{\mathbb{N}}   % Natural numbers
\newcommand{\Z}{\mathbb{Z}}   % Integral numbers
\newcommand{\Q}{\mathbb{Q}}   % Rational numbers
\newcommand{\R}{\mathbb{R}}   % Real numbers
\newcommand{\C}{\mathbb{C}}   % Complex numbers
\renewcommand{\H}{\mathbb{H}} % Quaternion numbers
\newcommand{\A}{\mathbb{A}}   % Affine spaces
\renewcommand{\P}{\mathbb{P}} % Projective spaces

\newcommand{\correspondence}[2]{ % Correspondence
  \cbr{
    \begin{array}{c}
      #1
    \end{array}
  }
  \qquad
  \leftrightsquigarrow
  \qquad
  \cbr{
    \begin{array}{c}
      #2
    \end{array}
  }
}

\newcommand{\intd}[4]{\int_{#1}^{#2} \, #3 \, \dif #4}                     % Single integral
\newcommand{\iintd}[4]{\iint_{#1} \, #2 \, \dif #3 \, \dif #4}             % Double integral
\newcommand{\iiintd}[5]{\iint_{#1} \, #2 \, \dif #3 \, \dif #4 \, \dif #5} % Triple integral

\newcommand{\onebytwo}[2]{     % One by two matrix
  \begin{pmatrix}
    #1 & #2
  \end{pmatrix}
}
\newcommand{\onebythree}[3]{   % One by three matrix
  \begin{pmatrix}
    #1 & #2 & #3
  \end{pmatrix}
}
\newcommand{\twobyone}[2]{     % Two by one matrix
  \begin{pmatrix}
    #1 \\
    #2
  \end{pmatrix}
}
\newcommand{\twobytwo}[4]{     % Two by two matrix
  \begin{pmatrix}
    #1 & #2 \\
    #3 & #4
  \end{pmatrix}
}
\newcommand{\threebyone}[3]{   % Three by one matrix
  \begin{pmatrix}
    #1 \\
    #2 \\
    #3
  \end{pmatrix}
}
\newcommand{\threebythree}[9]{ % Three by three matrix
  \begin{pmatrix}
    #1 & #2 & #3 \\
    #4 & #5 & #6 \\
    #7 & #8 & #9
  \end{pmatrix}
}

\renewcommand{\aa}{\mathfrak{a}} % Fraktur a
\newcommand{\bb}{\mathfrak{b}}   % Fraktur b
\newcommand{\cc}{\mathfrak{c}}   % Fraktur c
\newcommand{\dd}{\mathfrak{d}}   % Fraktur d
\newcommand{\ee}{\mathfrak{e}}   % Fraktur e
\newcommand{\ff}{\mathfrak{f}}   % Fraktur f
\renewcommand{\gg}{\mathfrak{g}} % Fraktur g
\newcommand{\hh}{\mathfrak{h}}   % Fraktur h
\newcommand{\ii}{\mathfrak{i}}   % Fraktur i
\newcommand{\jj}{\mathfrak{j}}   % Fraktur j
\newcommand{\kk}{\mathfrak{k}}   % Fraktur k
\renewcommand{\ll}{\mathfrak{l}} % Fraktur l
\newcommand{\mm}{\mathfrak{m}}   % Fraktur m
\newcommand{\nn}{\mathfrak{n}}   % Fraktur n
\newcommand{\oo}{\mathfrak{o}}   % Fraktur o
\newcommand{\pp}{\mathfrak{p}}   % Fraktur p
\newcommand{\qq}{\mathfrak{q}}   % Fraktur q
\newcommand{\rr}{\mathfrak{r}}   % Fraktur r
\renewcommand{\ss}{\mathfrak{s}} % Fraktur s
\renewcommand{\tt}{\mathfrak{t}} % Fraktur t
\newcommand{\uu}{\mathfrak{u}}   % Fraktur u
\newcommand{\vv}{\mathfrak{v}}   % Fraktur v
\newcommand{\ww}{\mathfrak{w}}   % Fraktur w
\newcommand{\xx}{\mathfrak{x}}   % Fraktur x
\newcommand{\yy}{\mathfrak{y}}   % Fraktur y
\newcommand{\zz}{\mathfrak{z}}   % Fraktur z

\renewcommand{\AA}{\mathcal{A}} % Calligraphic A
\newcommand{\BB}{\mathcal{B}}   % Calligraphic B
\newcommand{\CC}{\mathcal{C}}   % Calligraphic C
\newcommand{\DD}{\mathcal{D}}   % Calligraphic D
\newcommand{\EE}{\mathcal{E}}   % Calligraphic E
\newcommand{\FF}{\mathcal{F}}   % Calligraphic F
\newcommand{\GG}{\mathcal{G}}   % Calligraphic G
\newcommand{\HH}{\mathcal{H}}   % Calligraphic H
\newcommand{\II}{\mathcal{I}}   % Calligraphic I
\newcommand{\JJ}{\mathcal{J}}   % Calligraphic J
\newcommand{\KK}{\mathcal{K}}   % Calligraphic K
\newcommand{\LL}{\mathcal{L}}   % Calligraphic L
\newcommand{\MM}{\mathcal{M}}   % Calligraphic M
\newcommand{\NN}{\mathcal{N}}   % Calligraphic N
\newcommand{\OO}{\mathcal{O}}   % Calligraphic O
\newcommand{\PP}{\mathcal{P}}   % Calligraphic P
\newcommand{\QQ}{\mathcal{Q}}   % Calligraphic Q
\newcommand{\RR}{\mathcal{R}}   % Calligraphic R
\renewcommand{\SS}{\mathcal{S}} % Calligraphic S
\newcommand{\TT}{\mathcal{T}}   % Calligraphic T
\newcommand{\UU}{\mathcal{U}}   % Calligraphic U
\newcommand{\VV}{\mathcal{V}}   % Calligraphic V
\newcommand{\WW}{\mathcal{W}}   % Calligraphic W
\newcommand{\XX}{\mathcal{X}}   % Calligraphic X
\newcommand{\YY}{\mathcal{Y}}   % Calligraphic Y
\newcommand{\ZZ}{\mathcal{Z}}   % Calligraphic Z

\newcommand{\notb}[1]{\br{\neg #1}}               % Negation
\newcommand{\orb}[2]{\br{#1 \lor #2}}             % Disjunction
\newcommand{\andb}[2]{\br{#1 \land #2}}           % Conjunction
\newcommand{\impb}[2]{\br{#1 \rightarrow #2}}     % Implication
\newcommand{\iffb}[2]{\br{#1 \leftrightarrow #2}} % Biconditional
\newcommand{\fab}[1]{\br{\forall #1}}             % Universal quantifier
\newcommand{\teb}[1]{\br{\exists #1}}             % Existential quantifier
\newcommand{\eqb}[2]{\br{#1 = #2}}                % Equal to
\newcommand{\ltb}[2]{\br{#1 < #2}}                % Less than
\newcommand{\leb}[2]{\br{#1 \le #2}}              % Less than or equal to
\newcommand{\neb}[2]{\br{#1 \ne #2}}              % Not equal to
\newcommand{\inb}[2]{\br{#1 \in #2}}              % An element of
\newcommand{\nib}[2]{\br{#1 \notin #2}}           % Not an element of
\newcommand{\subb}[2]{\br{#1 \subseteq #2}}       % Subset of

\newcommand{\jacobi}[2]{\br{\tfrac{#1}{#2}}}        % Jacobi symbol
\newcommand{\Unit}[1]{\br{\dfrac{\Z}{#1\Z}}^\times} % Unit group vertical
\newcommand{\unit}[1]{\br{\Z / #1\Z}^\times}        % Unit group horizontal

% Tikz

\tikzset{
  arrow symbol/.style={"#1" description, allow upside down, auto=false, draw=none, sloped},
  subset/.style={arrow symbol={\subset}},
  cong/.style={arrow symbol={\cong}}
}

% Fancy header

\pagestyle{fancy}
\lhead{\module}
\rhead{\nouppercase{\leftmark}}

% Make title

\title{\module}
\author{Lectured by \lecturer \\ Typed by David Kurniadi Angdinata}
\date{\term}

\begin{document}

% Title page
\maketitle
\cover
\vfill
\begin{abstract}
\noindent\syllabus
\end{abstract}

\pagebreak

% Contents page
\tableofcontents

\pagebreak

% Document page
\setcounter{section}{-1}

\section{Introduction}

\subsection{Introduction}

\lecture{1}{Friday}{11/01/19}

Combines topological spaces with algebraic objects, which are groups.
\begin{itemize}
\item How to show that a torus is not homeomorphic to a sphere?
\item How to show that $ \RR^n \ncong \RR^m $ if $ n \ne m $?
\end{itemize}
We will follow chapter one and two from
\begin{itemize}
\item A Hatcher, Algebraic topology, 2002
\end{itemize}
The following are prerequisites.
\begin{itemize}
\item Point set topology. Topological spaces, continuous maps, product and quotient topologies, Hausdorff spaces, etc.
\item Basic group theory. Normal subgroups and quotients, isomorphism theorems, free groups, presentation of groups, etc.
\end{itemize}

\subsection{Some underlying geometric notions}

\subsubsection{Homotopy and homotopy type}

Let $ X $ and $ Y $ be topological spaces and $ \I = \sbr{0, 1} $.

\begin{definition*}
A \textbf{homotopy} is a continuous map $ F : X \times \I \to Y $. For every $ t \in \I $ we obtain a continuous map
$$ \function[f_t]{X}{Y}{x}{f_t\br{x} = F\br{x, t}}. $$
\end{definition*}

\begin{definition*}
Two continuous maps $ f_0, f_1 : X \to Y $ are \textbf{homotopic} if there exists a homotopy $ F : X \times \I \to Y $ such that
$$ f_0\br{x} = F\br{x, 0}, \qquad f_1\br{x} = F\br{x, 1}, \qquad x \in X. $$
We write $ f_0 \cong f_1 $. This is an equivalence relation. \footnote{Exercise}
\end{definition*}

\begin{definition*}
Let $ A \subseteq X $ be a subspace. A \textbf{retraction} of $ X $ onto $ A $ is a continuous map $ r : X \to A $ such that $ r\br{X} = A $ and $ \eval{r}_A = \id_A $.
\end{definition*}

\begin{example*}
If $ X \ne \emptyset $, $ p \in X $, then $ X $ retracts to $ p $ by the constant map $ X \to \cbr{p} $.
\end{example*}

\begin{definition*}
A \textbf{deformation retraction} of $ X $ onto $ A \subseteq X $ is a retraction that is homotopic to the identity. That is, there is a continuous map
$$ \function[F]{X \times \I}{A}{\br{x, t}}{f_t\br{x}}, $$
such that $ f_0 = \id_X $ and $ f_1 : X \to A $ is the deformation retraction.
\end{definition*}

\begin{example*}
The closed $ n $-dimensional \textbf{$ n $-disc}
$$ \D^n = \cbr{x \in \RR^n \st \abs{x} \le 1} $$
deformation retracts to $ \br{0, \dots, 0} \in \RR^n $. Let $ f_t\br{x} = t \cdot x $. Then $ t = 1 $ implies that $ f_1 = \id_{\D^n} $ and $ t = 0 $ implies that $ f_0 : \D^n \to \br{0, \dots, 0} $.
\end{example*}

\pagebreak

\begin{example*}
Let $ \S^n $ be the \textbf{$ n $-sphere},
$$ \partial \D^{n + 1} = \S^n = \cbr{x \in \RR^n \st \abs{x} = 1}. $$
The cylinder $ \S^n \times \I $ deformation retracts to $ \S^n \times \cbr{0} $, by defining $ f_t\br{x, r} = \br{x, t \cdot r} $.
\end{example*}

An observation is that if $ X $ is a topological space, and $ f : X \to \cbr{p} $ for $ p \in X $ is a deformation retraction of $ X $ to $ p $, then $ X $ is path-connected. Indeed, if $ F : X \times \I \to X $ is a homotopy from $ \id_X $ to $ f $ and $ x \in X $ is a point, then this gives a path
$$ \function{\I}{X}{t}{F\br{x, t}} $$
that connects $ x $ to $ p $. This implies that not all retractions are deformation retractions.

\begin{example*}
A retraction that is not a deformation retraction. Take a space that is not path-connected and retract it to a point. Let $ X = \cbr{0, 1} $ with discrete topology. Then $ x \mapsto 0 $ is a retraction, but not a deformation retraction because $ X $ is not path-connected.
\end{example*}

\begin{definition*}
A continuous map $ f : X \to Y $ is a \textbf{homotopy equivalence} if there is a continuous map $ g : Y \to X $ such that $ fg \cong \id_Y $ and $ gf \cong \id_X $. If there exists a homotopy equivalence between $ X $ and $ Y $, $ X $ and $ Y $ are \textbf{homotopy equivalent} or they have the same \textbf{homotopy type}.
\end{definition*}

\begin{lemma}
A deformation retraction $ f : X \to A $ is a homotopy equivalence.
\end{lemma}

\begin{proof}
Let $ i : A \hookrightarrow X $ be the inclusion map. Then $ fi = \id_A $ and $ if = f \cong \id_X $ by definition.
\end{proof}

\begin{example*}
The disc with two holes is equivalent to
$$
\begin{tikzpicture}
\fill (0, 0) circle (0.1);
\draw (-0.5, 0) circle (0.5);
\draw (0.5, 0) circle (0.5);
\end{tikzpicture}.
$$
\end{example*}

\begin{example*}
$ \RR^n $ deformation retracts to a point, by $ f_t\br{x} = t \cdot x $.
\end{example*}

\begin{definition*}
\hfill
\begin{itemize}
\item $ X $ is \textbf{contractible} if it is homotopy equivalent to a point.
\item A continuous map is \textbf{nullhomotopic} if it is homotopy equivalent to a constant map.
\end{itemize}
\end{definition*}

\subsubsection{Cell complexes}

\begin{example*}
The \textbf{torus} $ \S^1 \times \S^1 $ is the union of a point, two open intervals, and the open disc $ \mathring{\D^2} $.
\end{example*}

These are called \textbf{cells}. Can think of discs $ \D^n $ glued together.

\lecture{2}{Tuesday}{15/01/19}

\begin{definition*}
A \textbf{CW-complex}, or \textbf{cell complex}, is a topological space $ X $ such that there exists a decomposition
$$ X = \bigcup_{n \in \NN} X^n, $$
where the $ X^n $ are constructed inductively in the following way.
\begin{itemize}
\item $ X^n $ is a discrete set.
\item For each $ n \ge 0 $ there is an collection of closed $ n $-discs $ \cbr{\D_\alpha^n} $ together with continuous maps $ \phi_\alpha : \partial \D_\alpha^n \to X^{n - 1} $, such that
$$ X^n = X^{n - 1} \sqcup \bigsqcup_\alpha \D_\alpha^n / \sim, $$
where $ x \sim \phi_\alpha\br{x} $ for all $ x \in \partial \D_\alpha^n $ for all $ \alpha $.
\item A subset $ U \subseteq X $ is open if and only if $ U \cap X^n $ is open for all $ n $.
\end{itemize}
\end{definition*}

\pagebreak

\begin{remark*}
\hfill
\begin{itemize}
\item As a set,
$$ X^n = X^{n - 1} \sqcup \bigsqcup_\alpha e_\alpha^n, $$
where each $ e_\alpha^n $ is homeomorphic to an open $ n $-disc. These $ e_\alpha^n $ are called the \textbf{$ n $-cells} of $ X $.
\item If $ X = X^m $ for some $ m $, then $ X $ is called \textbf{finite dimensional}. The minimal $ m $ such that $ X = X^m $ is the \textbf{dimension} of $ X $.
\end{itemize}
\end{remark*}

\begin{example*}
The following are CW-complexes.
$$ \sbr{0, 1}, \qquad \RR, \qquad \S^1, \qquad \text{a graph}, \qquad \S^n = \D^n / \partial \D^n. $$
Can also decompose CW-complexes.
\begin{itemize}
\item The sphere $ \S^2 $ is one $ 0 $-cell, one $ 1 $-cell, and two $ 2 $-cells.
\item The torus $ \S^1 \times \S^1 $ is one $ 0 $-cell, two $ 1 $-cells, and one $ 2 $-cell.
\item The M\"obius strip is two $ 0 $-cells, three $ 1 $-cells, and one $ 2 $-cell.
\item The Klein bottle is one $ 0 $-cell, two $ 1 $-cells, and one $ 2 $-cell.
\end{itemize}
\end{example*}

\begin{definition*}
If $ X $ is a CW-complex with finitely many cells the \textbf{Euler characteristic} $ \chi\br{X} $ of $ X $ is the number of even cells minus the number of odd cells.
\end{definition*}

\begin{fact*}
$ \chi\br{X} $ does not depend of the choice of cells decomposition.
\end{fact*}

\begin{example*}
\hfill
\begin{itemize}
\item $ \chi\br{\S^n} = 0 $ if $ n $ is odd and $ \chi\br{\S^n} = 2 $ if $ n $ is even.
\item $ \chi\br{\S^1 \times \S^1} = 0 $.
\end{itemize}
\end{example*}

This is the generalisation of the following observation by Leonhard Euler. Let $ P $ be a convex polyhedron, where $ V $ is the number of vertices of $ P $, $ E $ is the number of edges of $ P $, and $ F $ is the number of faces of $ P $. Then $ V - E + F = 2 $.

\begin{example*}
A topological space that is not a CW-complex. $ X = \cbr{0, 1} $ with trivial topology does not contain any closed points.
\end{example*}

\begin{fact*}
CW-complexes are always Hausdorff.
\end{fact*}

\pagebreak

\section{The fundamental group}

\subsection{Basic constructions}

\subsubsection{Paths and homotopy}

Let $ X $ be a topological space. A \textbf{path} is a continuous map $ f : \I \to X $, where $ \I = \sbr{0, 1} $.

\begin{definition*}
Two paths $ f_0 $ and $ f_1 $ are \textbf{homotopic} if there exists a homotopy between $ f_0 $ and $ f_1 $ preserving the endpoints, that is a continuous map
$$ \function[F]{\I \times \I}{X}{\br{s, t}}{f_t\br{s}}, $$
such that
$$ f_t\br{0} = f_0\br{0}, \qquad f_t\br{1} = f_0\br{1}, \qquad t \in \I, $$
$$ F\br{s, 0} = f_0\br{s}, \qquad F\br{s, 1} = f_1\br{s}, \qquad s \in \I. $$
\end{definition*}

\begin{example*}
Let $ X \subseteq \RR^n $ be a convex set. Then all the paths in $ X $ are homotopic if they have the same endpoints. Let $ f_0, f_1 : \I \to X $ be two paths such that $ f_0\br{0} = f_1\br{0} $ and $ f_0\br{1} = f_1\br{1} $. Define
$$ f_t\br{s} = \br{1 - t}f_0\br{s} + tf_1\br{s}. $$
\end{example*}

\begin{lemma}
Being homotopic is an equivalence relation on the set of paths with fixed endpoints. We will write $ f_0 \cong f_1 $ for two homotopic paths $ f_0 $ and $ f_1 $.
\end{lemma}

\begin{proof}
\hfill
\begin{itemize}
\item $ f $ is homotopic to $ f $.
\item If $ f_0 $ is homotopic to $ f_1 $ by a homotopy $ f_t $, then $ f_1 $ is homotopic to $ f_0 $ by the homotopy $ f_{1 - t} $.
\item If $ f_0 $ is homotopic to $ f_1 $ by a homotopy $ f_t $ and $ f_1 = g_0 $ is homotopic to $ g_1 $ by a homotopy $ g_t $, then $ f_0 $ is homotopic to $ g_1 $ by the homotopy
$$ h_t =
\begin{cases}
f_{2t} & 0 \le t \le \tfrac{1}{2} \\
g_{2t - 1} & \tfrac{1}{2} \le t \le 1
\end{cases}.
$$
Then
$$ \function[H]{\I \times \I}{X}{\br{s, t}}{h_t\br{s}} $$
is continuous because its restriction to the closed subsets $ \I \times \sbr{0, \tfrac{1}{2}} $ and $ \I \times \sbr{\tfrac{1}{2}, 1} $ is continuous, since if the restriction to two closed subsets is continuous then the restriction to the union of these subsets is continuous.
\end{itemize}
\end{proof}

\lecture{3}{Wednesday}{16/01/19}

Let $ X $ be a topological space and $ \I = \sbr{0, 1} $. If $ f : \I \to X $ is a path, $ \sbr{f} $ is the class of all paths on $ X $ homotopic to $ f $.

\begin{definition*}
Let $ f, g : \I \to X $ be two paths such that $ f\br{1} = g\br{0} $. The \textbf{product path} $ f \cdot g $ is the path
$$ \br{f \cdot g}\br{s} =
\begin{cases}
f\br{2s} & 0 \le s \le \tfrac{1}{2} \\
g\br{2s - 1} & \tfrac{1}{2} \le s \le 1
\end{cases}.
$$
\end{definition*}

A convention is that whenever we write $ f \cdot g $ we implicitly assume $ f\br{1} = g\br{0} $.

\begin{lemma}
\label{lem:1.2}
Let $ f_0, f_1, g_0, g_1 $ be paths on $ X $ such that $ f_1 \cong f_0 $ and $ g_0 \cong g_1 $. Then $ f_0 \cdot g_0 \cong f_1 \cdot g_1 $.
\end{lemma}

\begin{proof}
$$ \function{\I \times \I}{X}{\br{s, t}}{\br{f_t \cdot g_t}\br{s}} $$
is a homotopy between $ f_0 \cdot g_0 $ and $ f_1 \cdot g_1 $.
\end{proof}

\pagebreak

\begin{remark*}
Let $ \phi : \sbr{0, 1} \to \sbr{0, 1} $ be continuous such that $ \phi\br{0} = 0 $ and $ \phi\br{1} = 1 $. If $ f : \I \to X $ is a path, then $ f\phi \cong f $. This is a \textbf{reparametrisation}. Define
$$ \phi_t\br{s} = \br{1 - t}\phi\br{s} + ts, $$
then $ f\phi_t $ is a homotopy between $ f\phi $ and $ f $.
\end{remark*}

For $ x \in X $, let the \textbf{constant path} at $ x $ be
$$ \function[\c_x]{\I}{X}{s}{x}. $$
For a path $ f : \I \to X $, define
$$ \function[f^{-1}]{\I}{X}{s}{f\br{1 - s}}. $$

\begin{lemma}
\label{lem:1.3}
Let $ f, g, h : \I \to X $ be paths. Then
\begin{enumerate}
\item $ \br{f \cdot g} \cdot h \cong f \cdot \br{g \cdot h} $,
\item $ f \cdot \c_{f\br{1}} \cong f $ and $ \c_{f\br{0}} \cdot f \cong f $, and
\item $ f \cdot f^{-1} \cong \c_{f\br{0}} $ and $ f^{-1} \cdot f \cong \c_{f\br{1}} $.
\end{enumerate}
\end{lemma}

\begin{proof}
\hfill
\begin{enumerate}
\item $ \br{\br{f \cdot g} \cdot h}\phi = f \cdot \br{g \cdot h} $, where
$$ \phi\br{s} =
\begin{cases}
\tfrac{s}{2} & s \in \sbr{0, \tfrac{1}{2}} \\
s - \tfrac{1}{4} & s \in \sbr{\tfrac{1}{2}, \tfrac{3}{4}} \\
2s - 1 & s \in \sbr{\tfrac{3}{4}, 1}
\end{cases},
$$
so $ \br{f \cdot g} \cdot h \cong f \cdot \br{g \cdot h} $ by reparametrisation.
\item Again reparametrisation, by
$$ \psi\br{s} =
\begin{cases}
2s & s \in \sbr{0, \tfrac{1}{2}} \\
1 & s \in \sbr{\tfrac{1}{2}, 1}
\end{cases},
\qquad \chi\br{s} =
\begin{cases}
0 & s \in \sbr{0, \tfrac{1}{2}} \\
2s - 1 & s \in \sbr{\tfrac{1}{2}, 1}
\end{cases}.
$$
\item Define
$$ H\br{s, t} =
\begin{cases}
f\br{\max\cbr{1 - 2s, t}} & s \in \sbr{0, \tfrac{1}{2}} \\
f\br{\max\cbr{2s - 1, t}} & s \in \sbr{\tfrac{1}{2}, 1}
\end{cases}.
$$
$ H $ is continuous, and
$$ H\br{s, 0} = f^{-1} \cdot f, \qquad H\br{s, 1} = \c_{f\br{1}}. $$
The inverse is similar.
\end{enumerate}
\end{proof}

\begin{definition*}
A \textbf{loop} with \textbf{basepoint} $ x_0 \in X $ is a path $ f : \I \to X $ such that $ f\br{0} = f\br{1} = x_0 $.
\end{definition*}

\begin{definition*}
Denote by $ \pi_1\br{X, x_0} $ the set of \textbf{homotopy classes} $ \sbr{f} $ of loops $ f : \I \to X $ with basepoint $ x_0 $.
\end{definition*}

\begin{proposition}
$ \pi_1\br{X, x_0} $ is a group with product $ \sbr{f}\sbr{g} = \sbr{f \cdot g} $ and neutral element $ \c_{x_0} : \I \to X $, the constant path at $ x_0 $.
\end{proposition}

\begin{proof}
Follows directly from Lemma \ref{lem:1.2} and Lemma \ref{lem:1.3}.
\end{proof}

\begin{definition*}
$ \pi_1\br{X, x_0} $ is the \textbf{fundamental group} of $ X $ at $ x_0 $.
\end{definition*}

\begin{example*}
Let $ X \subseteq \RR^n $ be a convex set and $ x_0 \in X $. Then $ \pi_1\br{X, x_0} = 0 $, since $ X $ is convex, so all loops are homotopic to each other.
\end{example*}

\pagebreak

\begin{example*}
\hfill
\begin{itemize}
\item The fundamental group of a space $ X $ with the trivial topology is trivial, since $ X $ is simply-connected, because all maps $ f : \I \to X $ are continuous, so path-connected and all paths are homotopic.
\item The fundamental group of a space $ X $ with the discrete topology is trivial, since $ f : \I \to X $ is continuous implies that $ f $ is constant.
\end{itemize}
\end{example*}

Assume $ x_0, x_1 \in X $ such that $ x_0 $ and $ x_1 $ are in the same path-component of $ X $. Let $ h : \I \to X $ be a path such that $ h\br{0} = x_0 $ and $ h\br{1} = x_1 $. Define
$$ \function[\beta_h]{\pi_1\br{X, x_1}}{\pi_1\br{X, x_0}}{\sbr{f}}{\sbr{h \cdot f \cdot h^{-1}}}. $$
This is well-defined by Lemma \ref{lem:1.2}.

\begin{proposition}
$ \beta_h : \pi_1\br{X, x_1} \to \pi_1\br{X, x_0} $ is an isomorphism.
\end{proposition}

\begin{proof}
It is a homomorphism, since
$$ \beta_h\sbr{f \cdot g} = \sbr{h \cdot f \cdot g \cdot h^{-1}} = \sbr{h \cdot f \cdot h^{-1}}\sbr{h \cdot g \cdot h^{-1}} = \beta_h\sbr{f} \cdot \beta_h\sbr{g}, $$
and $ \beta_h\sbr{\c_{x_1}} = \sbr{\c_{x_1}} $. It is bijective with $ \br{\beta_h}^{-1} = \beta_{h^{-1}} $.
\end{proof}

If $ X $ is path-connected, we often write $ \pi_1\br{X} $ instead of $ \pi_1\br{X, x_0} $.

\begin{definition*}
$ X $ is \textbf{simply-connected} if it is path-connected and $ \pi_1\br{X} = 0 $.
\end{definition*}

\begin{proposition}
\label{prop:1.6}
$ X $ is simply-connected if and only if there exists a unique homotopy class of paths between any two points of $ X $.
\end{proposition}

\begin{proof}
\hfill
\begin{itemize}
\item[$ \implies $] There exists a path between any two points. Let $ f $ and $ g $ be two paths from $ x_0 $ to $ x_1 $ for $ x_0, x_1 \in X $. Then $ f \cdot g^{-1} \cong g \cdot g^{-1} $, so
$$ f \cong f \cdot g^{-1} \cdot g \cong g \cdot g^{-1} \cdot g \cong g. $$
\item[$ \impliedby $] $ X $ is path-connected. Then $ x_1 = x_0 $, so all loops at $ x_0 $ are homotopic to each other, so $ \pi_1\br{X} = 0 $.
\end{itemize}
\end{proof}

\subsubsection{The fundamental group of the circle}

The goal is to show that $ \pi_1\br{\S^1} \cong \ZZ $.

\lecture{4}{Friday}{18/01/19}

\begin{definition*}
A \textbf{covering space} of a space $ X $ is a space $ \widetilde{X} $ and a continuous map $ p : \widetilde{X} \to X $ such that for each $ x \in X $ there is an open $ x \in U \subseteq X $ such that
\begin{itemize}
\item $ p^{-1}\br{U} = \bigcup_{j \in J} \widetilde{U_j} $, where $ \widetilde{U_j} \subseteq \widetilde{X} $ is open,
\item $ \widetilde{U_i} \cap \widetilde{U_j} = \emptyset $ if $ i \ne j $, and
\item $ \eval{p}_{\widetilde{U_j}} : \widetilde{U_j} \to U $ is a homeomorphism for all $ j \in J $.
\end{itemize}
Such a $ U $ is called \textbf{evenly covered}. The $ \widetilde{U_j} $ are called \textbf{sheets}.
\end{definition*}

\begin{example*}
$$ \function[p]{\RR}{\S^1}{s}{\br{\cos 2\pi s, \sin 2\pi s}}. $$
\end{example*}

\pagebreak

\begin{definition*}
Let $ p : \widetilde{X} \to X $ be a covering space. A \textbf{lift} of a continuous map $ f : Y \to X $ is a continuous map $ \widetilde{f} : Y \to \widetilde{X} $ such that $ p\widetilde{f} = f $, so
$$
\begin{tikzcd}
& \widetilde{X} \arrow{d}{p} \\
Y \arrow{ur}{\widetilde{f}} \arrow[swap]{r}{f} & X
\end{tikzcd}.
$$
\end{definition*}

\begin{proposition}[Unique lifting property]
\label{prop:1.34}
Let $ p : \widetilde{X} \to X $ be a covering space and $ f : Y \to X $ be a continuous map. If there are two lifts $ \widetilde{f_1}, \widetilde{f_2} : Y \to \widetilde{X} $ of $ f $ such that $ \widetilde{f_1}\br{y} = \widetilde{f_2}\br{y} $ for some $ y \in Y $ and if $ Y $ is connected, then $ \widetilde{f_1} = \widetilde{f_2} $.
\end{proposition}

\begin{proof}
Let $ y \in Y $ and $ U \subseteq X $ be an evenly covered neighbourhood of $ f\br{y} $. Then
$$ p^{-1}\br{U} = \bigcup_j \widetilde{U_j}. $$
Let $ \widetilde{U_1} $ be the sheet such that $ \widetilde{f_1}\br{y} \in \widetilde{U_1} $, and let $ \widetilde{U_2} $ be the sheet such that $ \widetilde{f_2}\br{y} \in \widetilde{U_2} $. Let $ N \subseteq Y $ be open and $ y \in N $ such that $ \widetilde{f_1}\br{N} \subseteq \widetilde{U_1} $ and $ \widetilde{f_2}\br{N} \subseteq \widetilde{U_2} $. We have $ p\widetilde{f_1} = p\widetilde{f_2} $. Then $ \widetilde{f_1}\br{y} = \widetilde{f_2}\br{y} $ if and only if $ \widetilde{U_1} = \widetilde{U_2} $, if and only if $ \eval{\widetilde{f_1}}_N = \eval{\widetilde{f_2}}_N $. Let
$$ A = \cbr{y \in Y \st \widetilde{f_1}\br{y} = \widetilde{f_2}\br{y}}, $$
so $ A $ is open and $ Y \setminus A $ is open. Thus $ A \ne \emptyset $ implies that $ A = Y $.
\end{proof}

\begin{proposition}[Homotopy lifting property]
\label{prop:1.30}
Let $ p : \widetilde{X} \to X $ be a covering space and $ F : Y \times \I \to X $ be a continuous map such that there exists a lift $ \widetilde{f_0} : Y \times \cbr{0} \to \widetilde{X} $ of $ \eval{F}_{Y \times \cbr{0}} $. Then there is a unique lift $ \widetilde{F} : Y \times \I \to \widetilde{X} $ of $ F $ such that $ \eval{\widetilde{F}}_{Y \times \cbr{0}} = \widetilde{f_0} $.
\end{proposition}

\begin{proof}
Let $ y_0 \in Y $ and $ t \in \I $. There are open $ y_0 \in N_t \subseteq Y $ and $ t \in \br{a_t, b_t} \subseteq \I $ such that $ F\br{N_t \times \br{a_t, b_t}} \subseteq U \subseteq X $, where $ U \subseteq X $ is open and evenly covered. Compactness of $ \I $ implies that there exist
$$ 0 = t_0 < \dots < t_m = 1, $$
and there exists $ y_0 \in N \subseteq Y $ open such that $ F\br{N \times \sbr{t_i, t_{i + 1}}} \subseteq U_i \subseteq X $, where $ U_i \subseteq X $ is open and evenly covered. We inductively construct a lift $ \eval{\widetilde{F}}_{N \times \I} $ of $ \eval{F}_{N \times \I} $.
\begin{itemize}
\item $ \eval{\widetilde{F}}_{N \times \sbr{0, 0}} = \eval{\widetilde{f_0}}_{N \times \sbr{0, 0}} $ exists.
\item Assume the lift has been constructed on $ N \times \sbr{0, t_i} $. Let $ \widetilde{U_i} \subseteq \widetilde{X} $ be such that $ \eval{p}_{\widetilde{U_i}} : \widetilde{U_i} \to U_i $ such that $ \widetilde{F}\br{y_0, t_i} \subseteq \widetilde{U_i} $. After shrinking $ N $, may assume $ \widetilde{F}\br{N \times \cbr{t_i}} \subseteq \widetilde{U_i} $. Define $ \widetilde{F} $ on $ N \times \sbr{t_i, t_{i + 1}} $ to be composition of $ F $ with the homeomorphism $ p^{-1} : U_i \to \widetilde{U_i} $.
\end{itemize}
After finitely many steps we obtain a lift $ \widetilde{F} : N \times \I \to \widetilde{X} $, where $ y_0 \in N \subseteq Y $ is open, so for each $ y \in Y $ there is a neighbourhood $ N_y \subseteq Y $ such that $ \eval{F}_{N_y \times \I} : N_y \times \I \to X $ lifts. For all $ y \in Y $, $ \cbr{y} \times \I $ is connected and can be lifted, so Proposition \ref{prop:1.34} implies that the lift of $ N \times \I $ is unique. Thus there is a unique lift $ \widetilde{F} : Y \times \I \to \widetilde{X} $.
\end{proof}

\begin{example*}
Let $ X $ be a topological space and $ A $ be discrete. Then $ p : X \times A \to X $ is a covering space. This is the \textbf{trivial covering}. Show the unique lifting property and the homotopy lifting property for the trivial covering. \footnote{Exercise}
\end{example*}

\begin{corollary}
Let $ f : \I \to X $ be a path, $ f\br{0} = x_0 $, and $ p : \widetilde{X} \to X $ be a covering space. For each $ \widetilde{x_0} \in p^{-1}\br{x_0} $, there is a unique lift $ \widetilde{f} : \I \to \widetilde{X} $ such that $ \widetilde{f}\br{0} = \widetilde{x_0} $.
\end{corollary}

\begin{proof}
Proposition \ref{prop:1.30} for $ Y $ a point.
\end{proof}

\pagebreak

\begin{theorem}
\label{thm:1.7}
Let $ x_0 = \br{1, 0} \in \S^1 $. Then $ \pi_1\br{\S^1, x_0} $ is the infinite cyclic group generated by the homotopy class of the loop
$$ \function[\omega]{\I}{\S^1}{s}{\br{\cos 2\pi s, \sin 2\pi s}}. $$
\end{theorem}

\begin{remark*}
\hfill
\begin{itemize}
\item $ \sbr{\omega}^n = \sbr{\omega_n} $, where
$$ \omega_n\br{s} = \br{\cos 2\pi ns, \sin 2\pi ns}. $$
\item
$$ \function[p]{\RR}{\S^1}{s}{\br{\cos 2\pi s, \sin 2\pi s}} $$
is a covering space.
\item $ \omega_n $ lifts to
$$ \function[\widetilde{\omega_n}]{\I}{\RR}{s}{ns}, $$
such that $ \widetilde{\omega_n}\br{0} = 0 $ and $ \widetilde{\omega_n}\br{1} = n $.
\end{itemize}
\end{remark*}

\begin{proof}[Proof of Theorem \ref{thm:1.7}]
\hfill
\begin{itemize}
\item If $ f : \I \to \S^1 $ is a loop at $ x_0 $, then the homotopy lifting property implies that there exists a lift $ \widetilde{f} : \I \to \RR $ such that $ \widetilde{f}\br{0} = 0 $. Since $ p\br{\widetilde{f}\br{1}} = f\br{1} = x_0 $, then $ \widetilde{f}\br{1} = n $ for some $ n \in \ZZ $. Then $ \widetilde{\omega_n} : \I \to \RR $ is another path such that $ \widetilde{\omega_n}\br{0} = 0 $ and $ \widetilde{\omega_n}\br{1} = n $, so $ \widetilde{f} \cong \widetilde{\omega_n} $. Let $ F : \I \times \I \to \RR $ be a homotopy equivalence between $ \widetilde{f} $ and $ \widetilde{\omega_n} $. Then $ pF : \I \times \I \to \S^1 $ gives a homotopy between $ p\widetilde{f} = f $ and $ p\widetilde{\omega_n} = \omega_n $.
\item Let $ m, n \in \ZZ $ and assume $ \omega_m \cong \omega_n $. Let $ F : \I \times \I \to \S^1 $ be a homotopy. Then
$$ F\br{0, t} = \omega_m\br{t}, \qquad F\br{1, t} = \omega_n\br{t}, \qquad F\br{s, 0} = F\br{s, 1} = x_0, \qquad s, t \in \I. $$
The unique lifting property implies that $ \widetilde{\omega_n}, \widetilde{\omega_m} : \I \to \RR $ are unique lifts such that $ \widetilde{\omega_n}\br{0} = 0 = \widetilde{\omega_m}\br{0} $. The homotopy lifting property implies that $ F $ lifts uniquely to a homotopy $ \widetilde{F} : \I \times \I \to \RR $ between $ \widetilde{\omega_n} $ and $ \widetilde{\omega_m} $, and $ \widetilde{F}\br{s, 1} \in \ZZ $ for all $ s \in \I $. Thus $ \widetilde{F}\br{s, 1} = n = m $, so $ \omega_m \cong \omega_n $ if and only if $ n = m $.
\end{itemize}
\end{proof}

\lecture{5}{Tuesday}{22/01/19}

Lecture 5 is a problems class.

\lecture{6}{Wednesday}{23/01/19}

\begin{theorem}
Every non-constant polynomial $ p \in \CC\sbr{z} $ has a root in $ \CC $.
\end{theorem}

\begin{proof}
May assume $ p\br{z} = z^n + a_1z^{n - 1} + \dots + a_n $. Assume $ p $ has no roots in $ \CC $. For each $ r \in \RR_{\ge 0} $ we obtain a loop
$$ \function[f_r]{\I}{\CC}{s}{\dfrac{p\br{re^{2\pi is}} / p\br{r}}{\abs{p\br{re^{2\pi is}} / p\br{r}}}}, $$
so $ \abs{f_r\br{s}} = 1 $. Then $ f_r\br{0} = 1 $ and $ f_r\br{1} = 1 $, so $ f_r $ is a loop based at $ 1 $. Then $ f_0 $ is the constant loop at $ 1 $, and $ f_r\br{s} $ depends continuously on $ r $, so $ f_r \cong f_0 $ for all $ r \in \RR_{\ge 0} $ and $ \sbr{f_r} = \sbr{f_0} = 0 \in \pi_1\br{\S^1} $. Fix $ r \in \RR_{\ge 0} $ such that $ r > 1 $ and $ r > \abs{a_1} + \dots + \abs{a_n} $. For $ \abs{z} = r $ we have
$$ \abs{z^n} > \br{\abs{a_1} + \dots + \abs{a_n}}\abs{z^{n - 1}} \ge \abs{a_1z^{n - 1}} + \dots + \abs{a_n} \ge \abs{a_1z^{n - 1} + \dots + a_n}. $$
Hence, for $ 0 \le t \le 1 $ the polynomial
$$ p_t\br{z} = z^n + t\br{a_1z^{n - 1} + \dots + a_n} $$
has no root $ z $ with $ \abs{z} = r $. Define
$$ F_r\br{t, s} = \dfrac{p_t\br{re^{2\pi is}} / p_t\br{r}}{\abs{p_t\br{re^{2\pi is}} / p_t\br{r}}}. $$
Then $ F_r\br{0, s} = \omega_n\br{s} $ and $ F_r\br{1, s} = f_r\br{s} $, so $ \sbr{\omega_n} = \sbr{f_r} = 0 \in \pi_1\br{\S^1} $. Theorem \ref{thm:1.7} implies that $ n = 0 $, so $ p $ is constant.
\end{proof}

See Hatcher Theorem 1.9 and Theorem 1.10 for more applications.

\pagebreak

\begin{proposition}
Let $ X $ and $ Y $ be path-connected topological spaces, $ x_0 \in X $, and $ y_0 \in Y $. Then
$$ \pi_1\br{X \times Y, \br{x_0, y_0}} \cong \pi_1\br{X, x_0} \times \pi_1\br{Y, y_0}. $$
\end{proposition}

\begin{proof}
A map
$$ \function[f]{Z}{X \times Y}{z}{\br{g\br{z}, h\br{z}}} $$
is continuous if and only if $ g : Z \to X $ and $ h : Z \to Y $ are continuous. For $ Z = \I $,
$$ \correspondence{\text{loops in} \ X \times Y \ \text{based} \ \br{x_0, y_0}}{\text{loops in} \ X \ \text{based} \ x_0 \ \} \times \{ \ \text{loops in} \ Y \ \text{based} \ y_0}. $$
Two loops
$$ \function[f_1]{\I}{X \times Y}{s}{\br{g_1\br{s}, h_1\br{s}}}, \qquad \function[f_2]{\I}{X \times Y}{s}{\br{g_2\br{s}, h_2\br{s}}} $$
are homotopic if and only if $ g_1 \cong g_2 $ and $ h_1 \cong h_2 $, so there is a bijection
$$ \pi_1\br{X \times Y, \br{x_0, y_0}} \cong \pi_1\br{X, x_0} \times \pi_1\br{Y, y_0}. $$
Then $ f_1 \cdot f_2 = \br{g_1 \cdot g_2, h_1 \cdot h_2} $ and the constant loop is mapped to the constant loop, so this is also a group isomorphism.
\end{proof}

\begin{example*}
The torus $ \S^1 \times \S^1 $ has
$$ \pi_1\br{\S^1 \times \S^1} \cong \pi_1\br{\S^1} \times \pi_1\br{\S^1} \cong \ZZ^2. $$
\end{example*}

\subsubsection{Induced homomorphisms}

Let $ X $ and $ Y $ be topological spaces, $ x_0 \in X $, and $ \phi : X \to Y $. An observation is that $ \phi $ induces a homomorphism
$$ \function[\phi_*]{\pi_1\br{X, x_0}}{\pi_1\br{Y, \phi\br{x_0}}}{\sbr{f}}{\sbr{\phi f}}. $$
$ \phi_* $ is well-defined, since if $ f_t $ is a homotopy between the loops $ f_0 $ and $ f_1 $ based at $ x_0 $, then $ \phi f_t $ is a homotopy of loops between $ \phi f_0 $ and $ \phi f_1 $. Moreover, $ \phi\br{f \cdot g} = \br{\phi f} \cdot \br{\phi g} $ and $ \phi $ maps the constant path at $ x_0 $ to the constant path at $ \phi\br{x_0} $, so $ \phi $ is a homomorphism.

\begin{proposition}
\hfill
\begin{enumerate}
\item Let $ \psi : X \to Y $ and $ \phi : Y \to Z $ be continuous maps between topological spaces, $ x_0 \in X $, and
$$ \psi_* : \pi_1\br{X, x_0} \to \pi_1\br{Y, \psi\br{x_0}}, \qquad \phi_* : \pi_1\br{Y, \psi\br{x_0}} \to \pi_1\br{Z, \phi\psi\br{x_0}}, $$
$$ \br{\phi\psi}_* : \pi_1\br{X, x_0} \to \pi_1\br{Z, \phi\psi\br{x_0}}. $$
Then $ \br{\phi\psi}_* = \phi_*\psi_* $.
\item Let $ \id_X : X \to X $ be the identity then
$$ \br{\id_X}_* : \pi_1\br{X, x_0} \to \pi_1\br{X, x_0} $$
is the identity.
\end{enumerate}
\end{proposition}

\begin{proof}
\hfill
\begin{enumerate}
\item Let $ f : \I \to X $ be a loop at $ x_0 $, then
$$ \br{\phi\psi}_*\br{\sbr{f}} = \sbr{\br{\phi\psi}f} = \sbr{\phi\br{\psi f}} = \phi_*\br{\sbr{\psi f}} = \phi_*\psi_*\br{\sbr{f}}. $$
\item $ \br{\id_X}_*\br{\sbr{f}} = \sbr{\id_Xf} = \sbr{f} $.
\end{enumerate}
\end{proof}

These two observations yield in particular that if $ \phi : X \to Y $ is a homeomorphism with inverse $ \psi : Y \to X $, then
$$ \phi_* : \pi_1\br{X, x_0} \to \pi_1\br{Y, \phi\br{x_0}} $$
is an isomorphism with inverse $ \psi_* $.

\pagebreak

\lecture{7}{Friday}{25/01/19}

\begin{proposition}
\label{prop:1.18}
Let $ \phi : X \to Y $ be a homotopy equivalence. Then
$$ \phi_* : \pi_1\br{X, x_0} \to \pi_1\br{Y, \phi\br{x_0}} $$
is an isomorphism for all $ x_0 \in X $.
\end{proposition}

Recall that if $ x_0, x_1 \in X $ and $ h : \I \to X $ is a path such that $ h\br{0} = x_0 $ and $ h\br{1} = x_1 $, then we obtain an isomorphism
$$ \function[\beta_h]{\pi_1\br{X, x_1}}{\pi_1\br{X, x_0}}{\sbr{f}}{\sbr{h \cdot f \cdot h^{-1}}}. $$

\begin{lemma}
\label{lem:1.19}
Let $ \phi_t : X \to Y $ be a homotopy and $ x_0 \in X $. Define the path
$$ \function[h]{\I}{Y}{s}{\phi_s\br{x_0}}, \qquad h\br{0} = \phi_0\br{x_0}, \qquad h\br{1} = \phi_1\br{x_0}. $$
Then $ \phi_{0*} = \beta_h\phi_{1*} $, that is the following diagram commutes.
$$
\begin{tikzcd}
& \pi_1\br{Y, \phi_1\br{x_0}} \arrow{dd}{\beta_h}[swap]{\sim} \\
\pi_1\br{X, x_0} \arrow{ur}{\phi_{1*}} \arrow[swap]{dr}{\phi_{0*}} & \\
& \pi_1\br{Y, \phi_0\br{x_0}}
\end{tikzcd}.
$$
\end{lemma}

\begin{proof}
For $ t \in \I $, define the path
$$ \function[h_t]{\I}{X}{s}{h\br{ts}}, \qquad h_t\br{0} = \phi_0\br{x_0}, \qquad h_t\br{1} = h\br{t} = \phi_t\br{x_0}. $$
Let $ f $ be a loop at $ x_0 $. Define
$$ F_t = h_t \cdot \br{\phi_tf} \cdot h_t^{-1}. $$
Then $ F_t $ is a loop at $ \phi_0\br{x_0} $, which is continuous in $ t $. So $ F_t $ is a homotopy of loops between
$$ F_0 = h_0 \cdot \br{\phi_0f} \cdot h_0^{-1} \cong \phi_0f, \qquad F_1 = h_1 \cdot \br{\phi_1f} \cdot h_1^{-1} = h \cdot \br{\phi_1f} \cdot h^{-1}. $$
Hence
$$ \phi_{0*}\br{\sbr{f}} = \sbr{\phi_0f} = \sbr{h \cdot \br{\phi_1f} \cdot h^{-1}} = \beta_h\br{\sbr{\phi_1f}} = \beta_h\phi_{1*}\br{\sbr{f}}. $$
\end{proof}

Lemma \ref{lem:1.19} implies in particular the following.

\begin{corollary}
If $ \psi : X \to X $ is continuous and $ \psi \cong \id_X $, then
$$ \psi_* : \pi_1\br{X, x_0} \to \pi_1\br{X, \psi\br{x_0}} $$
is an isomorphism for all $ x_0 \in X $.
\end{corollary}

\begin{proof}
By Lemma \ref{lem:1.19} there is a path $ h $ from $ \psi\br{x_0} $ to $ x_0 $ such that
$$
\begin{tikzcd}
& \pi_1\br{X, x_0} \arrow{dd}{\beta_h}[swap]{\sim} \\
\pi_1\br{X, x_0} \arrow{ur}{\br{\id_X}_*} \arrow[swap]{dr}{\psi_*} & \\
& \pi_1\br{X, \psi\br{x_0}}
\end{tikzcd},
$$
so $ \psi_* = \beta_h $ hence an isomorphism.
\end{proof}

\pagebreak

\begin{proof}[Proof of Proposition \ref{prop:1.18}]
Let $ \phi : X \to Y $ be a homotopy equivalence. Let $ \psi : Y \to X $ be a homotopy inverse of $ \phi $, that is $ \phi\psi \cong \id_Y $ and $ \psi\phi \cong \id_X $. Then
$$ \pi_1\br{X, x_0} \xrightarrow{\phi_*} \pi_1\br{Y, \phi\br{x_0}} \xrightarrow{\psi_*} \pi_1\br{X, \psi\phi\br{x_0}} \xrightarrow{\phi_*} \pi_1\br{Y, \psi\phi\psi\br{x_0}}. $$
Have to show that $ \phi_* $ is bijective. The observation above implies that $ \br{\psi\phi}_* = \psi_*\phi_* $ is an isomorphism, so $ \phi_* $ is injective and $ \psi_* $ is surjective. Similarly $ \br{\phi\psi}_* = \phi_*\psi_* $ is an isomorphism, so $ \psi_* $ is injective and $ \phi_* $ is surjective.
\end{proof}

\begin{lemma}
\label{lem:1.15}
Let $ X $ be a topological space and $ x_0 \in X $. Assume
$$ X = \bigcup_{\alpha \in \Lambda} A_\alpha, $$
such that
\begin{itemize}
\item the $ A_\alpha $ are all open and path-connected,
\item $ x_0 \in A_\alpha $ for all $ \alpha \in \Lambda $, and
\item all the intersections $ A_\alpha \cap A_\beta $ are path-connected for all $ \alpha, \beta \in \Lambda $.
\end{itemize}
If $ f $ is a loop in $ X $ at $ x_0 $, then we can write
$$ \sbr{f} = \sbr{h_1} \dots \sbr{h_m}, $$
such that the $ h_i $ are loops at $ x_0 $, and each contained in a single $ A_{\alpha_i} $.
\end{lemma}

\begin{proof}
$ f $ is continuous, so for all $ s \in \I $ there is an open neighbourhood $ V_s $ such that $ f\br{V_s} $ such that $ f\br{V_s} \subseteq A_\alpha $ for some $ \alpha $. We can choose $ V_s $ to be an interval $ \br{a_s, b_s} $ such that $ f\br{\sbr{a_s, b_s}} \subseteq A_\alpha $. Then $ \I $ is compact, so a finite number of such intervals cover $ \I $, so there is a partition
$$ 0 = s_0 < \dots < s_m = 1, $$
such that $ f\br{\sbr{s_{i - 1}, s_i}} \subseteq A_{\alpha_i} $ for some $ \alpha_i $. Let $ f_i $ be the path obtained by restricting $ f $ to $ \sbr{s_{i - 1}, s_i} $, and rescaling. Then $ f \cong f_1 \cdot \dots \cdot f_m $ for $ f_i \subseteq A_{\alpha_i} $ and $ A_{\alpha_i} \cap A_{\alpha_j} $ is path-connected. Let $ g_i $ be a path from $ x_0 $ to $ f\br{s_i} $ in $ A_{\alpha_i} \cap A_{\alpha_{i + 1}} $. Let $ g_0 $ and $ g_m $ be the constant loops at $ x_0 $. Then $ h_i = g_{i - 1} \cdot f_i \cdot g_i^{-1} $ is a loop based at $ x_0 $ and $ h_i \subseteq A_{\alpha_i} $. Thus
$$ f \cong \br{g_0 \cdot f_1 \cdot g_1^{-1}} \cdot \dots \cdot \br{g_{m - 1} \cdot f_m \cdot g_m^{-1}}, $$
so $ \sbr{f} = \sbr{h_1} \dots \sbr{h_m} $.
\end{proof}

\lecture{8}{Tuesday}{29/01/19}

\begin{example*}
M\"obius strip $ M $ deformation retracts to $ \S^1 $. Thus $ \phi : M \to \S^1 $ is a homotopy equivalence, so $ \pi_1\br{M} \cong \pi_1\br{\S^1} \cong \ZZ $.
\end{example*}

\begin{example*}
There is no deformation retraction of $ \S^1 $ to a point $ p \in \S^1 $ because $ \pi_1\br{\S^1} \ncong \pi_1\br{p} $.
\end{example*}

\begin{example*}
There is no retraction of the disc $ \D^2 $ to its boundary $ \S^1 \subseteq \D^2 $. Assume there is a retraction $ r : \D^2 \to \S^1 $, consider the embedding $ i : \S^1 \hookrightarrow \D^2 $. Then $ ri = \id_{\S^1} $. Thus
$$
\begin{tikzcd}[row sep=tiny]
\pi_1\br{\S^1} \arrow{r}{i_*} \arrow[cong]{d} & \pi_1\br{\D^2} \arrow{r}{r_*} \arrow[cong]{d} & \pi_1\br{\S^1} \arrow[cong]{d} \\
\ZZ & 0 & \ZZ
\end{tikzcd},
$$
so $ r_*i_*\br{\pi_1\br{\S^1}} = 0 $ but $ r_*i_* = \br{ri}_* = \id_{\pi_1\br{\S^1}} $, a contradiction.
\end{example*}

\begin{theorem}[Brouwer fixed point theorem]
Let $ h : \D^2 \to \D^2 $ be a continuous map. Then $ h $ has a fixed point, that is there exists $ x \in \D^2 $ such that $ h\br{x} = x $.
\end{theorem}

\begin{proof}
Assume $ h\br{x} \ne x $ for all $ x \in \D^2 $. Define $ r : \D^2 \to \S^1 $ by defining $ r\br{x} $ to be the intersection of the ray starting at $ h\br{x} $ towards $ x $ with $ \S^1 $. Then $ r $ is continuous, and if $ x \in \S^1 $, then $ r\br{x} = x $, so $ r $ is a retraction, a contradiction.
\end{proof}

\pagebreak

Lemma \ref{lem:1.15} implies that if $ U_1, U_2 \subseteq X $ are open and path-connected such that $ U_1 \cup U_2 = X $ and $ U_1 \cap U_2 $ is path-connected and $ x_0 \in U_1 \cap U_2 $, then every $ \sbr{f} \in \pi_1\br{X, x_0} $ can be factorised as
$$ \sbr{f} = \sbr{g_1}\sbr{h_1} \dots \sbr{g_n}\sbr{h_n}, $$
such that the $ g_i $ are loops at $ x_0 $ contained in $ U_1 $ and the $ h_i $ are loops at $ x_0 $ contained in $ U_2 $. In other words, $ i_1 : U_1 \hookrightarrow X $ and $ i_2 : U_2 \hookrightarrow X $, so
$$ i_{1*} : \pi_1\br{U_1, x_0} \to \pi_1\br{X, x_0}, \qquad i_{2*} : \pi_1\br{U_2, x_0} \to \pi_1\br{X, x_0}. $$
Lemma \ref{lem:1.15} implies that $ i_{1*}\br{\pi_1\br{U_1, x_0}} \cup i_{2*}\br{\pi_1\br{U_2, x_0}} $ generate $ \pi_1\br{X, x_0} $.

\begin{proposition}
$ \pi_1\br{\S^n} = 0 $ if $ n \ge 2 $.
\end{proposition}

\begin{proof}
Let
$$ U_1 = \S^n \setminus \cbr{\br{1, 0, \dots, 0}}, \qquad U_2 = \S^n \setminus \cbr{\br{-1, 0, \dots, 0}}. $$
Then $ U_1 \cong \RR^n $ and $ U_2 \cong \RR^n $, by stereographic projection. Then $ U_1 \cup U_2 = \S^n $ and $ U_1 \cap U_2 $ is path-connected. Let $ x_0 \in U_1 \cap U_2 $. Then $ \pi_1\br{U_1, x_0} = 0 $ and $ \pi_1\br{U_2, x_0} = 0 $, so Lemma \ref{lem:1.15} implies that $ \pi_1\br{\S^n, x_0} $.
\end{proof}

\subsection{Seifert-van Kampen theorem}

\subsubsection{Free products with amalgamation}

\begin{definition*}
If $ S $ is a set, then $ \F_S $ is the \textbf{free group} on $ S $. We can write any group $ G $ as a quotient of some free group $ \F_S $, $ G = \F_S / \abr{\abr{R}} $, where $ \abr{\abr{R}} $ is the \textbf{normal closure} of $ R \subseteq \F_S $, the smallest normal subgroup of $ \F_S $ containing $ R $. We write $ G = \abr{S \st R} $. This is called a \textbf{presentation} of $ G $.
\end{definition*}

Let $ G_0, G_1, G_2 $ be groups, and $ f_1 : G_0 \to G_1 $ and $ f_2 : G_0 \to G_2 $ be homomorphisms.

\begin{definition*}
A group $ H $ together with homomorphisms $ h_1 : G_1 \to H $ and $ h_2 : G_2 \to H $ such that $ h_1f_1 = h_2f_2 $ is an \textbf{amalgamated product} of $ G_1 $ and $ G_2 $ over $ G_0 $ if it satisfies the following universal property. For every group $ G $ and all homomorphisms $ h_1' : G_1 \to G $ and $ h_2' : G_2 \to G $ such that $ h_1'f_1 = h_2'f_2 $, there exists a unique homomorphism $ \alpha : H \to G $ such that $ h_1' = \alpha h_1 $ and $ h_2' = \alpha h_2 $, so
$$
\begin{tikzcd}
G_0 \arrow{r}{f_1} \arrow[swap]{d}{f_2} & G_1 \arrow{d}{h_1} \arrow[bend left=30]{ddr}{h_1'} & \\
G_2 \arrow[swap]{r}{h_2} \arrow[bend right=30, swap]{drr}{h_2'} & H \arrow[dashed]{dr}{\exists !\alpha} & \\
& & G
\end{tikzcd}.
$$
\end{definition*}

\begin{theorem}
Given $ f_1 : G_0 \to G_1 $ and $ f_2 : G_0 \to G_2 $. Then there exists an amalgamated product, unique up to isomorphism. We denote it by
$ G_1 \underset{G_0}{*} G_2 $.
\end{theorem}

\begin{proof}
Worksheet $ 2 $.
\end{proof}

\lecture{9}{Wednesday}{30/01/19}

$ G_0 = \cbr{\id} $ is the \textbf{free product}. We write $ G_1 * G_2 $ instead of $ G_1 \underset{\cbr{\id}}{*} G_2 $. Let $ G_1 = \abr{S_1 \st R_1} $ and $ G_2 = \abr{S_2 \st R_2} $. Then $ G_1 * G_2 = \abr{S_1 \sqcup S_2 \st R_1 \cup R_2} $, with injections $ G_i \hookrightarrow G_1 * G_2 $ for $ i = 1, 2 $. More generally,
$$ G_1 \underset{G_0}{*} G_2 \cong G_1 * G_2 / N. $$
where $ N $ is the normal closure of the set
$$ \cbr{f_1\br{g}f_2\br{g}^{-1} \st g \in G_0} \subseteq G_1 * G_2. $$

\pagebreak

\subsubsection{The Seifert-van Kampen theorem}

\begin{theorem}[Seifert-van Kampen]
\label{thm:seifertvankampen}
Let $ X $ be a topological space and $ U_1, U_2 \subseteq X $ be open and path-connected such that $ X = U_1 \cup U_2 $ and $ U_1 \cap U_2 $ is path-connected and let $ x_0 \in U_1 \cap U_2 $. Then
$$ \pi_1\br{X, x_0} \cong \pi_1\br{U_1, x_0} \underset{\pi_1\br{U_1 \cap U_2, x_0}}{*} \pi_2\br{U_2, x_0} \cong \pi_1\br{U_1, x_0} * \pi_1\br{U_2, x_0} / N, $$
where $ N $ is the normal closure of the set
$$ \cbr{j_{1*}\br{\omega}j_{2*}\br{\omega}^{-1} \st \omega \in \pi_1\br{U_1 \cap U_2, x_0}}, $$
and $ j_i : U_1 \cap U_2 \hookrightarrow U_i $, so
$$
\begin{tikzcd}
U_1 \cap U_2 \arrow[hookrightarrow]{r}{i_1} \arrow[hookrightarrow, swap]{d}{i_2} & U_1 \arrow[hookrightarrow]{d}{j_1} \\
U_2 \arrow[hookrightarrow, swap]{r}{j_2} & X
\end{tikzcd}
\qquad \implies \qquad
\begin{tikzcd}
\pi_1\br{U_1 \cap U_2, x_0} \arrow{r}{i_{1*}} \arrow[swap]{d}{i_{2*}} & \pi_1\br{U_1, x_0} \arrow{d}{j_{1*}} \\
\pi_1\br{U_2, x_0} \arrow[swap]{r}{j_{2*}} & \pi_1\br{U_1, x_0} \underset{\pi_1\br{U_1 \cap U_2, x_0}}{*} \pi_2\br{U_2, x_0}
\end{tikzcd}.
$$
\end{theorem}

\begin{proof}[Proof of Theorem \ref{thm:seifertvankampen}]
Appendix A.1.
\end{proof}

\lecture{10}{Friday}{01/02/19}

\begin{theorem}[Seifert-van Kampen, strong version]
Let $ X $ be a path-connected topological space such that
\begin{itemize}
\item $ X = \bigcup_\alpha A_\alpha $,
\item $ A_\alpha, A_\alpha \cap A_\beta, A_\alpha \cap A_\beta \cap A_\gamma $ are open and path-connected for all $ \alpha, \beta, \gamma $, and
\item $ x_0 \in \bigcap_\alpha A_\alpha $.
\end{itemize}
Then
$$ \pi_1\br{X, x_0} \cong \underset{\alpha}{*} \pi_1\br{A_\alpha, x_0} / N, $$
where $ N \subseteq \underset{\alpha}{*} \pi_1\br{A_\alpha, x_0} $ is the normal closure of the set
$$ \cbr{\br{i_{\alpha\beta}}_*\br{\omega}\br{i_{\beta\alpha}}_*\br{\omega}^{-1} \st \omega \in \pi_1\br{A_\alpha \cap A_\beta}}, $$
and $ i_{\alpha\beta} : A_\alpha \cap A_\beta \hookrightarrow A_\alpha $ is the inclusion.
\end{theorem}

\begin{example*}
Let $ \S^1 \vee \S^1 $ be the wedge product. Fix $ x \in \S^1 $ and $ y \in \S^1 $. Then
$$
\begin{tikzpicture}
\draw (-3.5, 0) node{$ \S^1 \vee \S^1 = \S^1 \sqcup \S^1 / x \sim y = $};
\fill (0, 0) circle (0.1);
\draw (-0.5, 0) circle (0.5) node[above]{$ b $};
\draw (0.5, 0) circle (0.5) node[above]{$ a $};
\end{tikzpicture}.
$$
Let
$$
\begin{tikzpicture}
\draw (-1.5, 0) node{$ A = $};
\fill (0, 0) circle (0.1);
\draw (-0.5, 0) circle (0.5);
\draw (0.5, 0.5) arc (90:270:0.5);
\end{tikzpicture},
\qquad
\begin{tikzpicture}
\draw (-1, 0) node{$ B = $};
\fill (0, 0) circle (0.1);
\draw (-0.5, 0.5) arc (90:-90:0.5);
\draw (0.5, 0) circle (0.5);
\end{tikzpicture},
\qquad
\begin{tikzpicture}
\draw (-1.5, 0) node{$ A \cap B = $};
\fill (0, 0) circle (0.1);
\draw (-0.5, 0.5) arc (90:-90:0.5);
\draw (0.5, 0.5) arc (90:270:0.5);
\end{tikzpicture}.
$$
Then $ \pi_1\br{A} \cong \abr{b} \cong \ZZ $, $ \pi_1\br{B} \cong \abr{a} \cong \ZZ $, and $ \pi_1\br{A \cap B} = \cbr{\id} $, and $ A, B, A \cap B $ are open and path-connected. Van Kampen implies that
$$ \pi_1\br{\S^1 \vee \S^1} \cong \pi_1\br{A} * \pi_1\br{B} \cong \ZZ * \ZZ \cong \F_{\cbr{a, b}}. $$
More generally, let $ X = \S_{a_1}^1 \vee \dots \vee \S_{a_n}^1 $. Induction implies that
$$ \pi_1\br{X} = \ZZ * \dots * \ZZ \cong \F_{\cbr{a_1, \dots, a_n}}. $$
Similarly, let $ X = \bigvee_{\alpha \in \Lambda} \S_\alpha^1 $. Strong version of van Kampen implies that
$$ \pi_1\br{X} = \underset{\alpha \in \Lambda}{*} \ZZ = \F_\Lambda. $$
\end{example*}

\pagebreak

\begin{example*}
Let $ T $ be a torus and $ x_0 \in T $. Let
$$ A = T \setminus \cbr{\text{small closed disc} \ D}, \qquad B = \cbr{\text{open set that contains} \ D \ \text{and} \ x_0}. $$
\begin{itemize}
\item $ A $ is homotopy equivalent to $ \S^1 \vee \S^1 $, so $ \pi_1\br{A} \cong \F_{\cbr{a, b}} $.
\item $ B $ is homeomorphic to $ \D^2 $, so $ \pi_1\br{B} = \cbr{\id} $.
\item $ A \cap B $ is homotopy equivalent to $ \S^1 $, so $ \pi_1\br{A \cap B} \cong \ZZ $.
\end{itemize}
Then $ A, B, A \cap B $ are open and path-connected. Van Kampen implies that
$$ \pi_1\br{T} \cong \pi_1\br{A} / \abr{\abr{i_*\br{\pi_1\br{A \cap B}}}}, $$
where $ i : A \cap B \hookrightarrow A $. Then
$$ \function[i_*]{\pi_1\br{A \cap B} = \abr{\omega}}{\pi_1\br{A}}{\omega}{aba^{-1}b^{-1}}, $$
so
$$ \pi_1\br{T} \cong \F_{\cbr{a, b}} / \abr{\abr{aba^{-1}b^{-1}}} = \abr{a, b \st aba^{-1}b^{-1}} \cong \ZZ^2. $$
\end{example*}

\subsubsection{Applications to CW-complexes}

Let $ X $ be a path-connected topological space. Let $ Y $ be the space obtained by attaching $ 2 $-cells $ \cbr{e_\alpha^2} $ to $ X $ along maps $ \phi_\alpha : \partial \D^2 = \S^1 \to X $. Consider the loops
$$ \function[\phi_\alpha']{\I}{X}{s}{\phi_\alpha\br{\cos 2\pi s, \sin 2\pi s}}, $$
based at $ \phi_\alpha'\br{0} $. Let $ \gamma_\alpha $ be a path from $ x_0 $ to $ \phi_\alpha'\br{0} $ for each $ \alpha $. Then $ \gamma_\alpha \cdot \phi_\alpha \cdot \gamma_\alpha^{-1} $ is a loop at $ x_0 $. After attaching $ e_\alpha^2 $, the loop $ \gamma_\alpha \cdot \phi_\alpha \cdot \gamma_\alpha^{-1} $ is homotopic to the constant loop at $ x_0 $. Let $ N \subseteq \pi_1\br{X, x_0} $ be the normal closure of all the elements of the form $ \sbr{\gamma_\alpha \cdot \phi_\alpha \cdot \gamma_\alpha^{-1}} $. The inclusion $ i : X \hookrightarrow Y $ yields
$$ i_* : \pi_1\br{X, x_0} \to \pi_1\br{Y, x_0}, $$
and $ N \subseteq \Ker i_* $.

\begin{proposition}
\label{prop:1.26}
This inclusion $ i : X \hookrightarrow Y $ induces a surjection
$$ i_* : \pi_1\br{X, x_0} \to \pi_1\br{Y, x_0}, $$
and $ \Ker i_* = N $, so
$$ \pi_1\br{Y, x_0} \cong \pi_1\br{X, x_0} / N. $$
\end{proposition}

\begin{proof}
Construct a space $ Z $ from $ Y $ by attaching a strip $ \I \times \I $ to $ Y $ by identifying the lower edge $ \I \times \cbr{0} $ with the path $ \gamma_\alpha $ and the right edge $ \cbr{1} \times \I $ with an arch on $ e_\alpha^2 $. Attach all the left edges of the strips with each other. Then $ Z $ deformation retracts to $ Y $. Choose a point $ y_\alpha \in e_\alpha^2 $ for each $ \alpha $, such that $ y_\alpha $ is not contained in $ X $ or in the attached strip. Let
$$ A = Z \setminus \bigcup_\alpha \cbr{y_\alpha}, \qquad B = Z \setminus X. $$
\begin{itemize}
\item $ A $ deformation retracts to $ X $.
\item $ B $ is homotopy equivalent to a point.
\item $ A \cap B $ is homotopy equivalent to
$$
\begin{tikzpicture}
\draw (-5, 0) node{$ \cbr{\text{paths} \ \gamma_\alpha \ \text{from} \ x_0 \ \text{to loops} \ \phi_\alpha'} = $};
\fill (0, 0) circle (0.1) node[above]{$ x_0 $};
\draw (0, 0) to node[above]{$ \gamma_\alpha $} (-1, 0);
\draw (0, 0) to node[above]{$ \gamma_\alpha $} (1, 0);
\draw (-1.5, 0) circle (0.5) node[above]{$ \phi_\alpha' $};
\draw (1.5, 0) circle (0.5) node[above]{$ \phi_\alpha' $};
\end{tikzpicture}.
$$
\end{itemize}
Then $ A, B, A \cap B $ are open and path-connected. Van Kampen implies that
$$ \pi_1\br{Y} \cong \pi_1\br{Z} = \pi_1\br{A} / \abr{\abr{j_*\br{\pi_1\br{A \cap B}}}}, $$
where $ j : A \cap B \hookrightarrow A $ is the inclusion. So $ \abr{\abr{j_*\br{\pi_1\br{A \cap B}}}} $ is exactly $ N $. Thus $ \pi_1\br{A} = \pi_1\br{X} $.
\end{proof}

\pagebreak

\lecture{11}{Tuesday}{05/02/19}

\begin{corollary}
For every group $ G $ there exists a two-dimensional CW-complex $ X_G $ such that $ \pi_1\br{X_G} = G $.
\end{corollary}

\begin{proof}
Let $ G = \abr{\cbr{g_\alpha} \st \cbr{r_\beta}} $ be a presentation of $ G $, that is $ G = \F_{\cbr{g_\alpha}} / \abr{\abr{\cbr{r_\beta}}} $. Seen last time that $ \pi_1\br{\bigvee_{g_\alpha} \S_{g_\alpha}^1} = \F_{\cbr{g_\alpha}} $. Each word $ r_\beta $ defines a loop in $ \bigvee_{g_\alpha} \S_{g_\alpha}^1 $. Attach $ 2 $-cells to $ \bigvee_{g_\alpha} \S_{g_\alpha}^1 $ along the loops defined by the relations $ \cbr{r_\beta} $. Call this new CW-complex $ Y $. Proposition \ref{prop:1.26} implies that
$$ \pi_1\br{Y, x_0} \cong \pi_1\br{X, x_0} / \abr{\abr{\cbr{r_\beta}}} \cong \F_{\cbr{g_\alpha}} / \abr{\abr{\cbr{r_\beta}}} \cong G. $$
\end{proof}

\begin{remark*}
Let $ X = \bigcup_n X^n $ be a CW-complex, path-connected. Proposition \ref{prop:1.26} can be used to show the following two facts.
\begin{itemize}
\item The inclusion $ X^1 \hookrightarrow X $ induces a surjective homomorphism $ \pi_1\br{X^1} \to \pi_1\br{X} $.
\item The inclusion $ X^2 \hookrightarrow X $ induces an isomorphism $ \pi_1\br{X^2} \to \pi_1\br{X} $.
\end{itemize}
\end{remark*}

\subsection{Covering spaces}

\subsubsection{Lifting properties}

Let $ X $ be a topological space. Recall that a covering space is $ p : \widetilde{X} \to X $ such that each $ x \in X $ has an open neighbourhood $ U $ such that
$$ p^{-1}\br{U} = \bigcup_\alpha \widetilde{U_\alpha}, $$
where $ U_\alpha $ are pairwise disjoint and $ \eval{p}_{\widetilde{U_\alpha}} : \widetilde{U_\alpha} \to U $ is a homeomorphism for all $ \alpha $.

\begin{example*}
$$
\begin{tikzpicture}
\draw (-5, 0) node{$ \function{\RR}{\S^1}{s}{\br{\cos 2\pi s, \sin 2\pi s}}, \quad \function{\S^1}{\S^1}{z}{z^n}, $};
\draw (0, 0) circle (0.5);
\fill (0.5, 0) circle (0.1);
\draw (1, 0) circle (0.5);
\fill (1.5, 0) circle (0.1);
\draw (2, 0) circle (0.5);
\draw (3.5, 0) node{$ \to \S^1 \vee \S^1 = $};
\draw (5, 0) circle (0.5);
\fill (5.5, 0) circle (0.1);
\draw (6, 0) circle (0.5);
\end{tikzpicture}.
$$
\end{example*}

Let $ f : Y \to X $ be a continuous map. A lift of $ f $ is a continuous map $ \widetilde{f} : Y \to \widetilde{X} $ such that $ p\widetilde{f} = f $, where $ p : \widetilde{X} \to X $ is a covering space. Let $ Y $ be connected.
\begin{itemize}
\item \textbf{Unique lifting property} states that if two lifts $ \widetilde{f_1} $ and $ \widetilde{f_2} $ of $ f $ coincide at one point, then they coincide on all of $ Y $.
\item \textbf{Homotopy lifting property} states that if $ f_t : Y \to X $ is a homotopy and $ \widetilde{f_0} : Y \to \widetilde{X} $ is a lift of $ f_0 $ then there exists a unique homotopy $ \widetilde{f_t} : Y \to \widetilde{X} $ of $ \widetilde{f_0} $ that lifts $ f_t $.
\end{itemize}

\begin{remark*}
\hfill
\begin{itemize}
\item If $ Y $ is a point, this is called the \textbf{path lifting property}. Let $ f : \I \to X $ be a path with $ f\br{0} = x_0 $. If $ \widetilde{x_0} \in p^{-1}\br{x_0} $, then there is a unique path $ \widetilde{f} : \I \to \widetilde{X} $ lifting $ f $ and starting at $ \widetilde{x_0} $.
\item In particular, the lift of a constant path is constant.
\item This implies in particular that the lift of a homotopy of paths is again a homotopy of paths. The endpoints $ f_t\br{0} $ and $ f_t\br{1} $ define constant paths as $ t $ varies.
\end{itemize}
\end{remark*}

Fix $ x_0 \in X $ and $ \widetilde{x_0} \in \widetilde{X} $ such that $ p\br{\widetilde{x_0}} = x_0 $, so
$$ p_* : \pi_1\br{\widetilde{X}, \widetilde{x_0}} \to \pi_1\br{X, x_0}. $$
To every element in $ \pi_1\br{X, x_0} $ we can associate a homotopy class of paths in $ \widetilde{X} $ starting at $ \widetilde{x_0} $.

\pagebreak

\begin{proposition}
\label{prop:1.31}
\hfill
\begin{enumerate}
\item $ p_* : \pi_1\br{\widetilde{X}, \widetilde{x_0}} \to \pi_1\br{X, x_0} $ is injective.
\item $ p_*\br{\pi_1\br{\widetilde{X}, \widetilde{x_0}}} \subseteq \pi_1\br{X, x_0} $ consists of the homotopy classes of loops at $ x_0 $ whose lifts to $ \widetilde{X} $ starting at $ \widetilde{x_0} $ are loops.
\end{enumerate}
\end{proposition}

\begin{proof}
\hfill
\begin{enumerate}
\item Let $ \widetilde{f_0} : \I \to \widetilde{X} $ be a loop at $ \widetilde{x_0} $ such that $ \sbr{\widetilde{f_0}} \in \Ker p_* $, so $ p\widetilde{f_0} = f_0 $ is homotopic to the constant loop at $ x_0 $. Let $ f_t : \I \to X $ be a homotopy between $ f_0 $ and the constant loop. Homotopy lifting property and remark implies that $ f_t $ lifts to a homotopy $ \widetilde{f_t} $ of paths between $ \widetilde{f_0} $ and the constant loop, so $ \sbr{\widetilde{f_0}} = \id \in \pi_1\br{\widetilde{X}, \widetilde{x_0}} $ and $ p_* $ is injective.
\item Let $ f : \I \to X $ be a loop at $ x_0 $ that lifts to a loop $ \widetilde{f} $ at $ \widetilde{x_0} $. Then $ p\widetilde{f} = f $, so $ p_*\br{\sbr{\widetilde{f}}} = \sbr{f} $. On the other hand, if $ f : \I \to X $ is a loop at $ x_0 $ such that there exists a loop $ \widetilde{f} : \I \to \widetilde{X} $ at $ \widetilde{x_0} $ with $ p_*\br{\sbr{\widetilde{f}}} = \sbr{f} $, then $ f $ is homotopic to $ p\widetilde{f} $. Homotopy lifting property implies that there exists a loop $ \widetilde{f'} : \I \to \widetilde{X} $ at $ x_0 $ such that $ p\widetilde{f'} = f $.
\end{enumerate}
\end{proof}

\lecture{12}{Wednesday}{06/02/19}

Let $ p : \widetilde{X} \to X $ be a covering space. Let $ U \subseteq X $ be an evenly covered neighbourhood of $ x \in X $. Let
$$ p^{-1}\br{U} = \bigsqcup_{\alpha \in \Lambda} \widetilde{U_\alpha}. $$
Then the cardinality $ \abs{p^{-1}\br{x}} $ of $ p^{-1}\br{x} $ is exactly the cardinality of $ \abs{\Lambda} $. The set of sheets is in bijection with $ p^{-1}\br{x} $. So the cardinality of $ p^{-1}\br{x} $ is locally constant. If $ X $ is connected, the cardinality of $ p^{-1}\br{x} $ is constant.

\begin{notation*}
Let $ X $ and $ Y $ be topological spaces, $ x \in X $, and $ y \in Y $. A continuous map
$$ f : \br{X, x} \to \br{Y, y} $$
is a continuous map $ f : X \to Y $ such that $ f\br{x} = y $.
\end{notation*}

\begin{proposition}
Let $ X $ and $ \widetilde{X} $ be path-connected and
$$ p : \br{\widetilde{X}, \widetilde{x_0}} \to \br{X, x_0} $$
be a covering space. Then the number of sheets of $ p $ equals the index of $ p_*\br{\pi_1\br{\widetilde{X}, \widetilde{x_0}}} $ in $ \pi_1\br{X, x_0} $.
\end{proposition}

\begin{proof}
Let $ g $ be a loop in $ X $ at $ x_0 $ and $ \widetilde{g} $ be its lift to $ \widetilde{X} $ starting at $ \widetilde{x_0} $. Let $ H = p_*\br{\pi_1\br{\widetilde{X}, \widetilde{x_0}}} $ and let $ \sbr{h} \in H $. Then $ h \cdot g $ lifts to a path $ \widetilde{h} \cdot \widetilde{g} $ in $ \widetilde{X} $ starting at $ \widetilde{x_0} $ with the same endpoint as $ \widetilde{g} $, because $ \widetilde{h} $ is a loop, by Proposition \ref{prop:1.31}. Define
$$ \function[\Phi]{\cbr{\text{cosets of} \ H \ \text{in} \ \pi_1\br{X, x_0}}}{p^{-1}\br{x_0}}{H\sbr{g}}{\widetilde{g}\br{1}}, $$
so $ \Phi $ is well-defined. Want to show that $ \Phi $ is bijective.
\begin{itemize}
\item $ \Phi $ is surjective because $ \widetilde{X} $ is path-connected. Let $ \widetilde{g} $ be a path in $ \widetilde{X} $ from $ \widetilde{x_0} $ to any point $ \widetilde{x_0'} \in p^{-1}\br{x_0} $, then $ g = p \cdot \widetilde{g} $ and $ \Phi\br{H\sbr{g}} = \widetilde{x_0'} $.
\item $ \Phi $ is injective, since if $ \Phi\br{H\sbr{g_1}} = \Phi\br{H\sbr{g_2}} $ then the lift $ \widetilde{g_1} \cdot \widetilde{g_2}^{-1} $ of $ g_1 \cdot g_2^{-1} $ defines a loop in $ \widetilde{X} $ at $ \widetilde{x_0} $. Proposition \ref{prop:1.31} implies that $ \sbr{g_1}\sbr{g_2}^{-1} \in H $, so $ H\sbr{g_1} = H\sbr{g_2} $.
\end{itemize}
\end{proof}

\pagebreak

We say that a topological space $ X $ has a certain property $ \br{P} $ \textbf{locally} if for each point $ x \in X $ and each neighbourhood $ U $ of $ x $ there is an open neighbourhood $ V \subseteq U $ having this property $ \br{P} $.

\begin{example*}
$ X $ is locally path-connected or $ X $ is locally simply-connected.
\end{example*}

\begin{proposition}
\label{prop:1.33}
Let
$$ p : \br{\widetilde{X}, \widetilde{x_0}} \to \br{X, x_0} $$
be a covering space and
$$ f : \br{Y, y_0} \to \br{X, x_0} $$
a continuous map, where $ Y $ is path-connected and locally path-connected. Then there is a lift
$$ \widetilde{f} : \br{Y, y_0} \to \br{\widetilde{X}, \widetilde{x_0}} $$
if and only if
$$ f_*\br{\pi_1\br{Y, y_0}} \subseteq p_*\br{\pi_1\br{\widetilde{X}, \widetilde{x_0}}}, $$
so
$$
\begin{tikzcd}
& \br{\widetilde{X}, \widetilde{x_0}} \arrow{d}{p} \\
\br{Y, y_0} \arrow{ur}{\widetilde{f}} \arrow[swap]{r}{f} & \br{X, x_0}
\end{tikzcd}.
$$
\end{proposition}

\begin{proof}
\hfill
\begin{itemize}
\item[$ \implies $] Clear, because $ f = p\widetilde{f} $ implies $ f_* = p_*\widetilde{f_*} $.
\item[$ \impliedby $] Assume
$$ f_*\br{\pi_1\br{Y, y_0}} \subseteq p_*\br{\pi_1\br{\widetilde{X}, \widetilde{x_0}}}. $$
For each $ y \in Y $ choose a path $ \gamma $ from $ y_0 $ to $ y $, so $ f\gamma $ is a path in $ X $ from $ x_0 $ to $ f\br{y} $. By path lifting, we can lift $ f\gamma $ to a path $ \widetilde{f\gamma} $ in $ \widetilde{X} $ starting at $ \widetilde{x_0} $. Define the map
$$ \function[\widetilde{f}]{\br{Y, y_0}}{\br{\widetilde{X}, \widetilde{x_0}}}{y}{\widetilde{f\gamma}\br{1}}. $$
\begin{itemize}
\item This map is well-defined, that is does not depend on the choice of $ \gamma $. Let $ \gamma' $ be another path from $ y_0 $ to $ y $. Then $ h_0 = \br{f\gamma'} \cdot \br{f\gamma}^{-1} $ is a loop at $ x_0 $ and
$$ \sbr{h_0} \in f_*\br{\pi_1\br{Y, y_0}} \subseteq p_*\br{\pi_1\br{\widetilde{X}, \widetilde{x_0}}}. $$
Proposition \ref{prop:1.31} implies that can lift $ h_0 $ to a loop $ \widetilde{h_0} $ at $ \widetilde{x_0} $. The first half of $ \widetilde{h_0} $ is $ \widetilde{f\gamma'} $ and the second half is $ \widetilde{f\gamma}^{-1} $, so $ \widetilde{f\gamma}\br{1} = \widetilde{f\gamma'}\br{1} $. Thus $ \widetilde{f} $ is well-defined.
\item We have $ p\widetilde{f} = f $, so $ \widetilde{f} $ lifts $ f $.
\item It remains to show that $ \widetilde{f} $ is continuous. Let $ y \in Y $ and let $ U $ be an evenly covered neighbourhood of $ f\br{y} $. Let $ \widetilde{U} $ be the sheet above $ U $ such that $ \widetilde{f}\br{y} \in \widetilde{U} $, so $ \eval{p}_{\widetilde{U}} : \widetilde{U} \to U $ is a homeomorphism. Let $ V \subseteq Y $ be a path-connected neighbourhood of $ y $ such that $ f\br{V} \subseteq U $. Fix a path $ \gamma $ from $ y_0 $ to $ y $. Let $ y' \in V $ be arbitrary and $ \eta $ be a path from $ y $ to $ y' $, so $ \gamma \cdot \eta $ is a path from $ y_0 $ to $ y' $. Then $ \br{f\gamma} \cdot \br{f\eta} $ is a path in $ U $ from $ x_0 $ to $ f\br{y'} $, and $ \widetilde{f\eta} = \br{\eval{p}_{\widetilde{U}}}^{-1}f\eta $, so $ \eval{\widetilde{f}}_V = \br{\eval{p}_{\widetilde{U}}}^{-1}f $. Thus $ \eval{\widetilde{f}}_V : V \to \widetilde{U} $ is continuous, so $ \widetilde{f} $ is continuous.
\end{itemize}
\end{itemize}
\end{proof}

\pagebreak

\subsubsection{The classification of covering spaces}

\lecture{13}{Friday}{08/02/19}

\begin{definition*}
A covering space $ p : \widetilde{X} \to X $ is a \textbf{universal cover} if $ \widetilde{X} $ is simply-connected.
\end{definition*}

\begin{definition*}
A topological space $ X $ is \textbf{semilocally simply-connected} if each $ x \in X $ has a neighbourhood $ U $ such that
$$ i_* : \pi_1\br{U, x} \to \pi_1\br{X, x} $$
is trivial, where $ i : U \hookrightarrow X $ is the inclusion.
\end{definition*}

\begin{example*}
Let $ X = \bigcup_n C_n \subseteq \RR^2 $ be the \textbf{Hawaiian earrings}, where $ C_n \subseteq \RR^2 $ is the circle of radius $ 1 / n $ and centre $ \br{1 / n, 0} $. Then $ X $ is not semilocally simply-connected.
\end{example*}

\begin{proposition}
If $ p : \widetilde{X} \to X $ is a universal cover, then $ X $ is semilocally simply-connected.
\end{proposition}

\begin{proof}
Let $ U \subseteq X $ be an evenly covered neighbourhood of $ x_0 \in X $, $ \widetilde{U} \subseteq \widetilde{X} $ be a sheet over $ U $, and $ \gamma \subseteq U $ be a loop at $ x_0 $, so $ \gamma $ lifts to a loop $ \widetilde{\gamma} \subseteq \widetilde{U} $ at $ \widetilde{x_0} $. Then $ \widetilde{\gamma} $ is homotopic to the constant loop at $ \widetilde{x_0} $. Composing this homotopy with $ p $ implies that $ \gamma $ is homotopic to the constant loop at $ x_0 $ in $ X $, so
$$ \pi_1\br{U, x_0} \to \pi_1\br{X, x_0} $$
is trivial.
\end{proof}

\begin{theorem}
Let $ X $ be path-connected, locally path-connected, and semilocally simply-connected. Then there exists a universal cover $ p : \widetilde{X} \to X $.
\end{theorem}

\begin{remark*}
If
$$ p : \br{\widetilde{X}, \widetilde{x_0}} \to \br{X, x_0} $$
is a universal cover, each point $ \widetilde{x} \in \widetilde{X} $ can be joined to $ \widetilde{x_0} $ by a unique homotopy class of paths, by Proposition \ref{prop:1.6}.
$$ \cbr{\text{points in} \ \widetilde{X}} \ \leftrightsquigarrow \ \cbr{\sbr{\gamma} \st \gamma \ \text{is a path in} \ \widetilde{X} \ \text{starting at} \ \widetilde{x_0}} \ \leftrightsquigarrow \ \cbr{\sbr{\gamma} \st \gamma \ \text{is a path in} \ X \ \text{starting at} \ x_0}, $$
by the homotopy lifting property.
\end{remark*}

\begin{proof}
Let $ x_0 \in X $, and
$$ \widetilde{X} = \cbr{\sbr{\gamma} \st \gamma \ \text{is a path in} \ X \ \text{starting at} \ x_0}, \qquad \function[p]{\widetilde{X}}{X}{\sbr{\gamma}}{\gamma\br{1}}. $$
Have to
\begin{enumerate}
\item give $ \widetilde{X} $ a topology,
\item show that $ p : \widetilde{X} \to X $ is a covering, and
\item show that $ \widetilde{X} $ is simply-connected.
\end{enumerate}
Recall that a \textbf{basis} for a topology on a set $ Y $ is a collection $ \BBB $ of subsets such that
\begin{itemize}
\item $ Y = \bigcup_{U \in \BBB} U $, and
\item if $ U_1, U_2 \in \BBB $ and $ y \in U_1 \cap U_2 $ then there exists $ V \in \BBB $ such that $ y \in V $ and $ V \subseteq U_1 \cap U_2 $.
\end{itemize}
A basis defines a topology on $ Y $, by $ A \subseteq Y $ is open if and only if $ A $ is the union of elements of $ \BBB $. A map $ f : Z \to Y $ is continuous if and only if $ f^{-1}\br{U} $ is open for all $ U \in \BBB $.

\pagebreak

\begin{enumerate}
\item Let $ \UUU $ be the collection of all path-connected open sets $ U \subseteq X $ such that $ \pi_1\br{U} \to \pi_1\br{X} $ is trivial. Then $ X = \bigcup_{U \in \UUU} U $ because $ X $ is semilocally simply-connected. Let $ U_1, U_2 \in \UUU $ and $ y \in U_1 \cap U_2 $, and let $ y \in V \subseteq U_1 \cap U_2 $ be path-connected and open. Then
$$
\begin{tikzcd}
V \arrow[hookrightarrow]{r} & U_1 \arrow[hookrightarrow]{r} & X \\
\pi_1\br{V} \arrow{r} \arrow[bend right=15, swap]{rr}{\text{trivial}} & \pi_1\br{U_1} \arrow{r}{\text{trivial}} & \pi_1\br{X}
\end{tikzcd},
$$
so $ V \in \UUU $, so $ \UUU $ is a basis for the topology on $ X $. For $ U \in \UUU $ and $ \gamma $ a path in $ X $ from $ x_0 $ to a point in $ U $, we define
$$ U_{\sbr{\gamma}} = \cbr{\sbr{\gamma \cdot \eta} \st \eta \ \text{a path in} \ U \ \text{such that} \ \eta\br{0} = \gamma\br{1}} \subseteq \widetilde{X}. $$
$ U_{\sbr{\gamma}} $ only depends on the class $ \sbr{\gamma} $, so $ \eval{p}_{U_{\sbr{\gamma}}} : U_{\sbr{\gamma}} \to U $ is bijective. Surjective because $ U $ is path-connected and injective because all paths $ \eta $ in $ U $ with the same endpoint are homotopic. Claim that $ \cbr{U_{\sbr{\gamma}}} $ forms a basis on $ \widetilde{X} $.
\begin{itemize}
\item $ \bigcup_{U \in \UUU, \ \gamma} U_{\sbr{\gamma}} = \widetilde{X} $, because $ \bigcup_{U \in \UUU} U = X $.
\item Observe that if $ \sbr{\gamma'} \in U_{\sbr{\gamma}} $ then $ U_{\sbr{\gamma}} = U_{\sbr{\gamma'}} $. If $ \gamma' = \gamma \cdot \eta $ for $ \eta $ a path in $ U $, then elements in $ U_{\sbr{\gamma'}} $ have the form $ \sbr{\gamma \cdot \eta \cdot \mu} $, so $ U_{\sbr{\gamma'}} \subseteq U_{\sbr{\gamma}} $. The elements in $ U_{\sbr{\gamma}} $ have the form
$$ \sbr{\gamma \cdot \mu} = \sbr{\gamma \cdot \eta \cdot \eta^{-1} \cdot \mu} = \sbr{\gamma' \cdot \eta^{-1} \cdot \mu}, $$
so $ U_{\sbr{\gamma}} \subseteq U_{\sbr{\gamma'}} $. Consider $ U_{\sbr{\gamma}} $ and $ V_{\sbr{\gamma'}} $ and let $ \sbr{\gamma''} \in U_{\sbr{\gamma}} \cap V_{\sbr{\gamma'}} $, so $ U_{\sbr{\gamma}} = U_{\sbr{\gamma''}} $ and $ V_{\sbr{\gamma'}} = V_{\sbr{\gamma''}} $. Let $ W \in \UUU $ such that $ W \subseteq U \cap V $ and such that $ \gamma''\br{1} \in W $, so $ W_{\sbr{\gamma''}} \subseteq U_{\sbr{\gamma''}} \cap V_{\sbr{\gamma''}} $ and $ \sbr{\gamma''} \in W_{\sbr{\gamma''}} $. This proves the claim.
\end{itemize}
\item $ \eval{p}_{U_{\sbr{\gamma}}} : U_{\sbr{\gamma}} \to U $ is a homeomorphism. It is bijective, let $ V_{\sbr{\gamma'}} \subseteq U_{\sbr{\gamma}} $ be an element of the basis, so $ p\br{V_{\sbr{\gamma'}}} = V \in \UUU $. Then $ p^{-1}\br{V} \cap U_{\sbr{\gamma}} = V_{\sbr{\gamma'}} $. Thus $ p : \widetilde{X} \to X $ is continuous. If $ U \in \UUU $, then
$$ p^{-1}\br{U} = \bigsqcup_{\sbr{\gamma}} U_{\sbr{\gamma}}, $$
so $ p : \widetilde{X} \to X $ is a covering space.
\item Let $ \widetilde{x_0} \in \widetilde{X} $ be the class of the constant path at $ x_0 $. Let $ \sbr{\gamma} \in \widetilde{X} $ be arbitrary. Then $ \gamma : \sbr{0, 1} \to X $ and $ \gamma\br{0} = x_0 $. Let $ \gamma_t $ be the path in $ X $ defined by
$$ \gamma_t\br{s} =
\begin{cases}
\gamma\br{s} & s \in \sbr{0, t} \\
\gamma\br{t} & s \in \sbr{t, 1}
\end{cases}.
$$
Then
$$ \function[\widetilde{\gamma}]{\I}{\widetilde{X}}{t}{\sbr{\gamma_t}} $$
is a path in $ \widetilde{X} $ from $ \widetilde{x_0} $ to $ \sbr{\gamma} $, so $ \widetilde{X} $ is path-connected. Recall that $ p_*\br{\pi_1\br{\widetilde{X}, \widetilde{x_0}}} $ consists of the classes of loops at $ x_0 $ in $ X $ that lifts to loops in $ \widetilde{X} $ at $ \widetilde{x_0} $. Let $ \sbr{\gamma} \in p_*\br{\pi_1\br{\widetilde{X}, \widetilde{x_0}}} $. Then $ \gamma $ lifts to a loop at $ \widetilde{x_0} $ by $ t \mapsto \sbr{\gamma_t} $. Because it is a loop we have $ \widetilde{x_0} = \sbr{\gamma_1} = \sbr{\gamma} $, so $ \gamma $ is homotopic to the constant loop. Thus $ p_*\br{\pi_1\br{\widetilde{X}, \widetilde{x_0}}} = \cbr{\id} $, so $ \widetilde{X} $ is simply-connected.
\end{enumerate}
\end{proof}

\lecture{14}{Tuesday}{12/02/19}

Lecture 14 is a problems class.

\pagebreak

\lecture{15}{Wednesday}{13/02/19}

Let $ p : \widetilde{X} \to X $ be a covering space, so $ p_*\br{\pi_1\br{\widetilde{X}, \widetilde{x_0}}} \subseteq \pi_1\br{X, x_0} $.

\begin{proposition}
\label{prop:1.36}
Let $ X $ be path-connected, locally path-connected, and semilocally simply-connected. Then for every subgroup $ H \subseteq \pi_1\br{X, x_0} $ there is a covering space $ p : X_H \to X $ such that $ p_*\br{\pi_1\br{X_H, \widetilde{x_0}}} = H $ for some basepoint $ x_0 $.
\end{proposition}

\begin{proof}
Let $ \widetilde{X} $ be as constructed above. Define $ X_H = \widetilde{X} / \sim $, where $ \sbr{\gamma} \sim \sbr{\gamma'} $ if $ \gamma\br{1} = \gamma'\br{1} $ and $ \sbr{\gamma \cdot \br{\gamma'}^{-1}} \in H $. This is an equivalence relation.
\begin{itemize}
\item $ \sbr{\gamma} \sim \sbr{\gamma} $ because $ \id \in H $.
\item $ \sbr{\gamma} \sim \sbr{\gamma'} $ implies that $ \sbr{\gamma'} \sim \sbr{\gamma} $ because $ H $ contains all its inverses.
\item $ \sbr{\gamma} \sim \sbr{\gamma'} $ and $ \sbr{\gamma'} \sim \sbr{\gamma''} $ implies that $ \sbr{\gamma} \sim \sbr{\gamma''} $ because $ H $ is closed under product.
\end{itemize}
Then
$$
\begin{tikzcd}
\widetilde{X} \arrow{r} \arrow{d} & \widetilde{X} / \sim = X_H \arrow{dl}{p} \\
X &
\end{tikzcd}.
$$
Let $ U_{\sbr{\gamma}} $ and $ U_{\sbr{\gamma'}} $ be basis neighbourhoods. If $ \sbr{\gamma} \sim \sbr{\gamma'} $ then $ \sbr{\gamma \cdot \eta} \sim \sbr{\gamma' \cdot \eta} $, so $ p $ is a covering space, and $ p^{-1}\br{U} = \bigcup_\gamma U_{\sbr{\gamma}} $. Let $ \widetilde{x_0} \in X_H $ be the equivalence class of the constant path $ \c_{x_0} $ at $ x_0 $. Let $ \gamma $ be a loop in $ X $ at $ x_0 $ such that $ \sbr{\gamma} \in p_*\br{\pi_1\br{X_H, \widetilde{x_0}}} $. Again $ t \mapsto \sbr{\gamma_t} $ is a lift of $ \gamma $ at $ \widetilde{x_0} $. Then
$$ t \mapsto \sbr{\gamma_t} \ \text{is a loop in} \ X_H \quad \iff \quad \sbr{\gamma_1} = \sbr{\gamma} = \sbr{\c_{x_0}} \ \text{in} \ X_H \quad \iff \quad \sbr{\gamma} \sim \sbr{\c_{x_0}} \quad \iff \quad \gamma \in H. $$
\end{proof}

\begin{definition*}
We say that two covering spaces $ p_1 : \widetilde{X_1} \to X $ and $ p_2 : \widetilde{X_2} \to X $ are \textbf{isomorphic} if there exists a homeomorphism $ f : \widetilde{X_1} \to \widetilde{X_2} $ such that
$$
\begin{tikzcd}
\widetilde{X_1} \arrow{rr}{f} \arrow[swap]{dr}{p_1} & & \widetilde{X_2} \arrow{dl}{p_2} \\
& X &
\end{tikzcd}.
$$
\end{definition*}

\begin{proposition}
\label{prop:1.37}
Let $ X $ be path-connected and locally path-connected and $ x_0 \in X $. Two path-connected covering spaces $ p_1 : \widetilde{X_1} \to X $ and $ p_2 : \widetilde{X_2} \to X $ are isomorphic via an isomorphism $ f : \widetilde{X_1} \to \widetilde{X_2} $ mapping a basepoint $ \widetilde{x_1} \in p_1^{-1}\br{x_0} $ to a basepoint $ \widetilde{x_2} \in p_2^{-1}\br{x_0} $ if and only if
$$ p_{1*}\br{\pi_1\br{\widetilde{X_1}, \widetilde{x_1}}} = p_{2*}\br{\pi_1\br{\widetilde{X_2}, \widetilde{x_2}}}. $$
\end{proposition}

\begin{proof}
\hfill
\begin{itemize}
\item[$ \implies $] If
$$ f : \br{\widetilde{X_1}, \widetilde{x_1}} \to \br{\widetilde{X_2}, \widetilde{x_2}} $$
is an isomorphism, then $ p_1 = p_2f $, so
$$ p_{1*}\br{\pi_1\br{\widetilde{X_1}, \widetilde{x_1}}} \subseteq p_{2*}\br{\pi_1\br{\widetilde{X_2}, \widetilde{x_2}}}, $$
and $ p_2 = p_1f^{-1} $, so
$$ p_{2*}\br{\pi_1\br{\widetilde{X_2}, \widetilde{x_2}}} \subseteq p_{1*}\br{\pi_1\br{\widetilde{X_1}, \widetilde{x_1}}}. $$

\pagebreak

\item[$ \impliedby $] Assume
$$ p_{1*}\br{\pi_1\br{\widetilde{X_1}, \widetilde{x_1}}} = p_{2*}\br{\pi_1\br{\widetilde{X_2}, \widetilde{x_2}}}. $$
By lifting criterion in Proposition \ref{prop:1.33}, we can lift $ p_1 $ to a continuous map
$$ \widetilde{p_1} : \br{\widetilde{X_1}, \widetilde{x_1}} \to \br{\widetilde{X_2}, \widetilde{x_2}}, $$
and $ p_2 $ to a continuous map
$$ \widetilde{p_2} : \br{\widetilde{X_2}, \widetilde{x_2}} \to \br{\widetilde{X_1}, \widetilde{x_1}}, $$
so $ p_1\widetilde{p_2} = p_2 $ and $ p_2\widetilde{p_1} = p_1 $.
$$
\begin{tikzcd}
\br{\widetilde{X_1}, \widetilde{x_1}} \arrow[bend left=15]{rr}{\widetilde{p_1}} \arrow[swap]{dr}{p_1} & & \br{\widetilde{X_2}, \widetilde{x_2}} \arrow{dl}{p_2} \arrow[bend left=15]{ll}{\widetilde{p_2}} \\
& \br{X, x_0} &
\end{tikzcd}.
$$
Then $ \widetilde{p_1}\widetilde{p_2} $ fixes the point $ \widetilde{x_2} \in \widetilde{X_2} $. By the unique lifting property in Proposition \ref{prop:1.34}, $ \widetilde{p_1}\widetilde{p_2} = \id_{\widetilde{x_2}} $. Similarly, $ \widetilde{p_2}\widetilde{p_1} = \id_{\widetilde{x_1}} $, so $ \widetilde{p_1} $ is an isomorphism.
\end{itemize}
\end{proof}

\lecture{16}{Friday}{15/02/19}

Fix $ x_0 \in X $, $ \widetilde{x_1} \in p_1^{-1}\br{x_0} $, and $ \widetilde{x_2} \in p_2^{-1}\br{x_0} $. A \textbf{basepoint preserving isomorphism}
$$ f : \br{\widetilde{X_1}, \widetilde{x_1}} \to \br{\widetilde{X_2}, \widetilde{x_2}} $$
is an isomorphism such that $ f\br{\widetilde{x_1}} = \widetilde{x_2} $.

\begin{theorem}[Galois correspondence]
Let $ X $ be path-connected, locally path-connected, and semilocally simply-connected, and $ x_0 \in X $. Then
\begin{enumerate}
\item there is a bijection
$$ \correspondence{\text{path-connected covering spaces} \ p : \br{\widetilde{X}, \widetilde{x_0}} \to \br{X, x_0} \\ \text{up to basepoint preserving isomorphisms}}{\text{subgroups} \\ H \subseteq \pi_1\br{X, x_0}}, $$
\item if we ignore the basepoints, this correspondence gives a bijection
$$ \correspondence{\text{path-connected covering spaces} \ p : \widetilde{X} \to X \\ \text{up to isomorphisms}}{\text{conjugacy classes of subgroups} \\ H \subseteq \pi_1\br{X, x_0}}. $$
\end{enumerate}
\end{theorem}

\begin{proof}
\hfill
\begin{enumerate}
\item To a covering space
$$ p : \br{\widetilde{X}, \widetilde{x_0}} \to \br{X, x_0}, $$
we associate the subgroup
$$ p_*\br{\pi_1\br{\widetilde{X}, \widetilde{x_0}}} \subseteq \pi_1\br{X, x_0}. $$
Proposition \ref{prop:1.36} and Proposition \ref{prop:1.37} show that this is well-defined on the isomorphism classes and it is bijective.

\pagebreak

\item Let $ p : \widetilde{X} \to X $ be a covering space and $ \widetilde{x_1}, \widetilde{x_2} \in p^{-1}\br{x_0} $. Let
$$ H_i = p_*\br{\pi_1\br{\widetilde{X}, \widetilde{x_i}}} \subseteq \pi_1\br{X, x_0}, \qquad i = 1, 2. $$
Let $ \widetilde{\gamma} $ be a path from $ \widetilde{x_1} $ to $ \widetilde{x_2} $. Let $ \gamma = p\widetilde{\gamma} $ be a loop at $ x_0 $. Let $ \sbr{f} \in \pi_1\br{X, x_0} $. Then $ \sbr{f} \in H_1 $ if and only if the lift $ \widetilde{f} $ is a loop at $ \widetilde{x_1} $. Then $ \widetilde{\gamma}^{-1} \cdot \widetilde{f} \cdot \widetilde{\gamma} $ is a loop at $ \widetilde{x_2} $, so
$$ p_*\br{\widetilde{\gamma}^{-1} \cdot \widetilde{f} \cdot \widetilde{\gamma}} = \gamma^{-1} \cdot f \cdot \gamma, $$
so $ \sbr{\gamma}^{-1}\sbr{f}\sbr{\gamma} \in H_2 $. Thus $ \sbr{\gamma}^{-1}H_1\sbr{\gamma} \subseteq H_2 $. Similarly, $ \sbr{\gamma}H_2\sbr{\gamma}^{-1} \subseteq H_1 $. Conversely, let $ H_1 \subseteq \pi_1\br{X, x_0} $ as above and $ \sbr{\delta} \in \pi_1\br{X, x_0} $ be an arbitrary element. Let $ \widetilde{\delta} $ be a lift of $ \delta $ such that $ \widetilde{\delta}\br{0} = \widetilde{x_0} $ and define $ \widetilde{x_3} = \widetilde{\delta}\br{1} $. Then the same construction yields
$$ p_*\br{\pi_1\br{\widetilde{X}, \widetilde{x_3}}} = \sbr{\delta}^{-1}H_1\sbr{\delta}. $$
\end{enumerate}
\end{proof}

\subsubsection{Deck transformations and group actions}

\begin{definition*}
Let $ p : \widetilde{X} \to X $ be a covering space. A \textbf{deck-transformation} is an isomorphism from $ \widetilde{X} $ to itself.
$$
\begin{tikzcd}
\widetilde{X} \arrow{rr}{f} \arrow[swap]{dr}{p} & & \widetilde{X} \arrow{dl}{p} \\
& X &
\end{tikzcd}.
$$
The group of deck-transformations is denoted by $ \G\br{\widetilde{X}} $.
\end{definition*}

\begin{example*}
\hfill
\begin{itemize}
\item Let
$$ \function[p]{\RR}{\S^1 \subseteq \CC}{t}{e^{2\pi i t}}. $$
Then $ f : \RR \to \RR $ such that $ p\br{f\br{t}} = p\br{t} $ if and only if $ e^{2\pi if\br{t}} = e^{2\pi it} $, if and only if $ f\br{t} = t + n $, so $ \G\br{\RR} \cong \ZZ $.
\item Let
$$ \function[p]{\S^1}{\S^1}{z}{z^n}. $$
Then $ \G\br{\S^1} \cong \ZZ / n\ZZ $.
\end{itemize}
\end{example*}

An observation is that if $ \widetilde{X} $ is path-connected then $ f \in \G\br{\widetilde{X}} $ is uniquely determined by where it sends a single point.
$$
\begin{tikzcd}
\widetilde{X} \arrow[swap]{rr}{f} \arrow[bend left=30]{rr}{f'} \arrow[swap]{dr}{p} & & \widetilde{X} \arrow{dl}{p} \\
& X &
\end{tikzcd}.
$$
If $ f\br{x} = f'\br{x} $ for a single $ x $, by unique lifting $ f = f' $. So the identity is the only deck-transformation with a fixed point.

\pagebreak

\begin{definition*}
A covering space $ p : \widetilde{X} \to X $ is \textbf{normal}, or \textbf{regular}, or \textbf{Galois}, if for each $ x \in X $ and every pair $ \widetilde{x}, \widetilde{x'} \in p^{-1}\br{x} $ there is an $ f \in \G\br{\widetilde{X}} $ such that $ f\br{\widetilde{x}} = \widetilde{x'} $.
\end{definition*}

\begin{example*}
\hfill
\begin{itemize}
\item $ p : \RR \to \S^1 $ is normal.
\item $ p : \S^1 \to \S^1 $ is normal.
\end{itemize}
\end{example*}

\begin{proposition}
Let
$$ p : \br{\widetilde{X}, \widetilde{x_0}} \to \br{X, x_0} $$
be a path-connected covering space, and $ X $ be path-connected and locally path-connected. Then $ p : \widetilde{X} \to X $ is normal if and only if
$$ H = p_*\br{\pi_1\br{\widetilde{X}, \widetilde{x_0}}} \subseteq \pi_1\br{X, x_0} $$
is a normal subgroup.
\end{proposition}

\begin{proof}
Let $ \widetilde{x_1} \in p^{-1}\br{x_0} $, let $ \widetilde{\gamma} $ be a path from $ \widetilde{x_0} $ to $ \widetilde{x_1} $ and $ \gamma = p\br{\widetilde{\gamma}} $. Then $ \sbr{\gamma} $ conjugates $ H $ to $ p_*\br{\pi_1\br{\widetilde{X}, \widetilde{x_1}}} $ so $ \sbr{\gamma}H\sbr{\gamma}^{-1} = H $, if and only if $ H = p_*\br{\pi_1\br{\widetilde{X}, \widetilde{x_1}}} $, by Proposition \ref{prop:1.37} if and only if $ f\br{\widetilde{x_0}} = \widetilde{x_1} $. So $ \G\br{\widetilde{X}} $ acts transitively on $ p^{-1}\br{x_0} $ if and only if $ H \subseteq \pi_1\br{X, x_0} $ is a normal subgroup. Let $ x_0' \in X $ be another point and $ h $ a path from $ x_0 $ to $ \widetilde{x_0} $. Let $ \widetilde{h} $ be a lift of $ h $ such that $ \widetilde{h}\br{0} = \widetilde{x_0} $. Set $ \widetilde{x_0} = \widetilde{h}\br{1} $ and $ p\br{\widetilde{x_0'}} = x_0' $. Then
$$
\begin{tikzcd}
\pi_1\br{\widetilde{X}, \widetilde{x_0}} \arrow{r}{\beta_{\widetilde{h}}} \arrow[swap]{d}{p_*} & \pi_1\br{\widetilde{X}, \widetilde{x_0'}} \arrow{d}{p_*} \\
\pi_1\br{X, x_0} \arrow[swap]{r}{\beta_h} & \pi_1\br{X, x_0'}
\end{tikzcd}.
$$
Thus $ H \subseteq \pi_1\br{X, x_0} $ is normal if and only if
$$ p_*\br{\pi_1\br{\widetilde{X}, \widetilde{x_0'}}} \subseteq \pi_1\br{X, x_0'} $$ is normal, as before if and only if $ \G\br{\widetilde{X}} $ acts transitively on $ p^{-1}\br{x_0'} $.
\end{proof}

\lecture{17}{Tuesday}{19/02/19}

\begin{proposition}
Let
$$ p : \br{\widetilde{X}, \widetilde{x_0}} \to \br{X, x_0} $$
be a covering space, $ X $ be path-connected and locally path-connected, and $ \widetilde{X} $ be path-connected. Let $ H = p_*\br{\pi_1\br{\widetilde{X}, \widetilde{x_0}}} $ and $ N\br{H} \subseteq \pi_1\br{X, x_0} $ be the normaliser of $ H $. Then $ \G\br{\widetilde{X}} $ is isomorphic to $ N\br{H} / H $. In particular,
\begin{itemize}
\item if $ \widetilde{X} $ is normal, then
$$ \G\br{\widetilde{X}} \cong \pi_1\br{X, x_0} / H, $$
\item if $ \widetilde{X} $ is the universal cover, then
$$ \G\br{\widetilde{X}} \cong \pi_1\br{X, x_0}. $$
\end{itemize}
\end{proposition}

\begin{proof}
Read the proof of this in Hatcher. \footnote{Exercise}
\end{proof}

\pagebreak

\begin{example*}
Let $ X = \S^1 \vee \S^1 $, so $ \pi_1\br{X} = \F_{\cbr{a, b}} $. Then the following are covering spaces.
\begin{itemize}
\item A normal covering space
$$
\begin{tikzpicture}
\draw (-1, 0) node{$ \widetilde{X} = $};
\draw (0, 0) circle (0.5) node[above]{$ a $};
\fill (0.5, 0) circle (0.1) node[below]{$ \widetilde{x_0} $};
\draw (1, 0) circle (0.5) node[above]{$ b $};
\fill (1.5, 0) circle (0.1);
\draw (2, 0) circle (0.5) node[above]{$ a $};
\draw (7, 0) node{$ p_*\br{\pi_1\br{\widetilde{X}, \widetilde{x_0}}} = \abr{a, b^2, bab^{-1}} \overset{2}{\subseteq} \F_{\cbr{a, b}} $};
\end{tikzpicture}.
$$
In general, a two-oriented graph is a covering space of $ X $.
\item Not a normal covering space
$$
\begin{tikzpicture}
\draw (-1, 0) node{$ \widetilde{X} = $};
\draw (0, 0) circle (0.5) node[above]{$ a $};
\fill (0.5, 0) circle (0.1);
\draw (1, 0) circle (0.5) node[above]{$ b $};
\fill (1.5, 0) circle (0.1) node[below]{$ \widetilde{x_0} $};
\draw (2, 0) circle (0.5) node[above]{$ a $};
\fill (2.5, 0) circle (0.1);
\draw (3, 0) circle (0.5) node[above]{$ b $};
\draw (8, 0) node{$ p_*\br{\pi_1\br{\widetilde{X}, \widetilde{x_0}}} = \abr{b^2, bab^{-1}, a^2, aba^{-1}} $};
\end{tikzpicture}.
$$
\item A normal covering space
$$
\begin{tikzpicture}
\draw (-0.5, 0) node{$ \widetilde{X} = $};
\draw [dashed] (0, -0.5) to (1, -0.5);
\fill (1, -0.5) circle (0.1);
\draw (1, 0) circle (0.5) node[above]{$ a $};
\draw (1, -0.5) to node[above]{$ b $} (2.5, -0.5);
\fill (2.5, -0.5) circle (0.1) node[below]{$ \widetilde{x_0} $};
\draw (2.5, 0) circle (0.5) node[above]{$ a $};
\draw (2.5, -0.5) to node[above]{$ b $} (4, -0.5);
\fill (4, -0.5) circle (0.1);
\draw (4, 0) circle (0.5) node[above]{$ a $};
\draw [dashed] (4, -0.5) to (5, -0.5);
\draw (9, 0) node{$ p_*\br{\pi_1\br{\widetilde{X}, \widetilde{x_0}}} = \abr{b^nab^{-n} \st n \in \ZZ} $};
\end{tikzpicture}.
$$
The universal cover is a tree.
\end{itemize}
\end{example*}

\begin{example*}
Let $ T = \S^1 \times \S^1 $, so $ \pi_1\br{T} = \ZZ^2 $. This is abelian, so all covering spaces are normal. The universal cover is
$$ \function{\RR^2}{\S^1 \times \S^1}{\br{s, t}}{\br{e^{2\pi is}, e^{2\pi it}}}, $$
since $ \RR^2 $ is simply connected. Check that it is a covering space. \footnote{Exercise} More generally, if $ p : \widetilde{X} \to X $ and $ q : \widetilde{Y} \to Y $ are covering spaces then
$$ \function{\widetilde{X} \times \widetilde{Y}}{X \times Y}{\br{x, y}}{\br{p\br{x}, q\br{y}}} $$
is again a covering space. For example,
$$ \function{\S^1 \times \S^1}{\S^1 \times \S^1}{\br{z_1, z_2}}{\br{z_1^n, z_2^m}}. $$
\end{example*}

\begin{example*}
Worksheet $ 3 $ exercise $ 7 $. Let
$$ \RP^n = \RR^{n + 1} \setminus \cbr{0} / \sim = \S^n / \sim $$
be the \textbf{projective $ n $-space}, the space of all lines through the origin in $ \RR^{n + 1} $, where $ x \sim -x $. Let $ p : \S^n \to \RP^n $ be the quotient map. Claim that this is a covering space. Let $ \sbr{x} \in \RP^n $. Then $ p^{-1}\br{\sbr{x}} = \cbr{\pm x} $. Let $ U $ be an open neighbourhood of $ x $ such that $ U \cap \br{-U} = \emptyset $, so $ p\br{U} = \cbr{\sbr{x} \st x \in U} $. Then $ p^{-1}\br{p\br{U}} = U \cup \br{-U} $ is open and disjoint. Thus $ \eval{p}_U : U \to p\br{U} $ is a homeomorphism, so it is a covering space.
\begin{itemize}
\item $ n \ge 2 $ implies that $ \S^n $ is simply-connected, so $ \S^n \to \RP^n $ is a universal cover. Then
$$ \cbr{\id} = p_*\br{\pi_1\br{\S^n}} \overset{2}{\subseteq} \pi_1\br{\RP^n}, $$
so $ \abs{\pi_1\br{\RP^n}} = 2 $. Thus $ \pi_1\br{\RP^n} \cong \ZZ / 2\ZZ $.
\item $ n = 1 $ implies that $ \RP^1 = \S^1 $, so
$$ \function[p]{\S^1}{\S^1}{z}{z^2} $$
is a covering space.
\end{itemize}
\end{example*}

\pagebreak

\section{Homology}

\lecture{18}{Wednesday}{20/02/19}

Higher homotopy groups $ \pi_n\br{X, x_0} $ are groups of basepoint preserving homotopies of continuous $ \phi : \I^n \to X $ such that $ \phi\br{\partial \I^n} = x_0 $.

\begin{example*}
$$ \pi_1\br{\S^n} =
\begin{cases}
\ZZ & n = 1 \\
0 & \text{otherwise}
\end{cases},
\qquad \pi_2\br{\S^n} =
\begin{cases}
\ZZ & n = 2 \\
0 & \text{otherwise}
\end{cases},
$$
$$ \pi_3\br{\S^n} =
\begin{cases}
\ZZ & n = 2, 3 \\
0 & \text{otherwise}
\end{cases},
\qquad \pi_i\br{\S^2} =
\begin{cases}
\ZZ / 2\ZZ & i = 4, 5 \\
\ZZ / 12\ZZ & i = 6
\end{cases}.
$$
\end{example*}

Homology is more suitable. The following is the plan.
\begin{itemize}
\item Simplicial homology.
\item Singular homology.
\item Technical machinery to show that they coincide.
\item Applications.
\end{itemize}

\subsection{\texorpdfstring{$ \Delta $}{Delta}-complexes}

\begin{definition*}
Let $ m, n \ge 0 $.
\begin{itemize}
\item An \textbf{$ n $-simplex} in $ \RR^m $ is the convex hull of a set $ V $ of $ n + 1 $ points in $ \RR^m $ that are not all contained in an affine $ \br{n - 1} $-dimensional subspace of $ \RR^m $.
\item The \textbf{standard $ n $-simplex} is the convex hull of the standard basis $ \cbr{e_1, \dots, e_{n + 1}} $ in $ \RR^{n + 1} $,
$$ \cbr{\br{x_0, \dots, x_n} \in \RR^{n + 1} \st x_i \ge 0, \ x_0 + \dots + x_n = 1}. $$
\item An \textbf{ordered $ n $-simplex} is an $ n $-simplex with an ordering on the vertices. We denote it by $ \sbr{v_0, \dots, v_n} $, where $ v_0, \dots, v_n $ are the vertices in ascending order.
\item The \textbf{standard ordered $ n $-simplex} is the ordered $ n $-simplex
$$ \sbr{e_1, \dots, e_{n + 1}} $$
in $ \RR^{n + 1} $. It is denoted by $ \Delta^n $.
\item Let $ \sbr{v_0, \dots, v_{n + 1}} $ be an $ n $-simplex in $ \RR^m $ and let $ L \subseteq \RR^m $ be the affine subspace spanned by $ v_0, \dots, v_n $. Then there exists a unique affine morphism
$$ \function{L}{\RR^{n + 1}}{v_i}{e_{i + 1}}, \qquad i = 0, \dots, n. $$
This gives a homeomorphism from $ \sbr{v_0, \dots, v_n} $ to $ \Delta^n $ that preserves this ordering.
\item For $ n \ge 1 $, the \textbf{faces} of an ordered $ n $-simplex $ \sbr{v_0, \dots, v_n} $ are the ordered $ \br{n - 1} $-simplices
$$ \sbr{v_0, \dots, \widehat{v_i}, \dots, v_n}. $$
$ \widehat{v_i} $ means we omit the vertex $ v_i $.
\item The union of all the faces of a simplex $ \Delta $ is the \textbf{boundary} $ \partial\Delta $.
\item The \textbf{interior} of $ \Delta $ is $ \mathring{\Delta} = \Delta \setminus \partial\Delta $.
\end{itemize}
\end{definition*}

\begin{example*}
Let $ \Delta^2 = \sbr{e_1, e_2, e_3} $. Then $ \partial\Delta^2 = \sbr{e_1, e_2} \cup \sbr{e_1, e_3} \cup \sbr{e_2, e_3} $.
\end{example*}

\pagebreak

\begin{definition*}
Let $ X $ be a topological space. A \textbf{$ \Delta $-complex structure} on $ X $ is a collection of continuous maps
$$ \sigma_\alpha : \Delta^{\n\br{\alpha}} \to X, \qquad \alpha \in A, \qquad \n\br{\alpha} \in \NN, $$
such that
\begin{enumerate}
\item the restriction $ \eval{\sigma_\alpha}_{\mathring{\Delta}^{\n\br{\alpha}}} $ is injective for all $ \alpha \in A $ and for each $ x \in X $ there is a unique $ \alpha \in A $ such that $ x \in \sigma_\alpha\br{\mathring{\Delta}^{\n\br{\alpha}}} $,
\item the restriction of $ \sigma_\alpha $ to a face of $ \Delta^{\n\br{\alpha}} $ is equal to $ \sigma_\beta $ for some $ \beta \in A $ and $ \n\br{\beta} = \n\br{\alpha} - 1 $, and
\item $ U \subseteq X $ is open if and only if $ \sigma_\alpha^{-1}\br{U} $ is open in $ \Delta^{\n\br{\alpha}} $ for all $ \alpha \in A $.
\end{enumerate}
\end{definition*}

An observation is that $ \sigma : \bigsqcup_{\alpha \in A} \Delta^{\n\br{\alpha}} \to X $ induced by the $ \sigma_\alpha $ is a quotient map, since it is surjective by $ 1 $ and $ U \subseteq X $ is open if and only if $ \sigma^{-1}\br{U} $ is open by $ 3 $.

\begin{remark*}
One can show that an $ X $ with a $ \Delta $-complex structure is a CW-complex.
\end{remark*}

\begin{example*}
\hfill
\begin{itemize}
\item Torus or Klein bottle is two $ \Delta^2 $, three $ \Delta^1 $, and one $ \Delta^0 $.
\item $ \S^2 $ is a tetrahedron.
\item \textbf{Dunce hat}, by identifying all the three faces of the standard $ 2 $-simplex with each other, is one $ \Delta^2 $, one $ \Delta^1 $, and one $ \Delta^0 $.
\end{itemize}
\end{example*}

\subsection{Simplicial homology}

\subsubsection{Simplicial homology}

Let $ X $ be a $ \Delta $-complex. The group of \textbf{$ n $-chains} $ \Delta_n\br{X} $ is the free abelian group on the $ n $-simplices $ \sigma_\alpha : \Delta^{\n\br{\alpha}} \to X $, where $ \n\br{\alpha} = n $. So an element in $ \Delta_n\br{X} $ is of the form
$$ \sum_{\alpha \in A, \ \n\br{\alpha} = n} c_\alpha \cdot \sigma_\alpha, \qquad c_\alpha \in \ZZ, $$
where all but finitely many of the $ c_\alpha $ are zero.

\begin{example*}
Let $ K $ be a Klein bottle.
\begin{itemize}
\item $ \Delta_0\br{K} = \cbr{n \cdot v \st n \in \ZZ} = \ZZ \cdot v \cong \ZZ $.
\item $ \Delta_1\br{K} = \cbr{n_1 \cdot a + n_2 \cdot b + n_3 \cdot c \st n_1, n_2, n_3 \in \ZZ} = \ZZ \cdot a \oplus \ZZ \cdot b \oplus \ZZ \cdot c \cong \ZZ^3 $.
\item $ \Delta_2\br{K} = \cbr{n_1 \cdot U + n_2 \cdot V \st n_1, n_2 \in \ZZ} = \ZZ \cdot U \oplus \ZZ \cdot V \cong \ZZ^2 $.
\item $ \Delta_n\br{K} = 0 $ for $ n \ge 3 $.
\end{itemize}
Similarly for a torus $ T $.
\end{example*}

\lecture{19}{Friday}{22/02/19}

Define the \textbf{boundary homomorphism} by
$$ \function[\partial_n]{\Delta_n\br{X}}{\Delta_{n - 1}\br{X}}{\sigma_\alpha}{\sum_{i = 0}^n \br{-1}^i\eval{\sigma_\alpha}_{\sbr{v_0, \dots, \widehat{v_i}, \dots, v_n}}}. $$
Moreover, we define $ \partial_0 = 0 $.

\begin{example*}
Let $ \sigma : \sbr{v_0, v_1, v_2, v_3} \to X $. Then
$$ \partial_3\br{\sigma} = \eval{\sigma}_{\sbr{v_1, v_2, v_3}} - \eval{\sigma}_{\sbr{v_0, v_2, v_3}} + \eval{\sigma}_{\sbr{v_0, v_1, v_3}} - \eval{\sigma}_{\sbr{v_0, v_1, v_2}}. $$
\end{example*}

\begin{lemma}
\label{lem:2.1}
The composition
$$ \Delta_n\br{X} \xrightarrow{\partial_n} \Delta_{n - 1}\br{X} \xrightarrow{\partial_{n - 1}} \Delta_{n - 2}\br{X} $$
is the zero map.
\end{lemma}

\pagebreak

\begin{proof}
Let $ \sigma : \sbr{v_0, \dots, v_n} \to X $ be an $ n $-simplex. Then
$$ \partial_n\br{\sigma} = \sum_{i = 0}^n \br{-1}^i\eval{\sigma}_{\sbr{v_0, \dots, \widehat{v_i}, \dots, v_n}}, $$
so
$$ \br{\partial_{n - 1} \circ \partial_n}\br{\sigma} = \sum_{j < i} \br{-1}^i\br{-1}^j\eval{\sigma}_{\sbr{v_0, \dots, \widehat{v_j}, \dots, \widehat{v_i}, \dots, v_n}} + \sum_{j > i} \br{-1}^i\br{-1}^{j - 1}\eval{\sigma}_{\sbr{v_0, \dots, \widehat{v_i}, \dots, \widehat{v_j}, \dots, v_n}} = 0. $$
If $ n = 1 $, clear.
\end{proof}

\subsubsection{Algebraic situation}

A \textbf{chain complex} of abelian groups is a diagram $ \br{C_\bullet, \partial} $ of the form
$$ \dots \xrightarrow{\partial_{n + 1}} C_n \xrightarrow{\partial_n} \dots \xrightarrow{\partial_1} C_0 \xrightarrow{\partial_0} 0, $$
where the $ C_i $ are abelian groups and the $ \partial_n $ are group homomorphisms such that $ \partial_n \circ \partial_{n - 1} = 0 $ for all $ n $. Then $ \partial_n $ are \textbf{boundary homomorphisms}. The elements in $ C_n $ are \textbf{$ n $-chains}. Let
$$ \Z_n = \Ker \partial_n \subseteq C_n, \qquad \B_n = \Im \partial_{n + 1} \subseteq C_n. $$
The elements in $ \Z_n $ are \textbf{cycles} and the elements in $ \B_n $ are \textbf{boundaries}. Since $ \partial_{n + 1} \circ \partial_n = 0 $, we have that $ \B_n \subseteq \Z_n $. The \textbf{$ n $-th homology group} of this chain complex is defined by
$$ \H_n\br{C_\bullet, \partial} = \Z_n / \B_n. $$
So, by Lemma \ref{lem:2.1}
$$ \dots \xrightarrow{\partial_{n + 1}} \Delta_n\br{X} \xrightarrow{\partial_n} \dots \xrightarrow{\partial_1} \Delta_0\br{X} \xrightarrow{\partial_0} 0 $$
is a chain complex. The \textbf{$ n $-th simplicial homology group} is
$$ \H_n^\Delta\br{X} = \Ker \partial_n / \Im \partial_{n + 1}. $$

\begin{example*}
Let $ X = \S^1 $. Then
$$
\begin{tikzcd}[row sep=tiny]
\dots \arrow{r}{\partial_3} & \Delta_2\br{X} \arrow{r}{\partial_2} \arrow[cong]{d} & \Delta_1\br{X} \arrow{r}{\partial_1} \arrow[cong]{d} & \Delta_0\br{X} \arrow{r}{\partial_0} \arrow[cong]{d} & 0 \\
& 0 & \ZZ & \ZZ &
\end{tikzcd}.
$$
\begin{itemize}
\item $ \Ker \partial_0 = \ZZ $ and $ \Im \partial_1 = 0 $, so $ \H_0^\Delta\br{X} \cong \ZZ $.
\item $ \Ker \partial_1 = \Delta_1\br{X} $ and $ \Im \partial_2 = 0 $, so $ \H_1^\Delta\br{X} \cong \ZZ $.
\item $ \H_n^\Delta\br{X} = 0 $ if $ n \ge 2 $.
\end{itemize}
\end{example*}

\begin{example*}
Let $ T $ be a torus. Then
$$
\begin{tikzcd}[row sep=tiny]
\dots \arrow{r}{\partial_4} & \Delta_3\br{T} \arrow{r}{\partial_3} \arrow[cong]{d} & \Delta_2\br{T} \arrow{r}{\partial_2} \arrow[cong]{d} & \Delta_1\br{T} \arrow{r}{\partial_1} \arrow[cong]{d} & \Delta_0\br{T} \arrow{r}{\partial_0} \arrow[cong]{d} & 0 \\
& 0 & \ZZ \cdot U \oplus \ZZ \cdot V & \ZZ \cdot a \oplus \ZZ \cdot b \oplus \ZZ \cdot c & \ZZ \cdot v &
\end{tikzcd}.
$$
\begin{itemize}
\item $ \Ker \partial_0 = \ZZ $ and $ \Im \partial_1 = 0 $, so $ \H_0^\Delta\br{T} \cong \ZZ $.
\item $ \partial_2\br{U} = a + b - c $ and $ \partial_2\br{V} = a + b - c $, and $ \cbr{a, b, a + b - c} $ is a basis for $ \Delta_1\br{T} $.
$$ \Ker \partial_1 = \Delta_1\br{T}, \qquad \Im \partial_2 = \ZZ \cdot \br{a + b - c}, $$
so $ \H_1^\Delta\br{T} \cong \ZZ \oplus \ZZ $.
\item $ \H_2^\Delta\br{T} \cong \ZZ $. \footnote{Exercise}
\end{itemize}
\end{example*}

\lecture{20}{Tuesday}{26/02/19}

Lecture 20 is a problems class.

\pagebreak

\subsection{Singular homology}

\subsubsection{Singular homology}

\lecture{21}{Wednesday}{27/02/19}

A \textbf{singular $ n $-simplex} in a topological space $ X $ is a continuous map $ \sigma : \Delta^n \to X $. Let $ \C_n\br{X} $ be the free abelian group on the set of all singular simplices in $ X $, that is the elements in $ \C_n\br{X} $ are finite formal sums
$$ \sum_i n_i \sigma_i, \qquad n_i \in \ZZ, $$
where $ \sigma_i : \Delta^n \to X $ are singular $ n $-simplices. The elements in $ \C_n\br{X} $ are called \textbf{singular $ n $-chains}. Define a \textbf{boundary map}
$$ \function[\partial_n]{\C_n\br{X}}{\C_{n - 1}\br{X}}{\sigma}{\sum_{i = 0}^n \br{-1}^i\eval{\sigma}_{\sbr{v_1, \dots, \widetilde{v_i}, \dots, v_n}}}, $$
for a singular $ n $-simplex $ \sigma $. Extend it linearly to $ \C_n\br{X} $.

\begin{lemma}
$ \partial_n \circ \partial_{n + 1} = 0 $.
\end{lemma}

\begin{proof}
The same proof as for Lemma \ref{lem:2.1}.
\end{proof}

We obtain a chain complex
$$ \dots \xrightarrow{\partial_{n + 1}} \C_n\br{X} \xrightarrow{\partial_n} \dots \xrightarrow{\partial_1} \C_0\br{X} \xrightarrow{\partial_0} 0. $$

\begin{remark*}
Often we write $ \partial $ instead of $ \partial_n $.
\end{remark*}

We define the \textbf{$ n $-th singular homology group} by
$$ \H_n\br{X} = \Ker \partial_n / \Im \partial_{n + 1}. $$
An observation is that if $ X $ and $ Y $ are homeomorphic then $ \H_n\br{X} \cong \H_n\br{Y} $.

\begin{proposition}
\label{prop:2.6}
Let $ X $ be a topological space and $ X = \bigcup_\alpha X_\alpha $ be the decomposition into its path-components. Then
$$ \H_n\br{X} \cong \bigoplus_\alpha \H_n\br{X_\alpha}. $$
\end{proposition}

\begin{proof}
A singular $ n $-simplex $ \sigma : \Delta^n \to X $ has a path-connected image. So
$$ \C_n\br{X} = \bigoplus_\alpha \C_n\br{X_\alpha}. $$
The boundary maps $ \partial_n $ preserve this decomposition, so $ \partial_n\br{\C_n\br{X_\alpha}} \subseteq \C_{n - 1}\br{X_\alpha} $ implies that $ \Ker \partial_n $ and $ \Im \partial_{n + 1} $ split as well as direct sums, so
$$ \H_n\br{X} = \Ker \partial_n / \Im \partial_{n + 1} \cong \bigoplus_\alpha \H_n\br{X_\alpha}. $$
\end{proof}

\begin{proposition}
\label{prop:2.7}
If $ X $ is a path-connected, and as always $ X \ne \emptyset $, topological space, then
$$ \H_0\br{X} \cong \ZZ. $$
Hence for $ X $ arbitrary $ \H_0\br{X} $ is a direct sum of $ \ZZ $'s, one for each path-component.
\end{proposition}

\begin{proof}
$ \partial_0 = 0 $, so $ \H_0\br{X} = \C_0\br{X} / \Im \partial_1 $. Define
$$ \function[\epsilon]{\C_0\br{X}}{\ZZ}{\sum_i n_i\sigma_i}{\sum_i n_i}. $$

\pagebreak

Then $ \epsilon $ is surjective. Enough to show that $ \Ker \epsilon = \Im \partial_1 $. This implies by the isomorphism theorem $ \H_0\br{X} \cong \ZZ $. Let $ \sigma : \Delta^1 \to X $ be a $ 1 $-simplex. Then
$$ \partial_1\br{\sigma} = \eval{\sigma}_{\sbr{v_1}} - \eval{\sigma}_{\sbr{v_0}}, $$
so $ \epsilon\br{\partial_1\br{\sigma}} = 0 $, so $ \Im \partial_1 \subseteq \Ker \epsilon $. On the other hand, $ \epsilon\br{\sum_i n_i\sigma_i} = 0 $ implies that $ \sum_i n_i = 0 $. The $ \sigma_i $ correspond to points $ \sigma_i\br{\sbr{v}} $ in $ X $. Choose a basepoint $ x_0 \in X $ and let
$$ \function[\sigma_0]{\Delta^0}{X}{\Delta^0}{x_0} $$
be the singular $ 0 $-simplex. Let $ \tau_i $ be a path from $ x_0 $ to $ \sigma_i\br{\sbr{v}} $. Consider $ \tau_i $ as a singular $ 1 $-simplex $ \tau_i : \sbr{v_0, v_1} \to X $. We have $ \partial_1 \circ \tau_i = \sigma_i - \sigma_0 $, so
$$ \partial_1\br{\sum_i n_i\tau_i} = \sum_i n_i\br{\sigma_i - \sigma_0} = \sum_i n_i\sigma_i - \sum_i n_i\sigma_0 = \sum_i n_i\sigma_i. $$
Thus $ \Ker \epsilon \subseteq \Im \partial_1 $.
\end{proof}

\begin{proposition}
\label{prop:2.8}
If $ X $ is a point, then
$$ \H_n\br{X} =
\begin{cases}
\ZZ & n = 0 \\
0 & n > 0
\end{cases}.
$$
\end{proposition}

\begin{proof}
For each $ n $ there exists a unique singular $ n $-simplex $ \partial_n : \Delta^n \to X $, so $ \C_n\br{X} \cong \ZZ $ for all $ n $. Then
$$ \partial_n\br{\sigma_n} = \sum_{i = 0}^n \br{-1}^i\sigma_{n - 1} =
\begin{cases}
0 & n \ \text{odd} \\
\sigma_{n - 1} & n \ \text{even}
\end{cases},
$$
so $ \partial_n = 0 $ if $ n $ is odd and $ \partial_n $ is an isomorphism if $ n $ is even, and
$$
\begin{tikzcd}[row sep=tiny]
\dots \arrow{r}{\partial_2} & \C_1\br{X} \arrow{r}{\partial_1} \arrow[cong]{d} & \C_0\br{X} \arrow{r}{\partial_0} \arrow[cong]{d} & 0 \\
\dots \arrow{r}{0} & \ZZ \arrow{r}{\sim} & \ZZ \arrow{r}{0} & 0
\end{tikzcd},
$$
so $ \H_n = \Ker \partial_n / \Im \partial_{n + 1} = 0 $ if $ n \ge 1 $ and $ \H_0\br{X} \cong \ZZ $.
\end{proof}

\subsubsection{Reduced homology groups}

The \textbf{reduced homology groups} $ \widetilde{\H_n}\br{X} $ are the homology groups of the \textbf{augmented chain complex}
$$ \dots \xrightarrow{\partial_2} \C_1\br{X} \xrightarrow{\partial_1} \C_0\br{X} \xrightarrow{\partial_0} \ZZ \xrightarrow{\epsilon} 0, $$
where $ \epsilon $ is as in proof of Proposition \ref{prop:2.7}. Then
$$ \H_n\br{X} \cong \widetilde{\H_n}\br{X}, \qquad n \ge 1. $$
Seen in the proof of Proposition \ref{prop:2.7} that $ \epsilon $ is surjective and $ \epsilon \circ \partial_1 = 0 $, so $ \Im \partial_1 \subseteq \Ker \epsilon $, so $ \epsilon $ induces a surjective homomorphism
$$ \phi_\epsilon : \H_0\br{X} = \C_0\br{X} / \Im \partial_1 \to \ZZ. $$
Then $ \Ker \phi_\epsilon = \Ker \epsilon / \Im \partial_1 = \widetilde{\H_0}\br{X} $, so $ \H_0\br{X} / \widetilde{\H_0}\br{X} \cong \ZZ $, so
$$ \H_0\br{X} \cong \widetilde{\H_0}\br{X} \oplus \ZZ. $$

\pagebreak

\subsection{Homotopy invariance}

\lecture{22}{Friday}{01/03/19}

Let $ \br{A_\bullet, \partial} $ and $ \br{B_\bullet, \partial} $ be two chain complexes. A \textbf{chain map} $ f : \br{A_\bullet, \partial} \to \br{B_\bullet, \partial} $ is a collection of homomorphisms $ f_n : A_n \to B_n $ such that $ \partial \circ f_n = f_{n + 1} \circ \partial $, that is the following diagram commutes.
$$
\begin{tikzcd}
\dots \arrow{r}{\partial} & A_{n + 1} \arrow{r}{\partial} \arrow{d}{f_{n + 1}} & A_n \arrow{r}{\partial} \arrow{d}{f_n} & A_{n - 1} \arrow{r}{\partial} \arrow{d}{f_{n - 1}} & \dots \\
\dots \arrow{r}{\partial} & B_{n + 1} \arrow{r}{\partial} & B_n \arrow{r}{\partial} & B_{n - 1} \arrow{r}{\partial} & \dots
\end{tikzcd}.
$$

If $ X $ and $ Y $ are topological spaces and $ f : X \to Y $ is a continuous map define the homomorphisms
$$ \function[f_\#]{\C_n\br{X}}{\C_n\br{Y}}{\sigma : \Delta^n \to X}{f \circ \sigma : \Delta^n \to Y}, $$
and extend it linearly to $ \C_n\br{X} $. Then
$$ \br{f_\# \circ \partial}\br{\sigma} = f_\#\br{\sum_{i = 0}^n \br{-1}^i\eval{\sigma}_{\sbr{v_0, \dots, \widehat{v_i}, \dots, v_n}}} = \sum_{i = 0}^n \eval{\br{f \circ \sigma}}_{\sbr{v_0, \dots, \widehat{v_i}, \dots, v_n}} = \br{\partial \circ f_\#}\br{\sigma}, $$
so $ f_\# \circ \partial = \partial \circ f_\# $, so $ f_\# $ defines a chain map
$$
\begin{tikzcd}
\dots \arrow{r}{\partial} & \C_{n + 1}\br{X} \arrow{r}{\partial} \arrow{d}{f_\#} & \C_n\br{X} \arrow{r}{\partial} \arrow{d}{f_\#} & \C_{n - 1}\br{X} \arrow{r}{\partial} \arrow{d}{f_\#} & \dots \\
\dots \arrow{r}{\partial} & \C_{n + 1}\br{Y} \arrow{r}{\partial} & \C_n\br{Y} \arrow{r}{\partial} & \C_{n - 1}\br{Y} \arrow{r}{\partial} & \dots
\end{tikzcd}.
$$
$ f_\# $ maps cycles to cycles, since $ \alpha \in \C_n\br{X} $ such that $ \partial \circ \alpha = 0 $, so
$$ \br{\partial \circ f_\#}\br{\alpha} = \br{f_\# \circ \partial}\br{\alpha} = 0. $$
$ f_\# $ maps boundaries to boundaries, since
$$ f_\# \circ \br{\partial \circ \beta} = \partial \circ \br{f_\# \circ \beta}. $$
$ f_\#\br{\Ker \partial_n} \subseteq \Ker \partial_n $ and $ f_\#\br{\Im \partial_{n + 1}} \subseteq \Im \partial_{n + 1} $, so $ f_\# $ induces a homomorphism
$$ f_* : \H_n\br{X} \to \H_n\br{Y}. $$
The following are observations.
\begin{itemize}
\item $ X \xrightarrow{g} Y \xrightarrow{f} Z $, so $ \br{f \circ g}_\# = f_\# \circ g_\# $, since
$$ \Delta^n \xrightarrow{\sigma} X \xrightarrow{g} Y \xrightarrow{f} Z, $$
so $ f \circ \br{g \circ \sigma} = \br{f \circ g} \circ \sigma $, so $ \br{f \circ g}_* = f_* \circ g_* $.
\item $ \br{\id_X}_* = \id_{\H_n\br{X}} $.
\end{itemize}

\begin{theorem}
\label{thm:2.10}
If two continuous maps $ f, g : X \to Y $ are homotopic, then
$$ f_* = g_* : \H_n\br{X} \to \H_n\br{Y}. $$
\end{theorem}

\begin{corollary}
If $ f : X \to Y $ is a homotopy equivalence, then
$$ f_* : \H_n\br{X} \to \H_n\br{Y} $$
is an isomorphism.
\end{corollary}

\begin{proof}
Let $ g : Y \to X $ be a continuous map such that $ f \circ g \cong \id_Y $ and $ g \circ f = \id_X $. Then $ f_* \circ g_* = \br{f \circ g}_* = \br{\id_Y}_* = \id $. Similarly $ g_* \circ f_* = \id $, so $ f_* $ is an isomorphism.
\end{proof}

\begin{example*}
$$ \H_n\br{\RR^k} =
\begin{cases}
\ZZ & n = 0 \\
0 & \text{otherwise}
\end{cases},
\qquad \widetilde{\H_n}\br{\RR^k} = 0.
$$
\end{example*}

\pagebreak

\begin{proof}[Proof of Theorem \ref{thm:2.10}]
Let $ F : X \times \I \to Y $ be a homotopy from $ f $ to $ g $ and $ \sigma : \Delta_n \to X $ be a singular $ n $-simplex. Consider the map
$$ \Delta^n \times \I \xrightarrow{\sigma \times \I} X \times \I \xrightarrow{F} Y. $$
Then $ \Delta^n \times \I $ is not a simplex. But we can subdivide $ \Delta^n \times \I $ into $ \br{n + 1} $ simplices. In general, we can decompose $ \Delta^n \times \I $ into $ n + 1 $ $ \br{n + 1} $-simplices
$$ \sbr{v_0, \dots, v_i, w_i, \dots, w_n}, \qquad i = 0, \dots, n. $$
Define \textbf{prism-operators}
$$ \function[P]{\C_n\br{X}}{\C_{n + 1}\br{Y}}{\sigma}{\sum_{i = 0}^n \br{-1}^iF \circ \eval{\br{\sigma \times \id}}_{\sbr{v_0, \dots, v_i, w_i, \dots, w_n}}}, $$
for $ \sigma : \Delta^n \to X $ a singular $ n $-simplex, so
$$
\begin{tikzcd}
\dots \arrow{r}{\partial} & \C_{n + 1}\br{X} \arrow{r}{\partial} & \C_n\br{X} \arrow{r}{\partial} \arrow[swap]{dl}{P} \arrow{d}{f_\#}[swap]{g_\#} & \C_{n - 1}\br{X} \arrow{r}{\partial} \arrow{dl}{P} & \dots \\
\dots \arrow{r}{\partial} & \C_{n + 1}\br{Y} \arrow{r}{\partial} & \C_n\br{Y} \arrow{r}{\partial} & \C_{n - 1}\br{Y} \arrow{r}{\partial} & \dots
\end{tikzcd}.
$$
Claim that
$$ \partial \circ P = g_\# - f_\# - P \circ \partial, $$
if and only if $ g_\# - f_\# = \partial \circ P + P \circ \partial $. The claim implies the theorem, since if $ \alpha \in \C_n\br{X} $ is a cycle, then
$$ g_\#\br{\alpha} - f_\#\br{\alpha} = \br{\partial \circ P}\br{\alpha} + \br{P \circ \partial}\br{\alpha} = \br{\partial \circ P}\br{\alpha}, $$
so $ g_\#\br{\alpha} - f_\#\br{\alpha} $ is a boundary. Thus $ g_\#\br{\alpha} $ and $ f_\#\br{\alpha} $ are in the same homology class, so $ g_*\br{\sbr{\alpha}} = f_*\br{\sbr{\alpha}} $, where $ \sbr{\alpha} $ is the homology class of $ \alpha $. Let $ \sigma : \Delta^n \to X $ be a singular $ n $-simplex. Then
\begin{align*}
\br{\partial \circ P}\br{\sigma}
= \ & \partial\br{\sum_{i = 0}^n \br{-1}^iF \circ \eval{\br{\sigma \times \id}}_{\sbr{v_0, \dots, v_i, w_i, \dots, w_n}}} \\
= \ & \sum_{j \le i} \br{-1}^i\br{-1}^jF \circ \eval{\br{\sigma \times \id}}_{\sbr{v_0, \dots, \widehat{v_j}, \dots, v_i, w_i, \dots, w_n}} \\
+ & \sum_{j \ge i} \br{-1}^i\br{-1}^{j + 1}F \circ \eval{\br{\sigma \times \id}}_{\sbr{v_0, \dots, v_i, w_i, \dots, \widehat{w_j}, \dots, w_n}}.
\end{align*}
If $ i = j $ the two sums cancel except for
$$ F \circ \eval{\br{\sigma \times \id}}_{\sbr{\widehat{v_0}, w_0, \dots, w_n}} = g \circ \sigma = g_\#\br{\sigma}, \qquad -F \circ \eval{\br{\sigma \times \id}}_{\sbr{v_0, \dots, v_n, \widehat{w_n}}} = -f \circ \sigma = -f_\#\br{\sigma}. $$
The terms with $ i \ne j $ sum up to $ \br{P \circ \partial}\br{\sigma} $, since we have
\begin{align*}
\br{P \circ \partial}\br{\sigma}
= \ & \sum_{j < i} \br{-1}^i\br{-1}^jF \circ \eval{\br{\sigma \times \id}}_{\sbr{v_0, \dots, \widehat{v_j}, \dots, v_i, w_i, \dots, w_n}} \\
+ & \sum_{j > i} \br{-1}^i\br{-1}^{j + 1}F \circ \eval{\br{\sigma \times \id}}_{\sbr{v_0, \dots, v_i, w_i, \dots, \widehat{w_j}, \dots, w_n}}.
\end{align*}
\end{proof}

\lecture{23}{Tuesday}{05/03/19}

\begin{remark*}
One can show that there are also induced homomorphisms
$$ f_* : \widetilde{\H_n}\br{X} \to \widetilde{\H_n}\br{Y} $$
invariant under homotopy. \footnote{Exercise}
\end{remark*}

\pagebreak

\subsection{Exact sequences and excision}

\subsubsection{Exact sequences}

Let $ A \subseteq X $ be a subspace. What is the relationship between $ \H_n\br{A}, \H_n\br{X}, \H_n\br{X / A} $?

\begin{definition*}
A sequence of group homomorphisms of abelian groups
$$ \dots \xrightarrow{\alpha_{n + 1}} A_n \xrightarrow{\alpha_n} \dots $$
is \textbf{exact} at $ A_n $ if $ \Ker \alpha_n = \Im \alpha_{n + 1} $. The sequence is \textbf{exact} if it is exact at $ A_n $ for all $ n $.
\end{definition*}

An observation is if the sequence is exact, then
\begin{itemize}
\item $ \alpha_n\alpha_{n + 1} = 0 $, so exact sequences are chain complexes, and
\item the homology groups of this chain complex are all trivial.
\end{itemize}

\begin{example*}
\hfill
\begin{itemize}
\item $ 0 \to A \xrightarrow{\alpha} B $ is exact if and only if $ \Ker \alpha = 0 $, if and only if $ \alpha $ is injective.
\item $ A \xrightarrow{\alpha} B \to 0 $ is exact if and only if $ \Im \alpha = B $, if and only if $ \alpha $ is surjective.
\item $ 0 \to A \xrightarrow{\alpha} B \to 0 $ is exact if and only if $ \alpha $ is an isomorphism.
\item $ 0 \to A \xrightarrow{\alpha} B \xrightarrow{\beta} C \to 0 $ is exact if and only if $ \alpha $ is injective, $ \beta $ is surjective, and $ \Ker \beta = \Im \alpha $, hence $ \beta $ induces an isomorphism
$$ C \cong B / \Im \alpha = B / A. $$
This is called a \textbf{short exact sequence}.
\end{itemize}
\end{example*}

\begin{definition*}
Let $ X $ be a topological space and $ A \subseteq X $. Then $ A $ is a \textbf{strong deformation retract} of $ X $ if there exists a retraction $ r : X \to A $ such that $ r $ is homotopic to the identity, and $ F : \I \times X \to X $ continuous such that
$$ F\br{0, x} = x, \qquad F\br{1, x} = r\br{x}, \qquad F\br{t, a} = a, \qquad x \in X, \qquad a \in A, \qquad t \in \I. $$
Let $ X $ be a topological space and $ A \subseteq X $ a non-empty closed subspace. Then $ \br{X, A} $ is called a \textbf{good pair} if $ A $ has a neighbourhood in $ X $ that strongly deformation retracts to $ A $.
\end{definition*}

\begin{example*}
\hfill
\begin{itemize}
\item $ \br{\D^n, \S^{n - 1}} $ is a good pair, since $ \S^{n - 1} $ is a deformation retract of $ \D^n \setminus \cbr{0} $.
\item Let $ A = \cbr{1 / n \st n \in \NN} \cup \cbr{0} \subseteq \sbr{0, 1} $ then $ \br{\sbr{0, 1}, A} $ is not a good pair.
\end{itemize}
\end{example*}

\begin{theorem}
\label{thm:2.13}
Let $ \br{X, A} $ be a good pair, then there is an exact sequence
$$ \dots \to \widetilde{\H_1}\br{A} \xrightarrow{i_*} \widetilde{\H_1}\br{X} \xrightarrow{j_*} \widetilde{\H_1}\br{X / A} \xrightarrow{\partial} \widetilde{\H_0}\br{A} \xrightarrow{i_*} \widetilde{\H_0}\br{X} \xrightarrow{j_*} \widetilde{\H_0}\br{X / A} \to 0, $$
where $ i : A \hookrightarrow X $ is the inclusion and $ j : X \to X / A $ is the quotient.
\end{theorem}

\begin{corollary}
$$ \widetilde{\H_i}\br{\S^n} =
\begin{cases}
\ZZ & i = n \\
0 & i \ne n
\end{cases}.
$$
\end{corollary}

\pagebreak

\begin{proof}
$ \br{\D^n, \S^{n - 1}} $ is a good pair. Let $ n > 0 $. Recall that $ \D^n / \S^{n - 1} \cong \S^n $, so
$$
\begin{tikzcd}[column sep=small, row sep=tiny]
\dots \arrow{r} & \widetilde{\H_i}\br{\S^{n - 1}} \arrow{r}{i_*} & \widetilde{\H_i}\br{\D^n} \arrow{r}{j_*} \arrow[cong]{d} & \widetilde{\H_i}\br{\S^n} \arrow{r}{\partial} & \widetilde{\H_{i - 1}}\br{\S^{n - 1}} \arrow{r}{i_*} & \widetilde{\H_{i - 1}}\br{\D^n} \arrow{r}{j_*} \arrow[cong]{d} & \widetilde{\H_{i - 1}}\br{\S^n} \arrow{r} & \dots \\
& & 0 & & & 0 & &
\end{tikzcd}.
$$
Then $ \widetilde{\H_i}\br{\S^n} \cong \widetilde{\H_{i - 1}}\br{\S^{n - 1}} $ for $ i > 0 $, so
$$
\begin{tikzcd}[column sep=small, row sep=tiny]
\dots \arrow{r} & \widetilde{\H_1}\br{\S^{n - 1}} \arrow{r}{i_*} & \widetilde{\H_1}\br{\D^n} \arrow{r}{j_*} \arrow[cong]{d} & \widetilde{\H_1}\br{\S^n} \arrow{r}{\partial} & \widetilde{\H_0}\br{\S^{n - 1}} \arrow{r}{i_*} & \widetilde{\H_0}\br{\D^n} \arrow{r}{j_*} \arrow[cong]{d} & \widetilde{\H_0}\br{\S^n} \arrow{r} & 0 \\
& & 0 & & & 0 & &
\end{tikzcd}.
$$
$ n > 0 $ and $ i > 0 $, so $ \widetilde{\H_i}\br{\S^n} \cong \widetilde{\H_{i - 1}}\br{\S^{n - 1}} $, and $ \widetilde{\H_0}\br{\S^n} = 0 $. We know that $ \widetilde{\H_0}\br{\S^0} \cong \ZZ $ and $ \widetilde{\H_n}\br{\S^0} = 0 $, by Proposition \ref{prop:2.6} and Proposition \ref{prop:2.8}. Doing induction on $ n $,
$$ \widetilde{\H_i}\br{\S^n} =
\begin{cases}
\ZZ & i = n \\
0 & i \ne n
\end{cases}.
$$
\end{proof}

\begin{corollary}
\label{cor:2.15}
There exists no retraction $ r : \D^n \to \partial \D^n $.
\end{corollary}

\begin{proof}
Assume there exists such an $ r : \D^n \to \partial \D^n $. Let $ i : \partial \D^n \to \D^n $. Then $ ri = \id_{\partial \D^n} $, so $ r_*i_* = \br{ri}_* = \id $, so
$$
\begin{tikzcd}[row sep=tiny]
\widetilde{\H_{n - 1}}\br{\partial \D^n} \arrow{r}{i_*} \arrow[cong]{d} & \widetilde{\H_{n - 1}}\br{\D^n} \arrow{r}{r_*} \arrow[cong]{d} & \widetilde{\H_{n - 1}}\br{\partial \D^n} \arrow[cong]{d} \\
\ZZ & 0 & \ZZ
\end{tikzcd}.
$$
Thus $ i_* = 0 $ and $ r_* = 0 $, a contradiction.
\end{proof}

\begin{theorem}[Brouwer fixed point theorem]
Every continuous map $ f : \D^n \to \D^n $ has a fixed point.
\end{theorem}

\begin{proof}
Assume there exists a fixed point then construct as in dimension two a retraction $ \D^n \to \partial \D^n $, a contradiction to Corollary \ref{cor:2.15}.
\end{proof}

\subsubsection{Relative homology groups}

Let $ X $ be a topological space and $ A \subseteq X $ be a subspace. Define
$$ \C_n\br{X, A} = \C_n\br{X} / \C_n\br{A}. $$
Let $ \partial : \C_n\br{X} \to \C_{n - 1}\br{X} $ be the boundary map then $ \partial\br{\sigma : \Delta^n \to A} \in \partial\br{\C_n\br{A}} \subseteq \C_{n - 1}\br{A} $. So $ \partial $ induces a homomorphism
$$ \partial : \C_n\br{X, A} \to \C_{n - 1}\br{X, A}, $$
such that $ \partial \circ \partial = 0 $. This gives a chain complex
$$ \dots \to \C_{n + 1}\br{X, A} \xrightarrow{\partial} \C_n\br{X, A} \xrightarrow{\partial} \C_{n - 1}\br{X, A} \to \dots. $$
\begin{itemize}
\item The homology groups $ \H_n\br{X, A} $ of this complex are the \textbf{relative homology groups}.
\item The \textbf{relative $ n $-chains} are $ \C_n\br{X, A} $.
\item The \textbf{relative $ n $-cycles} are $ \Ker \partial \subseteq \C_n\br{X, A} $, of the form $ \sbr{\alpha} $ for $ \alpha \in \C_n\br{X} $ such that $ \partial\br{\alpha} \in \C_{n - 1}\br{A} $.
\item The \textbf{relative $ n $-boundaries} are $ \Im \partial \subseteq \C_n\br{X, A} $, of the form $ \sbr{\alpha} $ for $ \alpha \in \C_n\br{X} $ such that $ \alpha = \partial\beta + \gamma $ for $ \beta \in \C_{n + 1}\br{X} $ and $ \gamma \in \C_n\br{A} $.
\end{itemize}

\pagebreak

\lecture{24}{Wednesday}{06/03/19}

A \textbf{short exact sequence of chain complexes} is
$$ 0 \to \br{A_\bullet, \partial} \xrightarrow{i} \br{B_\bullet, \partial} \xrightarrow{j} \br{C_\bullet, \partial} \to 0, $$
for $ i $ and $ j $ chain maps, where
$$ 0 \to A_n \xrightarrow{i} B_n \xrightarrow{j} C_n \to 0 $$
is a short exact sequence for all $ n $, so
$$
\begin{tikzcd}[row sep=small]
& 0 \arrow{d} & 0 \arrow{d} & 0 \arrow{d} & \\
\dots \arrow{r}{\partial} & A_{n + 1} \arrow{r}{\partial} \arrow{d}{i} & A_n \arrow{r}{\partial} \arrow{d}{i} & A_{n - 1} \arrow{r}{\partial} \arrow{d}{i} & \dots \\
\dots \arrow{r}{\partial} & B_{n + 1} \arrow{r}{\partial} \arrow{d}{j} & B_n \arrow{r}{\partial} \arrow{d}{j} & B_{n - 1} \arrow{r}{\partial} \arrow{d}{j} & \dots \\
\dots \arrow{r}{\partial} & C_{n + 1} \arrow{r}{\partial} \arrow{d} & C_n \arrow{r}{\partial} \arrow{d} & C_{n - 1} \arrow{r}{\partial} \arrow{d} & \dots \\
& 0 & 0 & 0 &
\end{tikzcd}.
$$
A short exact sequence of chain complexes always yields a \textbf{long exact sequence} of homology groups
$$ \dots \to \H_n\br{A} \xrightarrow{i_*} \H_n\br{B} \xrightarrow{j_*} \H_n\br{C} \xrightarrow{\partial} \H_{n - 1}\br{A} \xrightarrow{i_*} \H_{n - 1}\br{B} \xrightarrow{j_*} \H_{n - 1}\br{C} \to \dots. $$
This is the \textbf{zig-zag lemma}. First we construct the \textbf{connecting map} $ \partial : \H_n\br{C} \to \H_{n - 1}\br{A} $. Let $ c \in C_n $ be a cycle.
\begin{itemize}
\item $ j $ is surjective, so $ c = j\br{b} $ for some $ b \in B_n $.
\item $ j\br{\partial\br{b}} = \partial\br{j\br{b}} = \partial c = 0 $, so $ \partial b \in \Ker j \subseteq B_{n - 1} $, so $ \partial\br{b} = i\br{a} $ for some $ a \in A_{n - 1} $, by exactness.
\item $ \partial\br{a} = 0 $, since $ i\br{\partial\br{a}} = \partial\br{i\br{a}} = \partial\br{\partial\br{b}} = 0 $ and $ i $ is injective, so $ \partial\br{a} = 0 $.
\end{itemize}
$$
\begin{tikzcd}
& a \in A_{n - 1} \arrow{d}{i} \\
b \in B_n \arrow{r}{\partial} \arrow{d}{j} & \in \partial\br{b} \in B_{n - 1} \\
c \in C_n &
\end{tikzcd}.
$$
Define
$$ \function[\partial]{\H_n\br{C}}{\H_{n - 1}\br{A}}{\sbr{c}}{\sbr{a}}. $$
This is well-defined.
\begin{itemize}
\item $ a $ is uniquely determined by $ \partial\br{b} $ because $ i $ is injective.
\item If we choose $ b' $ instead of $ b $, then $ j\br{b'} = j\br{b} $, so $ j\br{b' - b} = j\br{b'} - j\br{b} = 0 $, so $ b' - b \in \Ker j = \Im i $, hence $ b' - b = i\br{a'} $ for some $ a' \in A_n $, so $ b' = b + i\br{a'} $. If we replace $ b $ by $ b' = b + i\br{a'} $ this corresponds to replacing $ a $ by $ a + \partial\br{a'} $, because
$$ i\br{a + \partial\br{a'}} = i\br{a} + i\br{\partial\br{a'}} = \partial\br{b} + \partial\br{i\br{a'}} = \partial\br{b + i\br{a'}}, $$
and $ \sbr{a} = \sbr{a + \partial\br{a'}} $.
\item A different choice of $ c $ in its homology class has the form $ c + \partial\br{c'} $ for some $ c' \in C_{n + 1} $. Let $ b' \in B_{n + 1} $ such that $ j\br{b'} = c' $. Then
$$ c + \partial\br{c'} = c + \partial\br{j\br{b'}} = j\br{b} + j\br{\partial\br{b'}} = j\br{b + \partial\br{b'}}, $$
so $ b $ is replaced by $ b + \partial\br{b'} $ but $ \partial\br{b} = \partial\br{b + \partial b'} $, so $ \partial\br{b} $ is unchanged and hence $ a $ is unchanged.
\end{itemize}

\pagebreak

The map $ \partial : \H_n\br{C} \to \H_{n - 1}\br{A} $ is a homomorphism, since if $ \partial\br{\sbr{c_1}} = \sbr{a_1} $ and $ \partial\br{\sbr{c_2}} = \sbr{a_2} $ via elements $ b_1 $ and $ b_2 $ in $ B_n $, then
$$ j\br{b_1 + b_2} = j\br{b_1} + j\br{b_2} = c_1 + c_2, \qquad i\br{a_1 + a_2} = i\br{a_1} + i\br{a_2} = \partial\br{b_1} + \partial\br{b_2} = \partial\br{b_1 + b_2}, $$
so $ \partial\br{\sbr{c_1} + \sbr{c_2}} = \sbr{a_1} + \sbr{a_2} $.

\begin{theorem}
The sequence
$$ \dots \to \H_n\br{A} \xrightarrow{i_*} \H_n\br{B} \xrightarrow{j_*} \H_n\br{C} \xrightarrow{\partial} \H_{n - 1}\br{A} \xrightarrow{i_*} \H_{n - 1}\br{B} \xrightarrow{j_*} \H_{n - 1}\br{C} \to \dots $$
is exact.
\end{theorem}

\begin{proof}
Diagram chase, see Hatcher.
\end{proof}

Let $ i $ be the inclusion and $ j $ be the quotient.
$$
\begin{tikzcd}[row sep=small]
& 0 \arrow{d} & 0 \arrow{d} & \\
\dots \arrow{r}{\partial} & \C_n\br{A} \arrow{r}{\partial} \arrow{d}{i} & \C_{n - 1}\br{A} \arrow{r}{\partial} \arrow{d}{i} & \dots \\
\dots \arrow{r}{\partial} & \C_n\br{X} \arrow{r}{\partial} \arrow{d}{j} & \C_{n - 1}\br{X} \arrow{r}{\partial} \arrow{d}{j} & \dots \\
\dots \arrow{r}{\partial} & \C_n\br{X, A} \arrow{r}{\partial} \arrow{d} & \C_{n - 1}\br{X, A} \arrow{r}{\partial} \arrow{d} & \dots \\
& 0 & 0 &
\end{tikzcd}.
$$
This diagram commutes, so this is a short exact sequence of chain complexes. Zig-zag gives a long exact sequence of homology groups
$$ \dots \to \H_1\br{A} \xrightarrow{i_*} \H_1\br{X} \xrightarrow{j_*} \H_1\br{X, A} \xrightarrow{\partial} \H_0\br{A} \xrightarrow{i_*} \H_0\br{X} \xrightarrow{j_*} \H_0\br{X, A} \to 0. $$
What is $ \partial : \H_n\br{X, A} \to \H_{n - 1}\br{A} $? If $ \sbr{a} \in \H_n\br{X, A} $ is represented by a cycle $ \alpha \in \C_n\br{X} $, then $ \partial\br{\sbr{\alpha}} $ is the class of the cycle $ \partial\br{\alpha} $, so $ \partial\br{\sbr{\alpha}} = \sbr{\partial\br{\alpha}} $. We also obtain a short exact sequence of the augmented chain complex
$$
\begin{tikzcd}[row sep=small]
& 0 \arrow{d} & 0 \arrow{d} & 0 \arrow{d} & \\
\dots \arrow{r} & \C_1\br{A} \arrow{r} \arrow{d} & \C_0\br{A} \arrow{r} \arrow{d} & \ZZ \arrow{r} \arrow{d} & 0 \\
\dots \arrow{r} & \C_1\br{X} \arrow{r} \arrow{d} & \C_0\br{X} \arrow{r} \arrow{d} & \ZZ \arrow{r} \arrow{d} & 0 \\
\dots \arrow{r} & \C_1\br{X, A} \arrow{r} \arrow{d} & \C_0\br{X, A} \arrow{r} \arrow{d} & 0 \arrow{r} \arrow{d} & 0 \\
& 0 & 0 & 0 &
\end{tikzcd},
$$
so if $ A \ne \emptyset $, then $ \widetilde{\H_n}\br{X, A} = \H_n\br{X, A} $ for all $ n $. We also have a long exact sequence
$$ \dots \to \widetilde{\H_n}\br{A} \to \widetilde{\H_n}\br{X} \to \widetilde{\H_n}\br{X, A} \to \widetilde{\H_{n - 1}}\br{A} \to \widetilde{\H_{n - 1}}\br{X} \to \widetilde{\H_{n - 1}}\br{X, A} \to \dots. $$
An observation is if $ x_0 \in X $ then $ \H_n\br{X, x_0} \cong \widetilde{\H_n}\br{X} $ for all $ n $. Another observation is that a continuous map $ f : X \to Y $ such that $ f\br{A} \subseteq B $ induces a chain map
$$ f_\# : \C_n\br{X, A} \to \C_n\br{Y, B}, $$
since $ f_\# : \C_n\br{X} \to \C_n\br{Y} $ maps $ \C_n\br{A} $ to $ \C_n\br{B} $ so it is well-defined on the quotient, and hence homomorphisms
$$ f_* : \H_n\br{X, A} \to \H_n\br{Y, B}. $$
This is functorial, so $ \br{f \circ g}_* = f_* \circ g_* $.

\pagebreak

\lecture{25}{Friday}{08/03/19}

\begin{definition*}
A \textbf{homotopy} between two maps
$$ f, g : \br{X, A} \to \br{Y, B} $$
is a continuous map $ F : \I \times X \to Y $ such that
$$ F\br{0, x} = f\br{x}, \qquad F\br{1, x} = g\br{x}, \qquad F\br{s, a} \in B, \qquad x \in X, \qquad s \in \I, \qquad a \in A. $$
\end{definition*}

\begin{proposition}
If
$$ f, g : \br{X, A} \to \br{Y, B} $$
are homotopic, then
$$ f_* = g_* : \H_n\br{X, A} \to \H_n\br{Y, B}. $$
\end{proposition}

\begin{proof}
Analogous to proof of Theorem \ref{thm:2.10}. Prism operator $ P : \C_n\br{X} \to \C_{n + 1}\br{Y} $ maps $ \C_n\br{A} $ to $ \C_n\br{B} $ so it induces a map
$$ P' : \C_n\br{X} / \C_n\br{A} \to \C_{n + 1}\br{Y} / \C_{n + 1}\br{B}, $$
and $ \partial P' + P'\partial = g_\# - f_\# $, so $ f_* = g_* $.
\end{proof}

Let $ \br{X, A, B} $ be a triple for $ X $ a topological space and $ B \subset A \subset X $, so
$$ \br{A, B} \to \br{X, B} \to \br{X, A}. $$
There is a short exact sequence of chain complexes
$$
\begin{tikzcd}[row sep=tiny]
0 \arrow{r} & \C_n\br{A, B} \arrow{r} \arrow[cong]{d} & \C_n\br{X, B} \arrow{r} \arrow[cong]{d} & \C_n\br{X, A} \arrow{r} \arrow[cong]{d} & 0 \\
& \C_n\br{A} / \C_n\br{B} & \C_n\br{X} / \C_n\br{B} & \C_n\br{X} / \C_n\br{A} &
\end{tikzcd},
$$
so there is a long exact sequence
$$ \dots \to \H_n\br{A, B} \to \H_n\br{X, B} \to \H_n\br{X, A} \to \H_{n - 1}\br{A, B} \to \H_{n - 1}\br{X, B} \to \H_{n - 1}\br{X, A} \to \dots. $$

\subsubsection{Excision}

\begin{theorem}[Excision]
Let $ X $ be a topological space and $ Z \subset A \subset X $ be subspaces such that the closure $ \overline{Z} $ of $ Z $ is contained in the interior $ \mathring{A} $ of $ A $. Then the inclusion
$$ \br{X \setminus Z, A \setminus Z} \hookrightarrow \br{X, A} $$
induces isomorphisms
$$ \H_n\br{X \setminus Z, A \setminus Z} \xrightarrow{\sim} \H_n\br{X, A}, $$
for all $ n $. Equivalently, let $ A, B \subseteq X $ such that $ \mathring{A} \cup \mathring{B} = X $. Then the inclusion
$$ \br{B, A \cap B} \hookrightarrow \br{X, A} $$
induces isomorphisms
$$ \H_n\br{B, A \cap B} \xrightarrow{\sim} \H_n\br{X, A}, $$
for all $ n $.
\end{theorem}

Why equivalent? Set $ B = X \setminus Z $ and $ Z = X \setminus B $. Then $ A \cap B = A \setminus Z $ and $ \overline{Z} = X \setminus \mathring{B} $. Then $ \overline{Z} \subseteq \mathring{A} $ if and only if $ X = \mathring{A} \cup \mathring{B} $.

\begin{proof}
Hatcher page $ 119 $ to $ 124 $.
\end{proof}

\pagebreak

\begin{proposition}
\label{prop:2.22}
Let $ \br{X, A} $ be a good pair. Then the quotient map
$$ q : \br{X, A} \to \br{X / A, A / A} $$
induces isomorphisms
$$ q_* : \H_n\br{X, A} \xrightarrow{\sim} \H_n\br{X / A, A / A} \cong \widetilde{\H_n}\br{X / A}, $$
for all $ n $.
\end{proposition}

\begin{proof}
Let $ V \subseteq X $ be a neighbourhood of $ A $ that strongly deformation retracts to $ A $. Then $ \br{V, A} $ is homotopy equivalent to $ \br{A, A} $, so
$$ \H_n\br{V, A} \cong \H_n\br{A, A} = 0. $$
The triple $ \br{X, V, A} $ where $ A \subset V \subset X $ induces a long exact sequence
$$
\begin{tikzcd}[row sep=tiny]
\dots \arrow{r} & \H_n\br{V, A} \arrow{r} \arrow[cong]{d} & \H_n\br{X, A} \arrow{r} & \H_n\br{X, V} \arrow{r} & \H_{n - 1}\br{V, A} \arrow{r} \arrow[cong]{d} & \dots \\
& 0 & & & 0 &
\end{tikzcd},
$$
so
$$ \H_n\br{X, A} \cong \H_n\br{X, V}. $$
The same with the triple $ \br{X / A, V / A, A / A} $, so again
$$ \H_n\br{V / A, A / A} \cong \H_n\br{A / A, A / A} = 0. $$
This gives a long exact sequence
$$ \H_n\br{X / A, A / A} \cong \H_n\br{X / A, V / A}. $$
Consider the diagram
$$
\begin{tikzcd}
\H_n\br{X, A} \arrow{r}{\sim} \arrow{d}{q_*} & \H_n\br{X, V} \arrow{d}{q_*} & \H_n\br{X \setminus A, V \setminus A} \arrow{l}{\alpha}[swap]{\sim} \arrow{d}{j}[swap]{\sim} \\
\H_n\br{X / A, A / A} \arrow{r}{\sim} & \H_n\br{X / A, V / A} & \H_n\br{X / A \setminus A / A, V / A \setminus A / A} \arrow{l}{\beta}[swap]{\sim}
\end{tikzcd}.
$$
\begin{itemize}
\item This diagram commutes.
\item $ q : X \to X / A $ induces a homeomorphism $ X \setminus A \to X / A \setminus A / A $, so $ j $ is an isomorphism.
\item $ \alpha $ and $ \beta $ are isomorphisms by the excision theorem.
\end{itemize}
Thus
$$ q_* : \H_n\br{X, A} \to \H_n\br{X / A, A / A} $$
is an isomorphism.
\end{proof}

\begin{proof}[Proof of Theorem \ref{thm:2.13}]
Long exact sequence of pair $ \br{X, A} $ with reduced homology
$$ \dots \to \widetilde{\H_n}\br{A} \to \widetilde{\H_n}\br{X} \to \widetilde{\H_n}\br{X, A} \to \widetilde{\H_{n - 1}}\br{A} \to \widetilde{\H_{n - 1}}\br{X} \to \widetilde{\H_{n - 1}}\br{X, A} \to \dots, $$
so
$$ \widetilde{\H_n}\br{X, A} = \H_n\br{X, A} \cong \widetilde{\H_n}\br{X / A}, $$
by last time.
\end{proof}

\lecture{26}{Tuesday}{12/03/19}

\begin{corollary}
Let $ \cbr{X_\alpha} $ for $ \alpha \in A $ be a collection of topological spaces and $ x_\alpha \in X_\alpha $ such that $ \br{X_\alpha, x_\alpha} $ is a good pair, for all $ \alpha \in A $. Let $ \bigvee_\alpha X_\alpha $ be the wedge sum with respect to the points $ x_\alpha $. Then there is an isomorphism
$$ \widetilde{\H_n}\br{\bigsqcup_\alpha X_\alpha} \cong \bigoplus_\alpha \widetilde{\H_n}\br{X_\alpha} \xrightarrow{\sim} \widetilde{\H_n}\br{\bigvee_\alpha X_\alpha}. $$
\end{corollary}

\pagebreak

\begin{proof}
$ \br{X, A} = \br{\bigsqcup_\alpha X_\alpha, \bigsqcup_\alpha \cbr{x_\alpha}} $ is a good pair, so Proposition \ref{prop:2.22} implies that
$$ \H_n\br{X, A} \cong \H_n\br{\bigvee_\alpha X_\alpha, \bigsqcup_\alpha \cbr{x_\alpha} / \bigsqcup_\alpha \cbr{x_\alpha}} \cong \widetilde{\H_n}\br{\bigvee_\alpha X_\alpha}, $$
and
$$ \H_n\br{X, A} \cong \bigoplus_\alpha \H_n\br{X_\alpha, x_\alpha} \cong \bigoplus_\alpha \widetilde{\H_n}\br{X_\alpha}. $$
\end{proof}

\begin{example*}
$$ \widetilde{\H_n}\br{\S^1 \vee \S^1} \cong \widetilde{\H_n}\br{\S^1} \oplus \widetilde{\H_n}\br{\S^1} \cong
\begin{cases}
0 & n = 0 \\
\ZZ \oplus \ZZ & n = 1 \\
0 & n \ge 2
\end{cases}.
$$
$$ \widetilde{\H_n}\br{\S^1 \vee \S^1 \vee \S^2} \cong \widetilde{\H_n}\br{\S^1} \oplus \widetilde{\H_n}\br{\S^1} \oplus \widetilde{\H_n}\br{\S^2} \cong
\begin{cases}
0 & n = 0 \\
\ZZ \oplus \ZZ & n = 1 \\
\ZZ & n = 2 \\
0 & n \ge 3
\end{cases}.
$$
\end{example*}

Recall that
$$ \H_n^\Delta\br{\S^1 \times \S^1} =
\begin{cases}
\ZZ & n = 0 \\
\ZZ \oplus \ZZ & n = 1 \\
\ZZ & n = 2 \\
0 & n \ge 3
\end{cases}.
$$
We will see that singular and simplicial homology coincide in Appendix A.2, so $ \S^1 \vee \S^1 \vee \S^2 $ and $ \S^1 \times \S^1 $ have isomorphic homology groups, but they are not homotopy equivalent.

\begin{theorem}[Invariance of dimension]
Let $ U \subseteq \RR^m $ and $ V \subseteq \RR^n $ be open, non-empty. If $ U $ and $ V $ are homeomorphic, then $ m = n $.
\end{theorem}

\begin{proof}
For $ x \in U $ set $ A = \RR^m \setminus \cbr{x} $ and $ B = U $. Excision implies that
$$ \H_k\br{U, U \setminus \cbr{x}} \cong \H_k\br{\RR^m, \RR^m \setminus \cbr{x}}. $$
Long exact sequence of a pair implies that
$$
\begin{tikzcd}[row sep=tiny]
\dots \arrow{r} & \widetilde{\H_k}\br{\RR^m} \arrow{r} \arrow[cong]{d} & \widetilde{\H_k}\br{\RR^m, \RR^m \setminus \cbr{x}} \arrow{r} & \widetilde{\H_{k - 1}}\br{\RR^m \setminus \cbr{x}} \arrow{r} & \widetilde{\H_{k - 1}}\br{\RR^m} \arrow{r} \arrow[cong]{d} & \dots \\
& 0 & & & 0 &
\end{tikzcd},
$$
so $ \H_k\br{\RR^m, \RR^m \setminus \cbr{x}} \cong \widetilde{\H_{k - 1}}\br{\RR^m \setminus \cbr{x}} $. Then $ \RR^m \setminus \cbr{x} $ deformation retracts to $ \S^{m - 1} $, so
$$ \H_k\br{U, U \setminus \cbr{x}} =
\begin{cases}
\ZZ & k = m \\
0 & \text{otherwise}
\end{cases}.
$$
Similarly
$$ \H_k\br{V, V \setminus \cbr{x}} =
\begin{cases}
\ZZ & k = n \\
0 & \text{otherwise}
\end{cases}.
$$
Let $ h : U \to V $ be a homeomorphism then this induces isomorphisms
$$ h_* : \H_k\br{U, U \setminus \cbr{x}} \to \H_k\br{V, V \setminus \cbr{h\br{x}}}, $$
for all $ k $, so $ m = n $.
\end{proof}

\pagebreak

\subsubsection{Naturality}

\begin{proposition}[Naturality of connecting homomorphisms]
Let
$$ \br{A_\bullet, \partial}, \br{B_\bullet, \partial}, \br{C_\bullet, \partial}, \br{A_\bullet', \partial}, \br{B_\bullet', \partial}, \br{C_\bullet', \partial} $$
be chain complexes. Consider a commutative diagram of chain maps
$$
\begin{tikzcd}
0 \arrow{r} & A_\bullet \arrow{r}{i} \arrow{d}{\alpha} & B_\bullet \arrow{r}{j} \arrow{d}{\beta} & C_\bullet \arrow{r} \arrow{d}{\gamma} & 0 \\
0 \arrow{r} & A_\bullet' \arrow{r}{i'} & B_\bullet' \arrow{r}{j'} & C_\bullet' \arrow{r} & 0
\end{tikzcd},
$$
where the rows are short exact sequences. Then the induced diagram
$$
\begin{tikzcd}[column sep=small]
\dots \arrow{r} & \H_n\br{A} \arrow{r}{i_*} \arrow{d}{\alpha_*} & \H_n\br{B} \arrow{r}{j_*} \arrow{d}{\beta_*} & \H_n\br{C} \arrow{r}{\partial} \arrow{d}{\gamma_*} & \H_{n - 1}\br{A} \arrow{r}{i_*} \arrow{d}{\alpha_*} & \H_{n - 1}\br{B} \arrow{r}{j_*} \arrow{d}{\beta_*} & \H_{n - 1}\br{C} \arrow{r} \arrow{d}{\gamma_*} & \dots \\
\dots \arrow{r} & \H_n\br{A'} \arrow{r}{i_*'} & \H_n\br{B'} \arrow{r}{j_*'} & \H_n\br{C'} \arrow{r}{\partial} & \H_{n - 1}\br{A'} \arrow{r}{i_*'} & \H_{n - 1}\br{B'} \arrow{r}{j_*'} & \H_{n - 1}\br{C'} \arrow{r} & \dots
\end{tikzcd}
$$
is commutative.
\end{proposition}

\begin{proof}
The first two squares commute by functoriality.
$$ \function[\partial]{\H_n\br{C}}{\H_{n - 1}\br{A}}{\sbr{c}}{\sbr{a}}, $$
so
$$
\begin{tikzcd}
& a \in A_{n - 1} \arrow{d}{i} \\
b \in B_n \arrow{r}{\partial} \arrow{d}{j} & \in \partial\br{b} \in B_{n - 1} \\
c \in C_n &
\end{tikzcd}.
$$
Then $ \gamma\br{c} = \gamma\br{j\br{b}} = j'\br{\beta\br{b}} $ and $ i'\br{\alpha\br{a}} = \beta\br{i\br{a}} = \beta\br{\partial\br{b}} = \partial\br{\beta\br{b}} $, so
$$
\begin{tikzcd}
& \alpha\br{a} \in A_{n - 1}' \arrow{d}{i'} \\
\beta\br{b} \in B_n' \arrow{r}{\partial} \arrow{d}{j'} & \in \partial\br{\beta\br{b}} \in B_{n - 1}' \\
\gamma\br{c} \in C_n' &
\end{tikzcd},
$$
so $ \partial\sbr{\gamma\br{c}} = \sbr{\alpha\br{a}} $ and hence $ \partial\br{\gamma_*\sbr{c}} = \alpha_*\sbr{a} = \alpha_*\br{\partial\sbr{c}} $.
\end{proof}

\subsection{Mayer-Vietoris sequences}

\subsubsection{The Mayer-Vietoris sequence}

The main ingredient of the proof of the excision theorem is \textbf{barycentric subdivision}. Let $ X $ be a topological space and $ \UUU = \cbr{U_i} $ be a collection of subspaces whose interiors form an open cover of $ X $. Define $ \C_n^\UUU \subseteq \C_n\br{X} $ as the subgroup of all chains of the form $ \sum_i n_i\sigma_i $ such that the image of $ \sigma_i $ is contained in some $ U_j \in \UUU $. Then $ \partial : \C_n\br{X} \to \C_{n - 1}\br{X} $ satisfies $ \partial\br{\C_n^\UUU\br{X}} \subseteq \C_{n - 1}^\UUU\br{X} $ so the $ \C_n^\UUU\br{X} $ define a chain complex. Let $ \H_n^\UUU\br{X} $ be the homology groups with respect to this chain complex.

\begin{proposition}
The inclusion $ i : \C_n^\UUU\br{X} \hookrightarrow \C_n\br{X} $ induces isomorphisms $ \H_n^\UUU\br{X} \cong \H_n\br{X} $ for all $ n $.
\end{proposition}

\begin{proof}
Hatcher page $ 119 $.
\end{proof}

\begin{notation*}
If $ \UUU = \cbr{A, B} $ we write $ \C_n\br{A + B} $ instead of $ \C_n^\UUU\br{X} $.
\end{notation*}

\pagebreak

\begin{theorem}[Mayer-Vietoris sequence]
Let $ X $ be a topological space, $ A, B \subseteq X $ such that $ \mathring{A} \cup \mathring{B} = X $, and
$$ i_1 : A \cap B \hookrightarrow A, \qquad i_2 : A \cap B \hookrightarrow B, \qquad j_1 : A \hookrightarrow X, \qquad j_2 : B \hookrightarrow X $$
be inclusions. Then there is an exact sequence
$$ \dots \to \H_1\br{A \cap B} \xrightarrow{\Phi} \H_1\br{A} \oplus \H_1\br{B} \xrightarrow{\Psi} \H_1\br{X} \xrightarrow{\partial} \H_0\br{A \cap B} \xrightarrow{\Phi} \H_0\br{A} \oplus \H_0\br{B} \xrightarrow{\Psi} \H_0\br{X} \to 0, $$
where $ \Phi\br{x} = \br{i_{1*}\br{x}, -i_{2*}\br{x}} $, $ \Psi\br{x, y} = j_{1*}\br{x} + j_{2*}\br{y} $, and $ \partial $ is the connecting homomorphism.
\end{theorem}

\begin{proof}
Let a sequence of chain complexes be
$$ 0 \to \C_n\br{A \cap B} \xrightarrow{\phi} \C_n\br{A} \oplus \C_n\br{B} \xrightarrow{\psi} \C_n\br{A + B} \to 0, $$
where $ \phi\br{x} = \br{x, -x} $ and $ \psi\br{x, y} = x + y $.
\begin{itemize}
\item $ \phi $ is injective.
\item $ \Im \phi \subseteq \Ker \psi $.
\item If $ \br{x, y} \in \Ker \psi $, then $ y = -x $, and $ x \in \C_n\br{A} $ and $ y \in \C_n\br{B} $, so $ x \in \C_n\br{A \cap B} $, so $ \Ker \psi \subseteq \Im \phi $.
\item $ \psi $ is surjective by the definition of $ \C_n\br{A + B} $.
\end{itemize}
So this is a short exact sequence of chain complexes. This induces a long exact sequence of homology groups
$$
\begin{tikzcd}[column sep=tiny, row sep=tiny]
\dots \arrow{r} & \H_1\br{A \cap B} \arrow{r}{\Phi} & \H_1\br{A} \oplus \H_1\br{B} \arrow{r}{\Psi} & \H_1^{A + B}\br{X} \arrow{r}{\partial} \arrow[cong]{d} & \H_0\br{A \cap B} \arrow{r}{\Phi} & \H_0\br{A} \oplus \H_0\br{B} \arrow{r}{\Psi} & \H_0^{A + B}\br{X} \arrow{r} \arrow[cong]{d} & 0 \\
& & & \H_1\br{X} & & & \H_0\br{X} &
\end{tikzcd},
$$
by barycentric division.
\end{proof}

\lecture{27}{Wednesday}{13/03/19}

If $ A \cap B \ne \emptyset $ we can augment these chain complexes and obtain a short exact sequence between these augmented chain complexes
$$
\begin{tikzcd}
& \vdots \arrow{d} & \vdots \arrow{d} & \vdots \arrow{d} & \\
0 \arrow{r} & \C_0\br{A \cap B} \arrow{r}{\phi} \arrow{d}{\epsilon} & \C_0\br{A} \oplus \C_0\br{B} \arrow{r}{\psi} \arrow{d}{\epsilon} & \C_0\br{A + B} \arrow{r} \arrow{d}{\epsilon} & 0 \\
0 \arrow{r} & \ZZ \arrow{r} & \ZZ \oplus \ZZ \arrow{r} & \ZZ \arrow{r} & 0
\end{tikzcd}.
$$
This induces a long exact sequence of homology groups
$$ \dots \to \widetilde{\H_1}\br{A \cap B} \xrightarrow{\Phi} \widetilde{\H_1}\br{A} \oplus \widetilde{\H_1}\br{B} \xrightarrow{\Psi} \widetilde{\H_1}\br{X} \xrightarrow{\partial} \widetilde{\H_0}\br{A \cap B} \xrightarrow{\Phi} \widetilde{\H_0}\br{A} \oplus \widetilde{\H_0}\br{B} \xrightarrow{\Psi} \widetilde{\H_0}\br{X} \to 0. $$
This is the Mayer-Vietoris sequence for reduced homology groups.

\begin{note*}
This is the same as in the non-reduced case, but we need to assume that $ A \cap B \ne \emptyset $.
\end{note*}

An observation is that if $ A \cap B $ is path-connected, then $ \widetilde{\H_0}\br{A \cap B} = 0 $, so we have an exact sequence
$$
\begin{tikzcd}[row sep=tiny]
\dots \arrow{r} & \widetilde{\H_1}\br{A \cap B} \arrow{r}{\Phi} & \widetilde{\H_1}\br{A} \oplus \widetilde{\H_1}\br{B} \arrow{r}{\Psi} & \widetilde{\H_1}\br{X} \arrow{r}{\partial} & \widetilde{\H_0}\br{A \cap B} \arrow{r} \arrow[cong]{d} & \dots \\
& & & & 0 & & &
\end{tikzcd}.
$$
Thus
$$ \H_1\br{X} \cong \H_1\br{A} \oplus \H_1\br{B} / \Phi\br{\H_1\br{A \cap B}}. $$
This is the abelianised version of the theorem of Seifert-van Kampen.

\pagebreak

\begin{example*}
Let $ X = \S^n \subseteq \RR^{n + 1} $ and let $ x \in \S^n $. Define $ A = \S^n \setminus \cbr{x} $ and $ B = \S^n \setminus \cbr{-x} $. Then $ A $ and $ B $ are contractible, so $ \widetilde{\H_n}\br{A} = \widetilde{\H_n}\br{B} = 0 $ for all $ n $, and $ A \cap B $ deformation retracts to $ \S^{n - 1} $. Mayer-Vietoris implies that
$$
\begin{tikzcd}[row sep=tiny]
\dots \arrow{r} & \widetilde{\H_i}\br{A} \oplus \widetilde{\H_i}\br{B} \arrow{r} \arrow[cong]{d} & \widetilde{\H_i}\br{X} \arrow{r} & \widetilde{\H_{i - 1}}\br{A \cap B} \arrow{r} \arrow[cong]{d} & \widetilde{\H_{i - 1}}\br{A} \oplus \widetilde{\H_{i - 1}}\br{B} \arrow{r} \arrow[cong]{d} & \dots \\
& 0 & & \widetilde{\H_{i - 1}}\br{\S^{n - 1}} & 0 &
\end{tikzcd},
$$
so $ \widetilde{\H_i}\br{\S^n} \cong \widetilde{\H_{i - 1}}\br{\S^{n - 1}} $ for $ n \ge 1 $. We know $ \widetilde{\H_0}\br{\S^0} \cong \ZZ $ and $ \widetilde{\H_0}\br{\S^n} = 0 $ for $ n \ge 1 $, so induction and knowledge on $ \H_n\br{\S^0} $ implies that
$$ \widetilde{\H_k}\br{\S^n} =
\begin{cases}
\ZZ & k = n \\
0 & \text{otherwise}
\end{cases}.
$$
\end{example*}

\begin{example*}
Let $ U, V \subseteq \RR^n $ be two path-connected open subsets such that $ U \cup V = \RR^n $. Then $ U \cap V $ is path-connected as well. Enough to show that $ \H_0\br{U \cap V} \cong \ZZ $, if and only if $ \widetilde{\H_0}\br{U \cap V} = 0 $. Then $ U \cap V \ne \emptyset $ because $ \RR^n $ is connected, and $ U $ and $ V $ are open, so $ \mathring{U} = U $ and $ \mathring{V} = V $, so $ \mathring{U} \cup \mathring{V} = \RR^n $. Mayer-Vietoris long exact sequence for reduced homology groups implies that
$$
\begin{tikzcd}[row sep=tiny]
\dots \arrow{r} & \widetilde{\H_1}\br{\RR^n} \arrow{r} \arrow[cong]{d} & \widetilde{\H_0}\br{U \cap V} \arrow{r} & \widetilde{\H_0}\br{U} \oplus \widetilde{\H_0}\br{V} \arrow{r} \arrow[cong]{d} & \widetilde{\H_0}\br{\RR^n} \arrow{r} \arrow[cong]{d} & 0 \\
& 0 & & 0 & 0 &
\end{tikzcd},
$$
since $ \RR^n $ is contractible, so $ \widetilde{\H_k}\br{\RR^n} = 0 $ for all $ k $, and $ \widetilde{\H_0}\br{U} = \widetilde{\H_0}\br{V} = 0 $, because $ U $ and $ V $ are path-connected. Thus $ \widetilde{\H_0}\br{U \cap V} = 0 $.
\end{example*}

\subsubsection{Classical applications}

\begin{definition*}
Let $ X $ and $ Y $ be topological spaces. A continuous map $ \phi : X \to Y $ is an \textbf{embedding} if it is a homeomorphism to its image.
\end{definition*}

\begin{example*}
If $ X $ is compact and $ Y $ is Hausdorff, and $ \phi : X \to Y $ is a continuous and injective map, then $ \phi $ is an embedding, since $ \phi : X \to \phi\br{X} $ is continuous and bijective and $ \phi\br{X} $ is Hausdorff, so worksheet $ 1 $ implies that $ \phi $ is a homeomorphism $ X \to \phi\br{X} $.
\end{example*}

\begin{proposition}
\label{prop:2b.1}
\hfill
\begin{enumerate}
\item Let $ h : \D^k \to \S^n $ be an embedding, then $ \widetilde{\H_i}\br{\S^n \setminus h\br{\D^k}} = 0 $ for all $ i $.
\item Let $ h : \S^k \to \S^n $ be an embedding, with $ k < n $, then
$$ \widetilde{\H_i}\br{\S^n \setminus h\br{\S^k}} =
\begin{cases}
\ZZ & i = n - k - 1 \\
0 & \text{otherwise}
\end{cases}.
$$
\end{enumerate}
\end{proposition}

\begin{corollary}
Let $ h : \S^1 \to \S^2 $ be an embedding. Then $ \S^2 \setminus h\br{\S^1} $ consists of exactly two path-components.
\end{corollary}

\begin{proof}
$ \widetilde{\H_0}\br{\S^2 \setminus h\br{\S^1}} \cong \ZZ $ by Proposition \ref{prop:2b.1}.
\end{proof}

\begin{corollary}[Jordan curve theorem]
Let $ h : \S^1 \to \RR^2 $ be an embedding. Then $ \RR^2 \setminus h\br{\S^1} $ consists of exactly two path-components.
\end{corollary}

\begin{proof}
$ \RR^2 $ is homeomorphic to $ \S^2 \setminus \cbr{x} $, by stereographic projection.
\end{proof}

Similarly, $ \RR^n \setminus h\br{\S^{n - 1}} $ consists of exactly two path-components.

\pagebreak

\lecture{28}{Friday}{15/03/19}

\begin{proof}[Proof of Proposition \ref{prop:2b.1}]
\hfill
\begin{enumerate}
\item Induction on $ k $.
\begin{itemize}[leftmargin=2cm]
\item[$ k = 0 $.] $ \S^n \setminus h\br{\D^0} \cong \RR^n $, so $ \widetilde{\H_i}\br{\S^n \setminus h\br{\D^n}} = 0 $ for all $ n $.
\item[$ k - 1 \mapsto k $.] Let $ h : \D^k \to \S^n $ be an embedding. Replace $ \D^k $ by $ \I^k $. For a contradiction, assume there is a cycle $ \alpha $ in $ \S^n \setminus h\br{\I^k} $ that is not a boundary in $ \S^n \setminus h\br{\I^k} $. Claim that there is a nested sequence of intervals
$$ \sbr{0, 1} = I_0 \supseteq I_1 \supseteq \dots, $$
such that $ I_i $ is of length $ \tfrac{1}{2^i} $ and such that $ \alpha $ is a cycle in $ \S^n \setminus h\br{\I^{k - 1} \times I_i} $ but not a boundary in $ \S^n \setminus h\br{\I^{k - 1} \times I_i} $. Let $ A = \S^n \setminus h\br{\I^{k - 1} \times \sbr{0, \tfrac{1}{2}}} $ and $ B = \S^n \setminus h\br{\I^{k - 1} \times \sbr{\tfrac{1}{2}, 1}} $, so $ A \cap B = \S^n \setminus h\br{\I^k} $ and $ A \cup B = \S^n \setminus h\br{\I^{k - 1} \times \cbr{\tfrac{1}{2}}} $. Induction hypothesis implies that $ \widetilde{\H_j}\br{A \cup B} = 0 $ for all $ j $. Mayer-Vietoris implies that
$$
\begin{tikzcd}[column sep=small, row sep=tiny]
\dots \arrow{r} & \widetilde{\H_{j + 1}}\br{A \cup B} \arrow{r} \arrow[cong]{d} & \widetilde{\H_j}\br{A \cap B} \arrow{r}{\sim} & \widetilde{\H_j}\br{A} \oplus \widetilde{\H_j}\br{B} \arrow{r} & \widetilde{\H_j}\br{A \cup B} \arrow{r} \arrow[cong]{d} & \dots \\
& 0 & & & 0 &
\end{tikzcd},
$$
so
$$ \widetilde{\H_j}\br{\S^n \setminus h\br{\I^k}} \cong \widetilde{\H_j}\br{\S^n \setminus h\br{\I^{k - 1} \times \sbr{0, \tfrac{1}{2}}}} \oplus \widetilde{\H_j}\br{\S^n \setminus h\br{\I^{k - 1} \times \sbr{\tfrac{1}{2}, 1}}}. $$
Hence $ \alpha $ is a cycle but not a boundary in $ \S^n \setminus h\br{\I^{k - 1} \times \sbr{0, \tfrac{1}{2}}} $ or $ \S^n \setminus h\br{\I^{k - 1} \times \sbr{\tfrac{1}{2}, 1}} $. This gives us $ I_1 $. Iterating, this proves the claim. By induction, $ \alpha $ is a boundary of some cycle $ \beta $ in $ \S^n \setminus h\br{\I^{k - 1} \times \cbr{x}} $ for any $ x \in \I $, so in particular, for $ \cbr{x} = \bigcap_i I_i $. Then $ \beta = \sum_i n_i\sigma_i $ is a sum of finitely many singular simplices. The images of the $ \sigma_i $ are compact. But $ \S^n \setminus h\br{\I^{k - 1} \times I_i} $ form an open cover of $ \S^n \setminus h\br{\I^{k - 1} \times \cbr{x}} $. So, by compactness, $ \beta $ is a chain in $ \S^n \setminus h\br{\I^{k - 1} \times I_i} $ for some $ i $. Thus $ \alpha $ is a boundary in $ \S^n \setminus h\br{\I^{k - 1} \times I_i} $, a contradiction.
\end{itemize}
\item Induction on $ k $.
\begin{itemize}[leftmargin=2cm]
\item[$ k = 0 $.] $ \S^n \setminus h\br{\S^0} \cong \S^{n - 1} \times \RR $, so
$$ \widetilde{\H_i}\br{\S^n \setminus h\br{\S^0}} \cong
\begin{cases}
\ZZ & i = n - k - 1 \\
0 & \text{otherwise}
\end{cases}.
$$
\item[$ k - 1 \mapsto k $.] Let $ h : \S^k \to \S^n $ be an embedding and $ \S^k = \D_+^k \cup \D_-^k $. Let $ A = \S^n \setminus h\br{\D_+^k} $ and $ B = \S^n \setminus h\br{\D_-^k} $, so $ 1 $ implies that $ \widetilde{\H_i}\br{A} = 0 $ and $ \widetilde{\H_i}\br{B} = 0 $ for all $ i $, and $ A \cap B = \S^n \setminus h\br{\S^k} $ and $ A \cup B = \S^n \setminus h\br{\S^{k - 1}} $. Mayer-Vietoris implies that
$$
\begin{tikzcd}[column sep=tiny, row sep=tiny]
\dots \arrow{r} & \widetilde{\H_{i + 1}}\br{A} \oplus \widetilde{\H_{i + 1}}\br{B} \arrow{r} \arrow[cong]{d} & \widetilde{\H_i}\br{A \cup B} \arrow{r}{\sim} & \widetilde{\H_{i + 1}}\br{A \cap B} \arrow{r} & \widetilde{\H_i}\br{A} \oplus \widetilde{\H_i}\br{B} \arrow{r} \arrow[cong]{d} & \dots \\
& 0 & & & 0 &
\end{tikzcd},
$$
by $ 1 $, so
$$ \widetilde{\H_{i + 1}}\br{\S^n \setminus h\br{\S^{k - 1}}} \cong \widetilde{\H_i}\br{\S^n \setminus h\br{\S^k}} \cong
\begin{cases}
\ZZ & i + 1 = n - \br{k - 1} - 1 \\
0 & \text{otherwise}
\end{cases},
$$
by induction.
\end{itemize}
\end{enumerate}
\end{proof}

\lecture{29}{Tuesday}{19/03/19}

Lecture 29 is a problems class.

\pagebreak

\subsection{Degree}

\lecture{30}{Wednesday}{20/03/19}

Let $ n \ge 1 $. We have seen that $ \H_n\br{\S^n} \cong \abr{a} \cong \ZZ $. Let $ f : \S^n \to \S^n $ be a continuous map, so $ f_* : \H_n\br{\S^n} \to \H_n\br{\S^n} $ is a homomorphism. Then $ f_* $ is given by $ f_*\br{\alpha} = d\alpha $ for some $ d \in \ZZ $ depending only on $ f $. This integer is the \textbf{degree} of $ f $.

\begin{proposition}
The following are observations.
\begin{enumerate}
\item $ \deg \id_{\S^n} = 1 $.
\item If $ f $ is not surjective, then $ \deg f = 0 $.
\item If $ f \cong g $, then $ f_* = g_* $, so $ \deg f = \deg g $.
\item $ \deg fg = \deg f\deg g $. In particular, if $ f $ is a homotopy equivalence, then $ \deg f = \pm 1 $.
\item Let
$$ \function[R_i]{\S^n}{\S^n}{\br{x_1, \dots, x_i, \dots, x_{n + 1}}}{\br{x_1, \dots, -x_i, \dots, x_{n + 1}}} $$
be the reflection map. Then $ \deg R_i = -1 $.
\item The antipodal map
$$ \function[-\id_{\S^n}]{\S^n}{\S^n}{x}{-x} $$
has degree $ \br{-1}^{n + 1} $.
\item If $ f : \S^n \to \S^n $ has no fixed points, then $ \deg f = \br{-1}^{n + 1} $.
\end{enumerate}
\end{proposition}

Hopf implies that if $ \deg f = \deg g $ then $ f \cong g $.

\begin{proof}
$ 1 $ and $ 3 $ are clear.
\begin{itemize}
\item[$ 2 $.] Let $ x_0 \in \S^n \setminus f\br{\S^n} $. So $ f $ factors as $ f = i \circ f' $, where
$$ \S^n \xrightarrow{f'} \S^n \setminus \cbr{x_0} \xhookrightarrow{i} \S^n. $$
$ \H_n\br{\S^n \setminus \cbr{x_0}} = 0 $ since $ \S^n \setminus \cbr{x_0} $ is contractible, so $ f_* = i_* \circ f_*' = 0 $.
\item[$ 4 $.] $ \br{fg}_* = f_*g_* $, and there exists $ g : \S^n \to \S^n $ such that $ fg \cong \id_{\S^n} $, so
$$ \deg f\deg g = \deg fg = \deg \id_{\S^n} = 1. $$
\item[$ 5 $.] Enough to show it for $ i = 1 $. Induction on $ n $.
\begin{itemize}[leftmargin=2cm]
\item[$ n = 1 $.] $ R_1\br{x_1, x_2} = \br{-x_1, x_2} $. Then $ \omega : t \mapsto \br{\cos 2\pi t, \sin 2\pi t} $ implies that $ R_1\br{\sbr{\omega}} = -\sbr{\omega} $, so $ \deg R_1 = -1 $.
\item[$ n - 1 \mapsto n $.] Claim that there is an isomorphism $ \phi : \H_n\br{\S^n} \xrightarrow{\sim} \H_{n - 1}\br{\S^{n - 1}} $ such that
$$
\begin{tikzcd}
\H_n\br{\S^n} \arrow{r}{\phi} \arrow{d}{R_{1*}} & \H_{n - 1}\br{\S^{n - 1}} \arrow{d}{R_{1*}} \\
\H_n\br{\S^n} \arrow{r}{\phi} & \H_{n - 1}\br{\S^{n - 1}}
\end{tikzcd}
$$
commutes. Let
$$ N = \br{0, \dots, 0, 1}, \qquad S = \br{0, \dots, 0, -1}, \qquad U = \S^n \setminus \cbr{N}, \qquad V = \S^n \setminus \cbr{S}, $$
so $ R_1\br{U} = U $ and $ R_1\br{V} = V $. There is a commutative diagram of chain maps
$$
\begin{tikzcd}
0 \arrow{r} & \C_\bullet\br{U \cap V} \arrow{r} \arrow{d}{R_{1\#}} & \C_\bullet\br{U} \oplus \C_\bullet\br{V} \arrow{r} \arrow{d}{R_{1\#} \oplus R_{1\#}} & \C_\bullet\br{U + V} \arrow{r} \arrow{d}{R_{1\#}} & 0 \\
0 \arrow{r} & \C_\bullet\br{U \cap V} \arrow{r} & \C_\bullet\br{U} \oplus \C_\bullet\br{V} \arrow{r} & \C_\bullet\br{U + V} \arrow{r} & 0
\end{tikzcd}.
$$

\pagebreak

This induces a commutative diagram
$$
\begin{tikzcd}
\H_n\br{\S^n} \arrow{r}{\partial} \arrow{d}{R_{1*}} & \H_{n - 1}\br{U \cap V} \arrow{d}{R_{1*}} & \H_{n - 1}\br{\S^{n - 1}}\arrow[swap]{l}{i_*} \arrow{d}{R_{1*}} \\
\H_n\br{\S^n} \arrow{r}{\partial} & \H_{n - 1}\br{U \cap V} & \H_{n - 1}\br{\S^{n - 1}}\arrow[swap]{l}{i_*}
\end{tikzcd},
$$
where
$$ \function[i]{\S^{n - 1}}{U \cap V}{\br{x_1, \dots, x_n}}{\br{x_1, \dots, x_n, 0}} $$
is a homotopy equivalence. Then $ i_* $ is an isomorphism because $ i $ is a homotopy equivalence and $ \partial $ is an isomorphism as seen last week. The first square commutes by naturality and the second square commutes by functoriality.
\end{itemize}
\item[$ 6 $.] $ -\id_{\S^n} = R_1 \dots R_{n + 1} $, so
$$ \deg -\id_{\S^n} = \deg R_1 \dots \deg R_{n + 1} = \br{-1}^{n + 1}. $$
\item[$ 7 $.] If $ f\br{x} \ne x $ for all $ x \in \S^n $, then the line segment from $ f\br{x} $ to $ -x $ defined by
$$ t \mapsto \br{1 - t}f\br{x} - tx $$
does not pass through the origin. Define
$$ f_t\br{x} = \dfrac{\br{1 - t}f\br{x} - tx}{\abs{\br{1 - t}f\br{x} - tx}}, $$
so $ f_t $ is a homotopy from $ f $ to $ -\id_{\S^n} $. Thus
$$ \deg f = \deg -\id_{\S^n} = \br{-1}^{n + 1}. $$
\end{itemize}
\end{proof}

\begin{proposition}
If $ n $ is even, then $ \ZZ / 2\ZZ $ is the only non-trivial group that can act freely by homeomorphisms on $ \S^n $.
\end{proposition}

\begin{proof}
Let $ G $ be a group acting freely by homeomorphisms on $ \S^n $, so $ G \subseteq \Homeo \S^n $. So for $ f \in G $, $ \deg f = \pm 1 $ by $ 4 $, and $ \deg fg = \deg f\deg g $ for all $ f, g \in G $ by $ 3 $, so the degree defines a homeomorphism $ d : G \to \cbr{\pm 1} \cong \ZZ / 2\ZZ $. The action is free, so if $ g \in G \setminus \cbr{\id} $, then $ g $ has no fixed points, so $ 7 $ and $ n $ even implies that $ \deg g = \br{-1}^{n + 1} = -1 $. Then $ \Ker d = \cbr{\id} $, so $ d $ is injective, so $ G = \cbr{\id} $ or $ G \cong \ZZ / 2\ZZ $.
\end{proof}

\begin{definition*}
A \textbf{vector field} on $ \S^n $ is a continuous map $ v : \S^n \to \RR^{n + 1} $ such that for each $ x \in \S^n $, $ v\br{x} $ is \textbf{tangent} to $ \S^n $ at $ x $, that is $ v\br{x} $ and $ x $ are orthogonal.
\end{definition*}

\begin{theorem}[Hairy ball theorem]
$ \S^n $ admits a continuous vector field $ v : \S^n \to \RR^{n + 1} $ that is nowhere zero if and only if $ n $ is odd.
\end{theorem}

\begin{proof}
If $ v\br{x} \ne 0 $ for all $ x \in \S^n $, let
$$ \function[v']{\S^n}{\RR^{n - 1}}{x}{\dfrac{v\br{x}}{\abs{v\br{x}}}}. $$
Define
$$ f_t\br{x} = \cos \br{t\pi}x + \sin \br{t\pi}v'\br{x}. $$
Then $ f_t\br{x} \in \S^n $ for all $ x \in \S^n $ and for all $ t \in \I $, so $ f_t $ is a homotopy from $ \id_{\S^n} $ to $ -\id_{\S^n} $, so
$$ 1 = \deg \id_{\S^n} = \deg -\id_{\S^n} = \br{-1}^{n + 1}. $$
Thus $ n $ is odd. Conversely, if $ n = 2k - 1 $,
$$ v\br{x_1, \dots, x_{2k}} = \br{-x_2, x_1, \dots, -x_{2k}, x_{2k - 1}} $$
is a vector field on $ \S^n $.
\end{proof}

\pagebreak

\appendix

\section{Proofs}

\subsection{The Seifert-van Kampen theorem}

\begin{proof}[Proof of Theorem \ref{thm:seifertvankampen}]
Consider the natural homomorphism
$$ \Phi : \pi_1\br{U_1, x_0} * \pi_1\br{U_2, x_0} \to \pi_1\br{X, x_0}. $$
$ \Phi $ is surjective by Lemma \ref{lem:1.15}, and $ N \subseteq \Ker \Phi $. Want to show that $ N = \Ker \Phi $. A \textbf{factorisation} of an element $ \sbr{f} \in \pi_1\br{X, x_0} $ is a formal product $ \sbr{f_1} \dots \sbr{f_k} $ such that
\begin{itemize}
\item each $ f_i $ is a loop at $ x_0 $ in one of the $ U_i $ and $ \sbr{f_i} \in \pi_1\br{U_i, x_0} $ is its homotopy class, and
\item the loop $ f_1 \cdot \dots \cdot f_k $ is homotopic to $ f $ in $ X $.
\end{itemize}
A factorisation of $ \sbr{f} $ is a word in $ \pi_1\br{U_1, x_0} * \pi_1\br{U_2, x_0} $ that is mapped to $ \sbr{f} $ by $ \Phi $. Two factorisations of $ \sbr{f} $ are \textbf{equivalent} if they are related by finitely many of the following two moves.
\begin{itemize}
\item If $ \sbr{f_i} $ and $ \sbr{f_{i + 1}} $ lie in the same group $ \pi_1\br{U_i, x_0} $, exchange $ \sbr{f_i}\sbr{f_{i + 1}} $ with $ \sbr{f_i \cdot f_{i + 1}} $. These are the relations in $ \pi_1\br{U_i, x_0} * \pi_1\br{U_i, x_0} $.
\item If $ f_i $ is a loop in $ U_1 \cap U_2 $, consider $ \sbr{f_i} $ as an element in $ \pi_1\br{U_1, x_0} $ instead of $ \pi_1\br{U_2, x_0} $, and vice versa. These are the relations in $ \pi_1\br{U_1, x_0} * \pi_1\br{U_2, x_0} / N $.
\end{itemize}
Given $ \sbr{f} \in \pi_1\br{X, x_0} $, we want to show that any two factorisations of $ \sbr{f} $ are equivalent. Let $ \sbr{f_1} \dots \sbr{f_k} $ and $ \sbr{f_1'} \dots \sbr{f_l'} $ be two factorisations of $ \sbr{f} $, so the two loops $ f_1 \cdot \dots \cdot f_k $ and $ f_1' \cdot \dots \cdot f_k' $ are homotopic. Let $ F : \I \times \I \to X $ be a homotopy. By compactness, there exist
$$ 0 = s_0 < \dots < s_m = 1, \qquad 0 = t_0 < \dots < t_n = 1, $$
such that $ R_{i, j} = \sbr{s_{i - 1}, s_i} \times \sbr{t_{j - 1}, t_j} $ and $ F\br{R_{ij}} \subseteq U_1 $ or $ F\br{R_{ij}} \subseteq U_2 $. May assume $ 0 = s_0 < \dots < s_m = 1 $ subdivides the products $ f_1 \cdot \dots \cdot f_k $ and $ f_1' \cdot \dots \cdot f_l' $. Relabel the $ R_{ij} $ to $ R_1, \dots, R_{mn} $.
$$
\begin{array}{|c|c|c|}
\hline
mn - m + 1 & \dots & mn \\
\hline
\vdots & \ddots & \vdots \\
\hline
1 & \dots & m \\
\hline
\end{array}
$$
A path $ \gamma $ in $ \I \times \I $ from left to right gives a loop $ \eval{F}_\gamma $ in $ X $ at $ x_0 $. Let $ \gamma_r $ be the path separating the first $ r $ rectangles from the others, so
$$ \eval{F}_{\gamma_0} \cong f_1 \cdot \dots \cdot f_k, \qquad \eval{F}_{\gamma_{mn}} = f_1' \cdot \dots \cdot f_l'. $$
Let $ v $ be a grid point. Choose a path $ g_v $ in $ X $ from $ x_0 $ to $ F\br{v} $, such that $ g_v $ is contained in $ U_1 \cap U_2 $ if $ F\br{v} \in U_1 \cap U_2 $ and in a single $ U_i $ otherwise. This gives us a factorisation of $ \sbr{\eval{F}_{\gamma_r}} $ into loops only contained in $ U_1 $ or $ U_2 $. The factorisations associated to $ \gamma_r $ and $ \gamma_{r + 1} $ are equivalent, because the homotopy between $ \eval{F}_{\gamma_r} $ and $ \eval{F}_{\gamma_{r + 1}} $ by pushing $ \gamma_r $ through $ R_r $ takes place within a single $ U_i $.
\end{proof}

\pagebreak

\subsection{The equivalence of simplicial and singular homology}

\begin{lemma}[Five lemma]
Consider the following diagram of abelian groups
$$
\begin{tikzcd}
A \arrow{r}{i} \arrow{d}{\alpha} & B \arrow{r}{j} \arrow{d}{\beta} & C \arrow{r}{k} \arrow{d}{\gamma} & D \arrow{r}{l} \arrow{d}{\delta} & E \arrow{d}{\epsilon} \\
A \arrow{r}{i'} & B' \arrow{r}{j'} & C' \arrow{r}{k'} & D' \arrow{r}{l'} & E'
\end{tikzcd}.
$$
If the rows are exact and $ \alpha, \beta, \delta, \epsilon $ are isomorphisms, then $ \gamma $ is an isomorphism.
\end{lemma}

\begin{proof}
Enough to show
\begin{itemize}
\item if $ \beta $ and $ \delta $ are surjective and $ \epsilon $ is injective, then $ \gamma $ is surjective, and
\item if $ \beta $ and $ \delta $ are injective and $ \alpha $ is surjective, then $ \gamma $ is injective.
\end{itemize}
\end{proof}

Let $ n \ge 1 $. Then
$$ \H_n\br{\Delta^n, \partial\Delta^n} \cong \widetilde{\H_n}\br{\Delta^n / \partial\Delta^n} \cong \widetilde{\H_n}\br{\S^n} \cong \ZZ, $$
and $ \H_0\br{\Delta^0, \partial\Delta^0} \cong \ZZ $.

\begin{lemma}
\label{lemma}
$ \H_n\br{\Delta^n, \partial\Delta^n} $ is generated by the class of the cycle $ i_n : \Delta^n \to \Delta^n $.
\end{lemma}

\begin{proof}
$ i_n $ is a cycle. Induction on $ n $.
\begin{itemize}[leftmargin=2cm]
\item[$ n = 0 $.] $ \H_0\br{\Delta^0, \emptyset} $ is generated by $ \sbr{i_0} $.
\item[$ n - 1 \mapsto n $.] Let $ \Lambda \subseteq \partial\Delta^n $ be the union of all but one of the $ \br{n - 1} $-dimensional faces of $ \Delta^n $. Then $ \Delta^n $ strongly deformation retracts to $ \Lambda $, so
$$ \H_i\br{\Delta^n, \Lambda} = \H_i\br{\Lambda, \Lambda} = 0. $$
Long exact sequence of the triple $ \Lambda \subseteq \partial\Delta^n \subseteq \Delta^n $ implies that
$$
\begin{tikzcd}[column sep=small, row sep=tiny]
\dots \arrow{r} & \H_n\br{\Delta^n, \Lambda} \arrow{r} \arrow[cong]{d} & \H_n\br{\Delta^n, \partial\Delta^n} \arrow{r}{\sim} & \H_{n - 1}\br{\partial\Delta^n, \Lambda} \arrow{r} & \H_{n - 1}\br{\Delta^n, \Lambda} \arrow{r} \arrow[cong]{d} & \dots \\
& 0 & & & 0 &
\end{tikzcd}.
$$
Note that $ \partial\Delta^n / \Lambda $ is homeomorphic to $ \Delta^{n - 1} / \partial\Delta^{n - 1} $, which are good pairs, so
\begin{align*}
\H_n\br{\Delta^n, \partial\Delta^n}
& \cong \H_{n - 1}\br{\partial\Delta^n, \Lambda}
\cong \widetilde{\H_{n - 1}}\br{\partial\Delta^n / \Lambda} \\
& \cong \widetilde{\H_{n - 1}}\br{\Delta^{n - 1} / \partial\Delta^{n - 1}}
\cong \H_{n - 1}\br{\Delta^{n - 1}, \partial\Delta^{n - 1}}.
\end{align*}
One can check that $ \sbr{i_n} $ maps to $ \sbr{\pm i_{n - 1}} $ along these isomorphisms, so induction implies that $ \H_n\br{\Delta^n, \partial\Delta^n} $ is generated by $ \sbr{i_n} $.
\end{itemize}
\end{proof}

\pagebreak

Let $ X $ be a topological space with a $ \Delta $-complex structure, so there is a simplicial chain complex
$$ \dots \to \Delta_{n + 1}\br{X} \to \Delta_n\br{X} \to \Delta_{n - 1}\br{X} \to \dots. $$
Every simplicial chain complex can be viewed as a singular $ n $-chain, so we obtain an inclusion of chain complexes $ \Delta_\bullet\br{X} \to \C_\bullet\br{X} $.

\begin{theorem}
This inclusion of chain complexes induces an isomorphism $ \H_n^\Delta\br{X} \xrightarrow{\sim} \H_n\br{X} $ for all $ n $.
\end{theorem}

\begin{proof}
We only consider the case, where the $ \Delta $-complex structure on $ X $ is finite dimensional, that is $ \Delta_m\br{X} = 0 $ for all $ m > k $, and the maximal such $ k $ is $ \dim X $. Induction on $ k $, the dimension.
\begin{itemize}[leftmargin=2cm]
\item[$ k = 0 $.] $ \H_n^\Delta\br{X} \cong \H_n\br{X} $ for $ X $ points.
\item[$ k - 1 \mapsto k $.] Let $ X^l $ be the $ l $-skeleton of $ X $ consisting of all simplicies of dimension at most $ l $. Then $ \H_n^\Delta\br{X^k, X^{k - 1}} $ are the homology groups of the chain complex
$$ \dots \to \Delta_{k + 1}\br{X^k} / \Delta_{k + 1}\br{X^{k - 1}} \to \Delta_k\br{X^k} / \Delta_k\br{X^{k - 1}} \to \Delta_{k - 1}\br{X^k} / \Delta_{k - 1}\br{X^{k - 1}} \to \dots, $$
so
$$ \H_n^\Delta\br{X^k, X^{k - 1}} =
\begin{cases}
0 & n \ne k \\
\text{free abelian group with basis the $ k $-simplices of} \ X & n = k
\end{cases}.
$$
The short exact sequence of chain complexes
$$ 0 \to \Delta_n\br{X^{k - 1}} \to \Delta_n\br{X^k} \to \Delta_n\br{X^k} / \Delta_n\br{X^{k - 1}} \to 0 $$
gives a long exact sequence
$$
\begin{tikzcd}[column sep=tiny]
\ \arrow{r} & \H_{n + 1}^\Delta\br{X^k, X^{k - 1}} \arrow{r} \arrow{d}{\alpha} & \H_n^\Delta\br{X^{k - 1}} \arrow{r} \arrow{d}{\beta} & \H_n^\Delta\br{X^k} \arrow{r} \arrow{d}{\gamma} & \H_n^\Delta\br{X^k, X^{k - 1}} \arrow{r} \arrow{d}{\delta} & \H_{n - 1}^\Delta\br{X^{k - 1}} \arrow{d}{\epsilon} \arrow{r} & \ \\
\ \arrow{r} & \H_{n + 1}\br{X^k, X^{k - 1}} \arrow{r} & \H_n\br{X^{k - 1}} \arrow{r} & \H_n^\Delta\br{X^k} \arrow{r} & \H_n\br{X^k, X^{k - 1}} \arrow{r} & \H_{n - 1}\br{X^{k - 1}} \arrow{r} & \
\end{tikzcd},
$$
which commutes by naturality, where $ \beta $ and $ \epsilon $ are isomorphisms by induction. Consider the continuous map
$$ \Phi : \bigsqcup_\alpha \br{\Delta_\alpha^k, \partial\Delta_\alpha^k} \to \br{X^k, X^{k - 1}}. $$
This induces an isomorphism
$$ \H_n\br{X^k, X^{k - 1}} \cong \H_n\br{\bigsqcup_\alpha \Delta_\alpha^k, \bigsqcup_\alpha \partial\Delta^k} = \bigoplus_\alpha \H_n\br{\Delta_\alpha^k, \partial\Delta_\alpha^k}, $$
which is the free abelian group on $ i_{n\alpha} : \Delta_\alpha^n \to \Delta_\alpha^n $ by Lemma \ref{lemma}, so $ \alpha $ and $ \delta $ are isomorphisms. Thus five lemma implies that $ \gamma $ is an isomorphism.
\end{itemize}
\end{proof}

\end{document}