\documentclass{article}

% Packages

\usepackage{amssymb}
\usepackage{amsthm}
\usepackage[UKenglish]{babel}
\usepackage{enumitem}
\usepackage{fancyhdr}
\usepackage[margin=1in]{geometry}
\usepackage{graphicx}
\usepackage[utf8]{inputenc}
\usepackage{listings}
\usepackage{mathtools}
\usepackage{stmaryrd}
\usepackage{tikz-cd}
\usepackage{csquotes}

% Formatting

\setlength{\parindent}{0cm}
\addto\captionsUKenglish{\renewcommand{\abstractname}{Syllabus}}

% Environments

\ifx \thm\undefined
  \newtheorem{n}{}
\else
  \newtheorem{n}{}[\thm]
\fi

\theoremstyle{plain}
\newtheorem{algorithm}[n]{Algorithm}
\newtheorem{conjecture}[n]{Conjecture}
\newtheorem{corollary}[n]{Corollary}
\newtheorem{lemma}[n]{Lemma}
\newtheorem{proposition}[n]{Proposition}
\newtheorem{theorem}[n]{Theorem}

\theoremstyle{definition}
\newtheorem{axiom}{Axiom}
\newtheorem{definition}[n]{Definition}
\newtheorem{example}[n]{Example}
\newtheorem{exercise}{Exercise}
\newtheorem*{definition*}{Definition}
\newtheorem*{example*}{Example}

\theoremstyle{remark}
\newtheorem{notation}[n]{Notation}
\newtheorem{remark}[n]{Remark}
\newtheorem*{fact*}{Fact}
\newtheorem*{notation*}{Notation}
\newtheorem*{note*}{Note}
\newtheorem*{remark*}{Remark}

% Macros

\newcommand{\lecture}[3]{
  \marginpar{
    Lecture #1 \\
    #2
    #3
  }
}

\newcommand{\function}[5][]{
  \ifx &#1&
    \begin{array}{rcl}
      #2 & \to     & #3 \\
      #4 & \mapsto & #5
    \end{array}
  \else
    \begin{array}{crcl}
      #1 : & #2 & \to     & #3 \\
           & #4 & \mapsto & #5
    \end{array}
  \fi
}

\newcommand{\correspondence}[2]{
  \cb{
    \begin{array}{c}
      #1
    \end{array}
  }
  \qquad
  \leftrightsquigarrow
  \qquad
  \cb{
    \begin{array}{c}
      #2
    \end{array}
  }
}

\newcommand{\onebytwo}[2]{
  \begin{pmatrix}
    #1 & #2
  \end{pmatrix}
}

\newcommand{\onebythree}[3]{
  \begin{pmatrix}
    #1 & #2 & #3
  \end{pmatrix}
}

\newcommand{\twobyone}[2]{
  \begin{pmatrix}
    #1 \\
    #2
  \end{pmatrix}
}

\newcommand{\twobytwo}[4]{
  \begin{pmatrix}
    #1 & #2 \\
    #3 & #4
  \end{pmatrix}
}

\newcommand{\threebyone}[3]{
  \begin{pmatrix}
    #1 \\
    #2 \\
    #3
  \end{pmatrix}
}

\newcommand{\threebythree}[9]{
  \begin{pmatrix}
    #1 & #2 & #3 \\
    #4 & #5 & #6 \\
    #7 & #8 & #9
  \end{pmatrix}
}

% Commands

\newcommand{\ab}[1]{\left\langle #1 \right\rangle} % Angle brackets
\newcommand{\cb}[1]{\left\{ #1 \right\}}           % Curly brackets
\newcommand{\fb}[1]{\left\lfloor #1 \right\rfloor} % Floor brackets
\newcommand{\rb}[1]{\left( #1 \right)}             % Round brackets
\renewcommand{\sb}[1]{\left[ #1 \right]}           % Square brackets
\newcommand{\abs}[1]{\left\lvert #1 \right\rvert}  % Absolute brackets

\newcommand{\F}{\mathbb{F}}   % Finite fields
\newcommand{\N}{\mathbb{N}}   % Natural numbers
\newcommand{\Z}{\mathbb{Z}}   % Integral numbers
\newcommand{\Q}{\mathbb{Q}}   % Rational numbers
\newcommand{\R}{\mathbb{R}}   % Real numbers
\newcommand{\C}{\mathbb{C}}   % Complex numbers
\renewcommand{\H}{\mathbb{H}} % Quaternion numbers
\newcommand{\A}{\mathbb{A}}   % Affine space
\renewcommand{\P}{\mathbb{P}} % Projective space

\renewcommand{\aa}{\mathfrak{a}} % Ideal 1
\newcommand{\bb}{\mathfrak{b}}   % Ideal 2
\newcommand{\mm}{\mathfrak{m}}   % Maximal ideal 1
\newcommand{\nn}{\mathfrak{n}}   % Maximal ideal 2
\newcommand{\pp}{\mathfrak{p}}   % Prime ideal 1
\newcommand{\qq}{\mathfrak{q}}   % Prime ideal 2

\renewcommand{\AA}{\mathcal{A}} % Calligraphic A
\newcommand{\BB}{\mathcal{B}}   % Calligraphic B
\newcommand{\CC}{\mathcal{C}}   % Calligraphic C
\newcommand{\DD}{\mathcal{D}}   % Calligraphic D
\newcommand{\EE}{\mathcal{E}}   % Calligraphic E
\newcommand{\FF}{\mathcal{F}}   % Calligraphic F
\newcommand{\GG}{\mathcal{G}}   % Calligraphic G
\newcommand{\HH}{\mathcal{H}}   % Calligraphic H
\newcommand{\II}{\mathcal{I}}   % Calligraphic I
\newcommand{\JJ}{\mathcal{J}}   % Calligraphic J
\newcommand{\KK}{\mathcal{K}}   % Calligraphic K
\newcommand{\LL}{\mathcal{L}}   % Calligraphic L
\newcommand{\MM}{\mathcal{M}}   % Calligraphic M
\newcommand{\NN}{\mathcal{N}}   % Calligraphic N
\newcommand{\OO}{\mathcal{O}}   % Calligraphic O
\newcommand{\PP}{\mathcal{P}}   % Calligraphic P
\newcommand{\QQ}{\mathcal{Q}}   % Calligraphic Q
\newcommand{\RR}{\mathcal{R}}   % Calligraphic R
\renewcommand{\SS}{\mathcal{S}} % Calligraphic S
\newcommand{\TT}{\mathcal{T}}   % Calligraphic T
\newcommand{\UU}{\mathcal{U}}   % Calligraphic U
\newcommand{\VV}{\mathcal{V}}   % Calligraphic V
\newcommand{\WW}{\mathcal{W}}   % Calligraphic W
\newcommand{\XX}{\mathcal{X}}   % Calligraphic X
\newcommand{\YY}{\mathcal{Y}}   % Calligraphic Y
\newcommand{\ZZ}{\mathcal{Z}}   % Calligraphic Z

\newcommand{\notb}[1]{\rb{\neg #1}}               % Negation
\newcommand{\orb}[2]{\rb{#1 \lor #2}}             % Disjunction
\newcommand{\andb}[2]{\rb{#1 \land #2}}           % Conjunction
\newcommand{\impb}[2]{\rb{#1 \rightarrow #2}}     % Implication
\newcommand{\iffb}[2]{\rb{#1 \leftrightarrow #2}} % Biconditional
\newcommand{\fab}[1]{\rb{\forall #1}}             % Universal quantifier
\newcommand{\teb}[1]{\rb{\exists #1}}             % Existential quantifier
\newcommand{\eqb}[2]{\rb{#1 = #2}}                % Equal to
\newcommand{\ltb}[2]{\rb{#1 < #2}}                % Less than
\newcommand{\leb}[2]{\rb{#1 \le #2}}              % Less than or equal to
\newcommand{\neb}[2]{\rb{#1 \ne #2}}              % Not equal to
\newcommand{\inb}[2]{\rb{#1 \in #2}}              % An element of
\newcommand{\nib}[2]{\rb{#1 \notin #2}}           % Not an element of
\newcommand{\subb}[2]{\rb{#1 \subseteq #2}}       % Subset of

\newcommand{\jacobi}[2]{\rb{\tfrac{#1}{#2}}}        % Jacobi symbol
\newcommand{\Unit}[1]{\rb{\dfrac{\Z}{#1\Z}}^\times} % Unit group vertical
\newcommand{\unit}[1]{\rb{\Z / #1\Z}^\times}        % Unit group horizontal

% Tikz

\tikzset{
  arrow symbol/.style={"#1" description, allow upside down, auto=false, draw=none, sloped},
  subset/.style={arrow symbol={\subset}},
  cong/.style={arrow symbol={\cong}}
}

% Fancy header

\pagestyle{fancy}
\lhead{\module}
\rhead{\nouppercase{\leftmark}}

% Make title

\title{\module}
\author{Lectured by \lecturer \\ Typed by David Kurniadi Angdinata}
\date{\term}